\chapter[Model updates]{Updates to the EMEP MSC-W model, 2020--2021}
\label{ch:ModelUpdates}

%% authors? added all names seen in Log.changes

{\bf{David Simpson, Michael Gauss, Qing Mu, Svetlana Tsyro and Peter Wind}}
\vspace{30pt}

%\old{TODO}
\COMMENT{STARTED Aug 2021. Needs to be filled in by authors}

This chapter summarises the changes made to the EMEP MSC-W  model
since \citet{R2020:ModDev}, and along with changes discussed in
\citet{R2013:ModDev,R2015:ModDev,R2016:ModDev,R2017:ModDev,R2019:ModDev,R2020:ModDev} and
\citet{R2014:ModDev},
updates the standard description given in \citet{Simpson_et_al:EMEP}. The
model version used for reporting this year is denoted rv4.42, which has
had some major updates (especially with regard to emissions) compared to
the rv4.35 reported in \citet{R2020:ModDev}.
Table~\ref{tab:Updates} summarises
the changes made in the EMEP model since the version documented in
\citet{Simpson_et_al:EMEP}, and these changes are discussed in
more detail in Sects.~\ref{sec:updateOverview}-\ref{sec:updateOther}.

%Versions:
%
%  rv4.35 used  for R2020
%  rv4.33 used  for R2019
%  rv4.33 - open source June 2019
%  rv4.32 used for EMEP course, April 2019
%  rv4.17a used for R2018 runs  (July-ish?)
%  rv4.17 released 26/2/2018
%  rv4.16 interim 21/12/2017 - used for N2O5 paper, wheat calculations
%  rv4.15 released 8/9/2017

\section{Overview of changes} 
\label{sec:updateOverview}

%%%%%%%%%%%%%%%%%%%%%%%%%%%%%%%%%%%%%%%%%%%%%%%%%%%%%%%%%%%%%%%%%%%%%%%%%%%%%
\begin{table}
\begin{footnotesize}
\caption{Summary of major EMEP MSC-W model versions from 2012--2020.
Extends Table S1 of \citealt{Simpson_et_al:EMEP}
  COMMENT: Note - recent changes and tags are a little confused. Dave will fix.
}
\label{tab:Updates}
\centering
\begin{tabular}{lp{11cm}l}
\hline
Version & Update                                        & Ref$^{(a)}$   \\
\hline
rv4.42  & Changed default Kz scheme &  This report        \\
        & Updated emissions from CAMS81 project: DMS, soil NO;  &  This report        \\
% 21st May 2021:
rv4.41  & Updated emissions from CAMS81 project: DMS, soil NO; % 16/5
          Modified various parameters concerning sea-salt and wet-radius calculations; % 16/5
   fracPM25 modified;
          Revised global monthly factor;  % 16/5
        & \\
% 13/5
rv4.40  & Shipping emissions now spread as 20\% below 20m, 80\% between 20-90m;  
          New CAMS81-based aircraft emissions; % 29-30/4
          New CAMS81-based DMS emissions; % 29/4
        & \\
%SKIP BUG fix on aircraft emissions  27/4
% 21/4
rv4.39 &
  GNFR-CAMS is now the default emissions system - see Sect.~\ref{ssec:gnfr};  % 21/4
          Uses global time-zone map;   % 19/3
          Revised global monthly factor?? or rv4.41?;  % 19/3
          New default Kz method introduced (Sect.~\ref{ssec:PBL});
          NWP model rh2m now used in place of earlier sub-grid calculation; % 9/2
          Upgrade of local fractions methodology;  % 7/11/2020
        & \\
% 23/10/2021:
rv4.35  & Various updates, including heavy   
          refactoring of local-fraction code, bug-fixes in MARS module,
          and updates in chemical mechanisms, default PM and NMVOC speciation and
          GenChem systems     & R2020            \\
rv4.34  & Public domain (Feb. 2020); EmChem19a, EmChem19p      & R2020            \\
rv4.33  & Public domain (June 2019);
         EmChem19, PAR bug-fix, EQSAM4clim    & R2019            \\
rv4.32  & Used for EMEP course, April 2019    &    \\
rv4.30  & Moved to new GenChem-based system  &   \\
%        &                                    &        \\
rv4.17a & Used for R2018. Small updates         & R2018      \\
rv4.17  & Public domain (Feb. 2018);
         Corrections in global land-cover/deserts; added
          'LOTOS' option for European \ce{NH3} emissions; corrections
          to snow cover & R2018 \\
rv4.16  & New radiation scheme (Weiss\&Norman); Added dry and wet deposition for \ce{N2O5};
         (Used for  \citealt{Stadtler2018,MillsGCB2018b}) & R2018   \\
rv4.15  & EmChem16 scheme & R2017 \\
%    Sect.\ref{sec:GNFR}--\ref{ss:Splits} & R2016  \\
%
rv4.14  & Updated chemical scheme & R2017       \\
%% rv4.13 + CRI was used in McFiggans. Difficult to describe combo
%        & \\
rv4.12  & New  global land-cover and BVOC & R2017       \\
%        & \\
rv4.10  &  Public domain (Oct. 2016)                 
         (Used for  \citealt{MillsGCB2018a}) &  R2016 \\
%        & \\
rv4.9   & Updates for GNFR sectors, DMS, sea-salt, dust, \ce{S_A} and  $\gamma$, \ce{N2O5} & \\ 
rv4.8   &  Public domain (Oct. 2015); ShipNOx introduced.                          
         Used for EMEP HTAP2 model calculations, see
         special issue:
         \url{www.atmos-chem-phys.net/special_issue390.html},
          and \citet{Jonson_et_al:2017}.              & R2015\\
rv4.7   & Used for reporting, summer 2015;
         New calculations of aerosol surface area; 
         New gas-aerosol uptake and \ce{N2O5} hydrolysis rates; 
         Added 3-D calculations pf aerosol extinction and AODs;
         Emissions - new flexible mechanisms for interpolation and merging sources;
         Global - monthly emissions from ECLIPSE project;
         Global -  LAI changes from LPJ-GUESS model;
         WRF meteorology \citep{SkamarockKlemp2008} can now
     be used directly in EMEP model. & R2015 \\
%        &                                                &\\
rv4.6   & Used for Euro-Delta SOA runs                   & R2015  \\
%QUERY        & Bug-fix for ammonium deposition & \\
       & Revised boundary condition treatments % & \\  % Vertical profiles
       ; ISORROPIA capability added & \\
%       &                                                &\\
rv4.5  & Sixth open-source (Sep 2014);                    
        Improved dust, sea-salt, SOA modelling          % &      \\
       ; AOD and extinction coefficient calculations  updated %& \\
       ; Data assimilation system added % & \\
       ; Hybrid vertical coordinates replace earlier sigma % & \\
       ; Flexibility of grid projection increased. & R2014\\
%SKIP        & ?? Point sources, plume rise, data-assimilation\\
%       &                                                &\\
rv4.4   & Fifth open-source (Sep 2013) %
       ; Improved dust and sea-salt modelling   %          &      \\
       ; AOD and extinction coefficient calculations added %  &\\
       ; gfortran compatibility improved            %      &      \\
                  & R2014, R2013\\
%       &                                                &\\
rv4.3   & Fourth public domain (Mar. 2013)  %
       ; Initial use of namelists           %            & \\
       ; Smoothing of MARS results         %            & \\
       ; Emergency module for volcanic ash and other events% & \\
       ; Dust and road-dust options added as defaults % & \\
       ; Advection algorithm changed  % & \\ % \citet{CLAPP98}    & \\
             & R2013\\ 
%        &                                                &\\
rv4.0   & Third public domain (Sep. 2012), as \citet{Simpson_et_al:EMEP}            & R2013\\ 
%        & As documented in \citet{Simpson_et_al:EMEP}    & \\
%v2011-06& Second public domain (Aug. 2011)                &\\ 
%rv3     & First public domain (Sep. 2008)                &\\ 
        &                                                &\\
\hline
\end{tabular}
Notes: (a) R2018 refers to EMEP Status report 1/2018, etc.
\end{footnotesize}
\end{table}
%%%%%%%%%%%%%%%%%%%%%%%%%%%%%%%%%%%%%%%%%%%%%%%%%%%%%%%%%%%%%%%%%%%%%%%%%%%%%

\begin{itemize}

\item Introduced new soil-NO, DMS and aircraft emissions, from CAMS-81 project
(Sect.~\ref{sec:updateEmis}).  -- Dave, Michael

\item Revised methods for vertical diffusion (Kz) \QUERY{and Hmix?} (Sect.~\ref{sec:updateKz}). -- Qing

\item Modified fine/coarse split of sea-salt (Sect.~\ref{ssec:updateSS}).

\item
Updated Local Fractions capabilities (Sect.~\ref{sec:updateLF}). -- Peter

\item  fracPM25 stuff (Sect.~\ref{sec:updateOther}). -- Dave
\item revised wet-radius calculations (Sect.~\ref{sec:updateOther}). --Dave

%\item The EMEP model's chemical pre-processing system and associated box-model (GenChem, boxChem) have been released as open-source.  See Sect.~\ref{sec:GChem}.

\item
Emissions speciation. New default and country-specific emission
speciations for NMVOC and \pmfine have been implemented.  See
Sect.~\ref{ssec:emissplits}.

\item
Numerous small changes to make the code more flexible and/or to
fix minor bugs.

\end{itemize}

In addition to these changes, articles on GenChem (EMEP's chemical pre-processing system)  and
the local-scale extension \COMMENT{Phrasing?} have now been published \citep{Simpson:GenChem,Denby:2020}

%%%%%%%%%%%%%%%%%%%%%%%%%%%%%%%%%%%%%%%%%%%%%%%%%%%%%%%%%%%%%%%%%%%%%%%%%%%%%%
\section{New emission inputs}
\label{sec:updateEmis}

From CAMS81...

\subsection{New basis for emission sectors: GNFR\_CAMS}
\label{ssec:gnfr}

\COMMENT{Text/Table from gmdPaper}
 
Gridded anthropogenic emissions from CEIP
were previously categorized into 11
SNAP sectors, but for many years now EMEP emission reporting has been conducted and prepared for
modelling using the 13-sector
GNFR system\footnote{
GNFR=Gridding nomenclature for reporting/UNECE nomenclature for reporting of emissions to air,
e.g. \citealt{CEIP2020:GNFR}}.
In 2020 a 19-sector emission system (`GNFR\_CAMS') was implemented in the EMEP model, to take care of
emissions provided by TNO as part of the Copernicus CAMS project \citep{Kuenen2021}. This extended emissions system
enables for example four road traffic sectors, F1--F4, with e.g. F1
 representing exhaust emissions from gasoline vehicles.
Such emission sectors are characterised in the model by
release heights, timefactors and species-splits (e.g. NOx to NO and
\ce{NO2}, or NMVOC to individual VOC surrogates) for each sector.

Table~\ref{tab:GNFRsectors} summarises the new GNFR\_CAMS sectors,
and the mapping indices used.
%\COMMENT{Add index explanation in SI table?}


\begin{table}
\caption{The `GNFR\_CAMS' 19-sector system, and mapping indices for release heights, timefactors and species-splits,
 which is now
 default in EMEP model. \label{tab:GNFRsectors}
}
\begin{tabular}{lrrrrlc}
\hline
code & Number  & \multicolumn{3}{c}{Index}      & sector & SNAP \\ \cline{3-5}
     &         & timefac  & height & emissplit  &        & equivalent \\
\hline
A &      01 &   1 &  1 &   1 &  Public Power  &  1 \\
B &      02 &   3 &  3 &   2 &  Industry  & 3, 4 \\
C &      03 &   2 &  2 &   3 &  Other Stationary Combustion & 2 \\
D &      04 &   4 &  4 &   4 &  Fugitive & ??  \\
E &      05 &   6 &  2 &   5 &  Solvents & 6 \\
F &      06 &   7 &  2 &   6 &  Road Transport & 7 \\
G &      07 &   8 &  2 &   7 &  Shipping & 8  \\
H &      08 &   8 &  7 &   8 &  Aviation & 8  \\
I &      09 &   8 &  2 &   9 &  Offroad & 8  \\
J &     10 &   9 &  6 &  10 &  Waste & ? \\
K &     11 &  10 &  2 &  11 &  Agri - Livestock & 10 \\
L &     12 &  10 &  2 &  12 &  Agri - Other & 10 \\
M &     13 &   5 &  5 &  13 &  Other & 11 \\
A1 &    14 &   1 &  1 &   1 &  PublicPower - Point & 1 \\
A2 &    15 &   1 &  3 &   1 &  PublicPower - Area & 1 \\
F1 &    16 &   7 &  2 &  16 &  Road Transport - Exhaust Gasoline & 7  \\
F2 &    17 &   7 &  2 &  17 &  Road Transport - Exhaust Diesel& 7   \\
F3 &    18 &   7 &  2 &  18 &  Road Transport - Exhaust LPGgas & 7  \\
F4 &    19 &   7 &  2 &  19 &  Road Transport - NonExhaust Other& 7   \\
\hline
\end{tabular}
\end{table}



\subsection{Soil NO emissions}
\label{ssec:soilNO}

\COMMENT{DAVE to COMPLETE}

\subsection{DMS emissions}
\label{ssec:DMS}

\COMMENT{MICHAEL to COMPLETE}

\subsection{Aircraft emissions}
\label{ssec:Aircraft}

\COMMENT{MICHAEL to COMPLETE}


\subsection{Revised fine/coarse splits of sea-salt emissions}
\label{ssec:updateSS}

The split of sea salt between the fine and coarse fractions have been revised based on the results of comparison of modelled \chem{Na+} in \PM[2.5] and \PM[10] with EMEP observations. Namely, the fraction of sea salt emitted in \PM[2.5] fraction has been slightly increased (while the total sea salt emissions in \PM[10] remain unchanged). 

More specifically, sea salt is emitted in seven size bins up to 10 \um in diameter, from which the smallest four bins were previously assigned to the \PM[2.5] fraction and the remaining three bins comprised coarse sea salt. Now, the fifth bin is split 50/50 \% between the \PM[2.5] and \PM[2.5-10] fractions.  


This modification has a minor effect on formation of coarse \noiii as the surface area of coarse sea salt has slightly decreased.
\COMMENT{SVETLANA to COMPLETE}



%%%%%%%%%%%%%%%%%%%%%%%%%%%%%%%%%%%%%%%%%%%%%%%%%%%%%%%%%%%%%%%%%%%%%%%%%%%%%%
\subsection{Emission speciation}
\label{ssec:emissplits}.

\COMMENT{OLD:}

The emissions speciation of NMVOC and primary PM (PPM) were updated
for the EmChem19a scheme to reflect recent data available from
the latest TNO/CAMS inventories (see chap.~\ref{chap:TNO} 
\COMMENT{Dave will sort out cross-refs}
, also
\citealt{CAMSemis2019}).  For NMVOC the main changes have been:

\begin{enumerate}
  \item Use TNO NMVOC speciation

    TNO provided NMVOC speciation data for 25 compounds from each
    GNFR\_CAMS category, as part of the CAMS-REG-v3.1.2 database,
    which were then mapped to the EmChem19a species.

  \item Improve country-specific road transport estimates

    TNO provided country-specific fractions of the four road-traffic
    sectors in the CAMS-REG-AP\_v2.2.1 inventory for 2015, specifically
    F1=gasoline exhaust, F2=diesel exchaust, F3=LPG exhaust, and
    F4=non-exhaust emissions. These were also aggregated to form
    country-specific NMVOC splits for the generic GNFR F category
    (road transport).

\end{enumerate}

% PM TNO CAMS-REG_AP_v2.2.1_2015_REF

For \pmfine and \pmten, we made use of country-data generated by TNO for
the TFMM Euro-DeltaCarb simulations, which gave EC, OM (named
OC in the files), Na, SO4 and other compounds, as well as the fraction
of modern carbon in the GNFR-C categories. These data were aggregated
to the EMEP models EC, OM, and remPPM compounds. Data were provided for
both the Ref1 and Ref2 cases (see Chap.~\ref{chap:TNO}), and we assume
that Ref1 data are appropriate for modelling with officially submitted
emissions, and Ref2 data appropriate for modelling when Ref2 emissions
are used (e.g. for the \textbf{EMEPwRef2C} simulations discussed in
Chapters~\ref{ch:chapterStatus}--\ref{ch:Condensables}).
\COMMENT{Dave will sort out text later}



%%%%%%%%%%%%%%%%%%%%%%%%%%%%%%%%%%%%%%%%%%%%%%%%%%%%%%%%%%%%%%%%%%%%%%%%%%%%%%
\section{Revised PBL parametrisations}
\label{sec:updateKz}

\COMMENT{QING to UPDATE}

%%%%%%%%%%%%%%%%%%%%%%%%%%%%%%%%%%%%%%%%%%%%%%%%%%%%%%%%%%%%%%%%%%%%%%%%%%%%%%
\section{Local Fractions}
\label{sec:updateLF}
The range of applications of the Local Fractions has been extended. 

The sources can be defined by either a list of countries or group of countries, as small squares surrounding the receptor grid (with user defined square sizes), or as regions defined as "masks" given in a NetCDF file.

Wet and dry deposition (separately) per source country can now be computed.

\COMMENT{PETER to UPDATE}

%%%%%%%%%%%%%%%%%%%%%%%%%%%%%%%%%%%%%%%%%%%%%%%%%%%%%%%%%%%%%%%%%%%%%%%%%%%%%%
\section{Other?}
\label{sec:updateOther}

fracMPM25 - Dave
revised wet-radius calculations --Dave

Something on pollen perhaps? -- Alvaro??



\clearpage
\bibliographystyle{copernicus}         % change bibliography-name after each
\renewcommand\bibname{References}      % bibliographystyle command!
\addcontentsline{toc}{section}{References}
\bibliography{Refs,EMEP_Reports,uEMEP}
