\chapter[Model updates]{Updates to the EMEP MSC-W model, 2020--2021}
\label{ch:ModelUpdates}

%% authors? added all names seen in Log.changes

{\bf{David Simpson, Robert Bergstr\"om, Svetlana Tsyro and Peter Wind}}
\vspace{30pt}

\old{TODO}

This chapter summarises the changes made to the EMEP MSC-W  model
since \citet{R2019:ModDev}, and along with changes discussed in
\citet{R2013:ModDev,R2014:ModDev,R2015:ModDev,R2016:ModDev,R2017:ModDev,R2019:ModDev},
updates the standard description given in \citet{Simpson_et_al:EMEP}. The
model version used for reporting this year is denoted rv4.35, which has
had some minor updates in the basic model since rv4.33 as documented
in  \citet{R2019:ModDev}, major updates in the `Local Fraction'
methodology, and also in the emissions underlying some of the model runs. Table~\ref{tab:Updates} summarises
the changes made in the EMEP model since the version documented in
\citet{Simpson_et_al:EMEP}, and Tables~\ref{tab:EMEPrun}--\ref{tab:EMEPwRef2C_oldVOC} compare the impacts of some of these changes.

%Versions:
%
%  rv4.35 used  for R2020
%  rv4.33 used  for R2019
%  rv4.33 - open source June 2019
%  rv4.32 used for EMEP course, April 2019
%  rv4.17a used for R2018 runs  (July-ish?)
%  rv4.17 released 26/2/2018
%  rv4.16 interim 21/12/2017 - used for N2O5 paper, wheat calculations
%  rv4.15 released 8/9/2017

\section{Overview of changes} 

%%%%%%%%%%%%%%%%%%%%%%%%%%%%%%%%%%%%%%%%%%%%%%%%%%%%%%%%%%%%%%%%%%%%%%%%%%%%%
\begin{table}
\begin{footnotesize}
\caption{Summary of major EMEP MSC-W model versions from 2012--2020.
Extends Table S1 of \citealt{Simpson_et_al:EMEP}}
\label{tab:Updates}
\centering
\begin{tabular}{lp{11cm}l}
\hline
Version & Update                                        & Ref$^{(a)}$   \\
\hline
        &                                               & \\
rv4.35  & Various updates, including heavy   
          refactoring of local-fraction code, bug-fixes in MARS module,
          and updates in chemical mechanisms, default PM and NMVOC speciation and
          GenChem systems     & This report      \\
rv4.34  & Public domain (Feb. 2020); EmChem19a, EmChem19p      & This report      \\
rv4.33  & Public domain (June 2019);
         EmChem19, PAR bug-fix, EQSAM4clim    & R2019            \\
rv4.32  & Used for EMEP course, April 2019    &    \\
rv4.30  & Moved to new GenChem-based system  &   \\
%        &                                    &        \\
rv4.17a & Used for R2018. Small updates         & R2018      \\
rv4.17  & Public domain (Feb. 2018);
         Corrections in global land-cover/deserts; added
          'LOTOS' option for European \ce{NH3} emissions; corrections
          to snow cover & R2018 \\
rv4.16  & New radiation scheme (Weiss\&Norman); Added dry and wet deposition for \ce{N2O5};
         (Used for  \citealt{Stadtler2018,MillsGCB2018b}) & R2018   \\
rv4.15  & EmChem16 scheme & R2017 \\
%    Sect.\ref{sec:GNFR}--\ref{ss:Splits} & R2016  \\
%
rv4.14  & Updated chemical scheme & R2017       \\
%% rv4.13 + CRI was used in McFiggans. Difficult to describe combo
%        & \\
rv4.12  & New  global land-cover and BVOC & R2017       \\
%        & \\
rv4.10  &  Public domain (Oct. 2016)                 
         (Used for  \citealt{MillsGCB2018a}) &  R2016 \\
%        & \\
rv4.9   & Updates for GNFR sectors, DMS, sea-salt, dust, \ce{S_A} and  $\gamma$, \ce{N2O5} & \\ 
rv4.8   &  Public domain (Oct. 2015); ShipNOx introduced.                          
         Used for EMEP HTAP2 model calculations, see
         special issue:
         \url{www.atmos-chem-phys.net/special_issue390.html},
          and \citet{Jonson_et_al:2017}.              & R2015\\
rv4.7   & Used for reporting, summer 2015;
         New calculations of aerosol surface area; 
         New gas-aerosol uptake and \ce{N2O5} hydrolysis rates; 
         Added 3-D calculations pf aerosol extinction and AODs;
         Emissions - new flexible mechanisms for interpolation and merging sources;
         Global - monthly emissions from ECLIPSE project;
         Global -  LAI changes from LPJ-GUESS model;
         WRF meteorology \citep{SkamarockKlemp2008} can now
     be used directly in EMEP model. & R2015 \\
%        &                                                &\\
rv4.6   & Used for Euro-Delta SOA runs                   & R2015  \\
%QUERY        & Bug-fix for ammonium deposition & \\
       & Revised boundary condition treatments % & \\  % Vertical profiles
       ; ISORROPIA capability added & \\
%       &                                                &\\
rv4.5  & Sixth open-source (Sep 2014);                    
        Improved dust, sea-salt, SOA modelling          % &      \\
       ; AOD and extinction coefficient calculations  updated %& \\
       ; Data assimilation system added % & \\
       ; Hybrid vertical coordinates replace earlier sigma % & \\
       ; Flexibility of grid projection increased. & R2014\\
%SKIP        & ?? Point sources, plume rise, data-assimilation\\
%       &                                                &\\
rv4.4   & Fifth open-source (Sep 2013) %
       ; Improved dust and sea-salt modelling   %          &      \\
       ; AOD and extinction coefficient calculations added %  &\\
       ; gfortran compatibility improved            %      &      \\
                  & R2014, R2013\\
%       &                                                &\\
rv4.3   & Fourth public domain (Mar. 2013)  %
       ; Initial use of namelists           %            & \\
       ; Smoothing of MARS results         %            & \\
       ; Emergency module for volcanic ash and other events% & \\
       ; Dust and road-dust options added as defaults % & \\
       ; Advection algorithm changed  % & \\ % \citet{CLAPP98}    & \\
             & R2013\\ 
%        &                                                &\\
rv4.0   & Third public domain (Sep. 2012), as \citet{Simpson_et_al:EMEP}            & R2013\\ 
%        & As documented in \citet{Simpson_et_al:EMEP}    & \\
%v2011-06& Second public domain (Aug. 2011)                &\\ 
%rv3     & First public domain (Sep. 2008)                &\\ 
        &                                                &\\
\hline
\end{tabular}
Notes: (a) R2018 refers to EMEP Status report 1/2018, etc.
\end{footnotesize}
\end{table}
%%%%%%%%%%%%%%%%%%%%%%%%%%%%%%%%%%%%%%%%%%%%%%%%%%%%%%%%%%%%%%%%%%%%%%%%%%%%%

\begin{itemize}

\item
Local Fractions: the Local Fractions method allows to track a large number of primary
emission sources effectively. The details of the methodology are described
in ~\cite{wind-2020}. The method has been generalized and can be used
to track emissions from a list of countries.

\item
The EMEP model's chemical pre-processing system and associated
box-model (GenChem, boxChem) have been released as open-source.
See Sect.~\ref{sec:GChem}.

\item
Chemical mechanisms: the three chemical mechanisms introduced
in \citet{R2019:ModDev}, EmChem19, CRI v2.2-emep, and CB6r2 have
been updated for greater compatability and debugged, and are now
denoted EmChem19a, CRIv2R5Em, CB6r2Em and EmChem19X \citep[see][in
preparation]{BergstromEmChem2020}.  Additionally, EmChem19p, which is
just EmChem19a plus code for four pollen species, has been introduced
to the open-source code.  See Sect.~\ref{sec:Chems}.

\item
A 19-sector emission system, GNFR\_CAMS, was introduced to take care of
emissions provided by TNO as part of CAMS. This extended emissions system
enables for example four road traffic sectors, F1--F4, with e.g. F1
 representing exhaust emissions from gasoline vehicles.

\item
Emissions speciation. New default and country-specific emission
speciations for NMVOC and \pmfine have been implemented.  See
Sect.~\ref{sec:emissplits}.

\item
Numerous small changes to make the code more flexible and/or to
fix minor bugs.

\end{itemize}

\section{Local Fraction}
\label{sec:LFrac}

The implementation of the method has been streamlined and generalized. New
features include tracking of pollutants over the entire domain
(the position of the source is then defined in a coarser resolution),
tracking of pollutants by country of origin, tracking of single species,
tracking of reduced nitrogen, tracking of sulfates and dry deposition. The
development of secondary pollutants is still at an experimental stage.

The method is efficient and the cost of tracking pollutants from 100
countries in a single run is negligible (less than 5\% additional
runtime).

This new approach allows to compute an entire SR matrix
for primary components (and with some preliminary
implementation of groups such as reduced N). 
%
%PW answer: the method is potentially more general. For now reduced N
%is for example implemented, (as explained above), and not considered
%as primary. The simple red N equilibrium chemistry is fully taken into
%account. For SOx, only some chemistry is taken into account. I do not
%want to be to detailed, as we do not present result for those, and it
%is under development} in a single
%
The interpretation of the contributions is slightly different than
the results obtained by traditional scenario runs. The Local Fraction
approach is essentially a tracking mechanism, and therefore only the first
order effects of the change of emissions will be reflected. For primary
particulate matter the differences between the two approaches are small.

It should be noted that primary particulate matter will have an influence
on SOA through gas/particle partitioning impacts \citep{Bergstrom2012},
and also on gas-particle reaction rates which are dependent on aerosol
surface area \citep{Stadtler2018} and thereby on the total particulate
matter concentrations. These secondary effects are not included in the
Local Fraction method.

Another more subtle difference, is the effect on advection. In a
scenario run, the concentrations of air pollutants will change. The
4th order Bott's scheme \citep{Bott1989a} uses those concentrations
to tune the advection parameters; the parameters are changed in a way
that tend to keep the pollutants together (in order to reduce numerical
diffusion). This spurious effect is not present in the Local Fraction
method.

%\COMMENT{TABLE NUMBERS NEED FINAL CHECK. HARD CODED!!  (PW checked 14 Aug) \bigskip}

\begin{figure}%[ht!]
\centering
\includegraphics*[width=0.9\linewidth]{FIGS_MOD/PPM_EMERwRef2C_LF_bruteforce_C2C.png}\\
\includegraphics*[width=0.9\linewidth]{FIGS_MOD/PPM_EMERwRef2C_LF_bruteforce_IMPORT2C.png}\\

\caption{Country to country contribution (C2C) and import from all other countries (IMPORT2C) for PPM$_{2.5}$ using the brute force methodology and the local fraction methodology (LF), both with EMEPwRef2C emissions, see chapter~\ref{ch:emis2018}.Units: ngm$^{-3}$
\label{fig:LF_SR}
}
\end{figure}

Table~\ref{ch:appx_sr2018}.8 shows the blame matrix for PM$_{fine}$
obtained by taking the differences of the result of a base run and a
scenario with 15\% reduction of primary PM$_{fine}$ emissions for one
country at a time.  Table~\ref{ch:appx_sr2018}.16.3 shows the corresponding
numbers obtained using the Local Fraction method, i.e. obtained in
a single run.  Table~\ref{ch:appx_sr2018}.16.2, similar to 
Table~\ref{ch:appx_sr2018}.8, but shows the blame matrix for {\sl primary}
PM$_{fine}$ from a scenario run.  Differences between 
Table~\ref{ch:appx_sr2018}.16.2 and Table~\ref{ch:appx_sr2018}.16.3 are due to the
``advection scheme effect''. Figure~\ref{fig:LF_SR} compares the results for each country. The differences are small, but the scenario
runs slightly overestimate the contributions of countries to itself (order of magnitude 2-7\% differences).

\section{Chemical mechanisms -- EmChem19a and EmChem19X}
\label{sec:Chems}

The EmChem19a scheme \citep{BergstromEmChem2020} is a small update of
the EmChem19 scheme introduced in \citet{R2019:ModDev}. 
EmChem19a is slightly simplified compared to EmChem19, including 
four fewer advected species and one less peroxy radical species. These 
simplifications were based on results from box model simulation tests at 
varying conditions and include the following:

\begin{itemize}
\item
Simplified C5DICAROOH + OH chemistry leading to the removal of C5134CO2OH
and C54CO from the scheme.

\item
Replacing PRRO2H by ACETOL.

\item
Replacing methyl vinyl ketone (MVK) by the model species MACR in the
simple monoterpene (APINENE) chemistry. This was made to be consistent
with the isoprene chemistry, which already uses this simplification
in EmChem19.

\end{itemize}

In addition, the following updates were made in EmChem19a:
\begin{itemize}
\item
A bug in the wet deposition parameters for the gas phase fraction of
volatility basis set (VBS) species, that was introduced in EmChem19,
was corrected.

\item
Wet deposition of peroxy acetic acid (\ce{CH3CO3H}) was added in EmChem19a.

\item
The reaction rates for $\alpha$-pinene$+$OH and $\alpha$-pinene$+$\ce{O3} were 
updated to the latest IUPAC recommendations.

\end{itemize}

A much larger version of EmChem --- denoted EmChem19X --- has also
been introduced.  The EmChem19X has a more detailed description of
aromatic chemistry and extended isoprene and monoterpene chemistries
(including MVK as a separate species). It also includes more organic
nitrates (including four additional peroxy acyl nitrates and a number of
nitrate containing peroxy radicals) and a number of other additions. The
EmChem19X is mainly intended for use in research projects and not for
the regular EMEP source-receptor modelling.

A manuscript including a detailed description of the new chemical schemes
(EmChem19a, EmChem19X, CRIv2R5Em, CB6r2Em), and evaluation of European
and global scale EMEP model results against ambient measurements,
and comparison to the GenChem version of the Master Chemical Mechanism
(MCMv3.3Em) in box model simulations, is currently being prepared for
submission \citep{BergstromEmChem2020}.

%%%%%%%%%%%%%%%%%%%%%%%%%%%%%%%%%%%%%%%%%%%%%%%%%%%%%%%%%%%%%%%%%%%%%%%%%%%%%%
\section{GenChem}
\label{sec:GChem}.

The EMEP model's chemical pre-processor, GenChem \citep{Simpson:GenChem}
includes the chemical pre-processor (GenChem.py), and a simple box-model
(boxChem). GenChem provides scripts and input files for converting
chemical equations into differential form for use in atmospheric chemical
transport models (CTMs) and/or the boxChem system.  Although GenChem is
primarily intended for users of the EMEP MSC-W CTM and related systems,
boxChem can be run as a stand-alone chemical solver, enabling for example
easy testing of chemical mechanisms against each other.


The code needed to run the GenChem  system is released as open-source
code under the GNU license, (\url{https://github.com/metno/genchem}),
with the user-guide provided at \url{https://genchem.readthedocs.io}.
\citet{Simpson:GenChem} presents an outline of the usage of the
GenChem system, explaining input and output files, and along with
\citet{BergstromEmChem2020} presents some examples of usage.



%%%%%%%%%%%%%%%%%%%%%%%%%%%%%%%%%%%%%%%%%%%%%%%%%%%%%%%%%%%%%%%%%%%%%%%%%%%%%%
\section{Emission speciation}
\label{sec:emissplits}.



The emissions speciation of NMVOC and primary PM (PPM) were updated
for the EmChem19a scheme to reflect recent data available from
the latest TNO/CAMS inventories (see chap.~\ref{chap:TNO}, also
\citealt{CAMSemis2019}).  For NMVOC the main changes have been:

\begin{enumerate}
  \item Use TNO NMVOC speciation

    TNO provided NMVOC speciation data for 25 compounds from each
    GNFR\_CAMS category, as part of the CAMS-REG-v3.1.2 database,
    which were then mapped to the EmChem19a species.

  \item Improve country-specific road transport estimates

    TNO provided country-specific fractions of the four road-traffic
    sectors in the CAMS-REG-AP\_v2.2.1 inventory for 2015, specifically
    F1=gasoline exhaust, F2=diesel exchaust, F3=LPG exhaust, and
    F4=non-exhaust emissions. These were also aggregated to form
    country-specific NMVOC splits for the generic GNFR F category
    (road transport).

\end{enumerate}

% PM TNO CAMS-REG_AP_v2.2.1_2015_REF

For \pmfine and \pmten, we made use of country-data generated by TNO for
the TFMM Euro-DeltaCarb simulations, which gave EC, OM (named
OC in the files), Na, SO4 and other compounds, as well as the fraction
of modern carbon in the GNFR-C categories. These data were aggregated
to the EMEP models EC, OM, and remPPM compounds. Data were provided for
both the Ref1 and Ref2 cases (see Chap.~\ref{chap:TNO}), and we assume
that Ref1 data are appropriate for modelling with officially submitted
emissions, and Ref2 data appropriate for modelling when Ref2 emissions
are used (e.g. for the \textbf{EMEPwRef2C} simulations discussed in
Chapters~\ref{ch:chapterStatus}--\ref{ch:Condensables}).


%%%%%%%%%%%%%%%%%%%%%%%%%%%%%%%%%%%%%%%%%%%%%%%%%%%%%%%%%%%%%%%%%%%%%%%%%%%%%%
\section{Emission inputs}
\label{sec:EmisInp}

The main change in emissions treatment has not been in the model code itself,
but in the use of the Ref2 inventory as described in chapters~\ref{ch:TNOref2} and \ref{ch:Condensables}.



\section{Documentation of model performance}
\label{sec:ModelPerf}

%=== rv4_34os for 2017
\begin{table}\small
\centering
\begin{center}
\caption{Evaluation statistics for the model version used in this Report (rv4\_35), run with 2017 emissions and meteorology (same setup as in Report 1/2019), against observations in 2017. Annual
  averages over all EMEP sites with measurements.  N$_{stat}$= number
  of stations, wd=wet deposition, cp= concentration in precipitation,
  Corr. = spatial correlation coefficient, RMSE = root mean square
  error, IOA = index of agreement. The requirement for data completeness is 75 and 25\% days with measurements for air concentrations and wet depositions/concentrations respectively.}
\label{tab:eval2017}
\begin{tabular}{c|ccccccc}
\hline\hline
Component  & N$_{stat}$ &Obs. & Mod. &Bias (\%) & RMSE & Corr.& IOA\\
\hline
%NO2_in_Air ugN/m3
\chem{NO_2} (\ugN)
  & 67 & 1.77 & 1.52 & -14 & 0.81 & 0.84 & 0.91\\%done2020
%SO2_in_Air ugS/m3
\chem{SO_2} (\ugS)
  & 59 & 0.41 & 0.33 & -19 & 0.66 & 0.57 & 0.63\\%done2020
%Sulphate_in_Air ugS/m3
\chem{SO_4^{2-}}, sea salt corrected (\ugS) % OBS CHOOSE CORRECT UNIT!!!!
  & 29 & 0.36 & 0.21 & -40 & 0.23 & 0.76 & 0.74\\%done2020
%Sulphate_in_Air ugS/m3
\chem{SO_4^{2-}}, including sea salt (\ugS) % OBS CHOOSE CORRECT UNIT!!!!
  & 34 & 0.46 & 0.29 & -36 & 0.25 & 0.77 & 0.72\\%done2020
%NO3-_in_Air ugN/m3
\chem{NO_3^-} (\ugN) % OBS CHOOSE CORRECT UNIT!!!!
  & 26 & 0.26 & 0.28 &  9  & 0.12 & 0.81 & 0.89\\%done2020
%HNO3_in_Air ugN/m3
\chem{HNO_3} (\ugN)
  & 17 & 0.12 & 0.10 & -17 & 0.08 & 0.54 & 0.68\\%done2020
%Sum_of_HNO3,_NO3-_in_air ugN/m3
\chem{NO_3^-}+\chem{HNO_3} (\ugN)
  & 37 & 0.42 & 0.42 &  -1 & 0.09 & 0.91 & 0.95\\%done2020
%Ammonia_in_Air ugN/m3
\chem{NH_3} (\ugN)
  & 22 & 0.53 & 0.65 &  24 & 0.38 & 0.84 & 0.89\\%done2020
%NH4+_in_Air ugN/m3
\chem{NH_4^+} (\ugN)
  & 26 & 0.45 & 0.38 & -16 & 0.25 & 0.67 & 0.81\\%done2020
%NH3+NH4+_in_Air ugN/m3
\chem{NH_3}+\chem{NH_4^+} (\ugN)
  & 33 & 1.13 & 1.70 &  51 & 2.41 & 0.48 & 0.41\\%done2020
%SO4_wet_dep. mgS/m2 
\chem{SO_4^{2-}} wd (\mgSm)
  & 52 & 9116  & 6658  & -27 & 100 & 0.63 & 0.75\\%done2020
%SO4_conc._in_precip. mgS/l (unit error removed in 2018)
\chem{SO_4^{2-}} cp (\mgSl)
  & 52 & 0.25 & 0.16 & -35 & 0.17 & 0.66 & 0.68\\%done2020
%Ammonium_wet_dep. mgN/m2 (unit error removed in 2018)
\chem{NH_4^+} wd (\mgNm)
  & 51 & 12646 & 15004 &  19 & 164 & 0.70 & 0.81\\%done2020
%Ammonium_conc._in_precip. mgN/l 
\chem{NH_4^+} cp (\mgNl)
  & 51 & 0.35 & 0.39 &  13 & 0.21 & 0.57 & 0.73 \\%done2020
%Nitrate_wet_dep. mgN/m2 (unit error removed in 2018)
\chem{NO_3^-} wd (\mgNm)
  & 53 & 10394 & 10401 &  0 & 87 & 0.79 & 0.89\\%done2020
%Nitrate_conc._in_precip. mgN/l 
\chem{NO_3^-} cp (\mgNl)
  & 43 & 0.30 & 0.27 & -12 & 0.24 & 0.46 & 0.58\\%done2020
Ozone daily max (ppb) 
  & 115 & 40.37 & 40.72 &  1 & 3.10 & 0.80 & 0.86\\%done2020
Ozone daily mean (ppb) 
  & 115 & 31.40 & 33.84 &  8 & 4.85 & 0.69 & 0.73\\%done2020
%PM10 ug/m3
\PM[10] (\ug)
  &  30 & 12.73 & 10.76 & -15 & 3.54 & 0.74 & 0.81\\%done2020
%PM25 ug/m3
\PM[2.5] (\ug)
  &  25 &  7.45 &  6.47 & -13 & 2.36 & 0.76 & 0.84\\%done2020
\hline\hline
\end{tabular}
\end{center}
\end{table}

In Tables~\ref{tab:eval2017}-\ref{tab:EMEPwRef2C_iiasaPM} the changes in the model performance due to the model version and emission input updates, described in \ref{Mod_2018} and in Section \ref{sec:emissplits}, are documented. Table \ref{tab:eval2017} provides the comparison statistics for model version rv4\_35 run for the year 2017 (using the same setup - emissions and meteorology - as in EMEP Status Report 1/2019) with EMEP observations in 2017. Comparison with the evaluation results of rv4\_33 model version for 2017, reported in 2019 (\citep{WEB2020:SN}, \citep{WEB2020:O3}, and \citep{WEB2020:PM}), shows that the differences due to the model version update are only minor. The main effects are due to bug correction in the MARS thermodynamic equilibrium module, which caused small (within few percent) changes in nitrogen and ammonium components. The largest effect (6-7\% higher concentrations in rv4\_35 run) is seen for PM and is due to the changes in ammonium nitrate and coarse nitrate.

Furthermore, Tables \ref{tab:EMEPrun}, \ref{tab:EMEPwRef2C_oldVOC} and \ref{tab:EMEPwRef2C_iiasaPM} show the evaluation statistics of model results against EMEP 2018 observations for a series of model test runs for 2018: EMEP run (with officially submitted emissions), EMEPwRef2C run using old speciation for NMVOCs and EMEPwRef2C run using old speciation for PM emissions. The emission input updates appear to effect only limited number of components, and most of the changes in performance statistics are negligibly small. These statistics can be compared with the evaluation of the EMEPwRef2C run for 2018  (Appendix \ref{ch:appx_modeleval}), reported in 'Supplementary material to EMEP Status Report 1/2020' (\citet{WEB2020:SN}, \citet{WEB2020:O3}, and \citet{WEB2020:PM}).


\begin{table}\small
\centering
\begin{center}
\caption{As Table \ref{tab:eval2017}, but for EMEP run (with officially reported emissions) for 2018 against EMEP observations in 2018.}
\label{tab:EMEPrun}
\begin{tabular}{c|ccccccc}
\hline\hline
Component  & N$_{stat}$ &Obs. & Mod. &Bias (\%) & RMSE & Corr.& IOA\\
\hline
%NO2_in_Air ugN/m3
\chem{NO_2} (\ugN)
  & 73 & 1.71 & 1.50 & -13 & 0.68 & 0.87 & 0.92\\%done2020
%SO2_in_Air ugS/m3
\chem{SO_2} (\ugS)
  & 57 & 0.30 & 0.26 & -13 & 0.21 & 0.63 & 0.78\\%done2020
%Sulphate_in_Air ugS/m3
\chem{SO_4^{2-}}, sea salt corrected (\ugS) % OBS CHOOSE CORRECT UNIT!!!!
  & 24 & 0.38 & 0.21 & -44 & 0.24 & 0.87 & 0.69\\%done2020
%Sulphate_in_Air ugS/m3
\chem{SO_4^{2-}}, including sea salt (\ugS) % OBS CHOOSE CORRECT UNIT!!!!
  & 32 & 0.48 & 0.29 & -39 & 0.24 & 0.87 & 0.71\\%done2020
%NO3-_in_Air ugN/m3
\chem{NO_3^-} (\ugN) % OBS CHOOSE CORRECT UNIT!!!!
  & 25 & 0.27 & 0.31 &  14 & 0.12 & 0.78 & 0.87\\%done2020
%HNO3_in_Air ugN/m3
\chem{HNO_3} (\ugN)
  & 17 & 0.12 & 0.10 & -18 & 0.08 & 0.54 & 0.68\\%done2020
%Sum_of_HNO3,_NO3-_in_air ugN/m3
\chem{NO_3^-}+\chem{HNO_3} (\ugN)
  & 34 & 0.42 & 0.42 &   0 & 0.08 & 0.94 & 0.97\\%done2020
%Ammonia_in_Air ugN/m3
\chem{NH_3} (\ugN)
  & 20 & 0.64 & 0.68 &   7 & 0.29 & 0.91 & 0.95\\%done2020
%NH4+_in_Air ugN/m3
\chem{NH_4^+} (\ugN)
  & 26 & 0.50 & 0.42 & -16 & 0.20 & 0.78 & 0.86\\%done2020
%NH3+NH4+_in_Air ugN/m3
\chem{NH_3}+\chem{NH_4^+} (\ugN)
  & 31 & 1.28 & 1.59 &  24 & 1.47 & 0.70 & 0.69\\%done2020
%SO4_wet_dep. mgS/m2 (unit error removed in 2018)
\chem{SO_4^{2-}} wd (\mgSm)
  & 43 &  9587 & 6149 & -36 & 253 & 0.76 & 0.61\\%done2020
%SO4_conc._in_precip. mgS/l (unit error removed in 2018)
\chem{SO_4^{2-}} cp (\mgSl)
  & 43 & 0.30 & 0.19 & -38 & 0.27 & 0.60 & 0.55\\%done2020
%Ammonium_wet_dep. mgN/m2 (unit error removed in 2018)
\chem{NH_4^+} wd (\mgNm)
  & 42 & 11065 & 11623 &  5 & 185 & 0.62 & 0.78\\%done2020
%Ammonium_conc._in_precip. mgN/l (unit error removed in 2018)
\chem{NH_4^+} cp (\mgNl)
  & 42 & 0.38 & 0.38 &  0 & 0.20 & 0.62 & 0.78\\%done2020
%Nitrate_wet_dep. mgN/m2 (unit error removed in 2018)
\chem{NO_3^-} wd (\mgNm)
  & 43 &  9003 &  8342 & -7 & 156 & 0.58 & 0.73\\%done2020
%Nitrate_conc._in_precip. mgN/l (unit error removed in 2018)
\chem{NO_3^-} cp (\mgNl)
  & 43 & 0.30 & 0.26 & -12 & 0.24 & 0.46 & 0.58\\%done2020
Ozone daily max (ppb) 
  & 117 & 42.60 & 41.51 & -3 & 3.00 & 0.82 & 0.85\\%done2020
Ozone daily mean (ppb) 
  & 117 & 33.00 & 34.41 &  4 & 4.15 & 0.72 & 0.76\\%done2020
%PM10 ug/m3
\PM[10] (\ug)
  &  31 &  14.09 &  10.49 & -26 & 4.65 & 0.68 & 0.69\\%done2020
%PM25 ug/m3
\PM[2.5] (\ug)
  &  26 &   8.28 &   6.66 & -20 & 2.84 & 0.76 & 0.82\\%done2020
\hline\hline
\end{tabular}
\end{center}
\end{table}


\begin{table}\small
\centering
\begin{center}
\caption{As Table \ref{tab:EMEPrun}, but for EMEPwRef2C run with old NMVOC speciation}
\label{tab:EMEPwRef2C_oldVOC}
\begin{tabular}{c|ccccccc}
\hline\hline
Component  & N$_{stat}$ &Obs. & Mod. &Bias (\%) & RMSE & Corr.& IOA\\
\hline
%NO2_in_Air ugN/m3
\chem{NO_2} (\ugN)
  & 73 & 1.71 & 1.49 & -13 & 0.68 & 0.87 & 0.92\\%done2020
%SO2_in_Air ugS/m3
\chem{SO_2} (\ugS)
  & 57 & 0.30 & 0.26 & -13 & 0.21 & 0.63 & 0.78\\%done2020
%Sulphate_in_Air ugS/m3
\chem{SO_4^{2-}}, sea salt corrected (\ugS) % OBS CHOOSE CORRECT UNIT!!!!
  & 24 & 0.38 & 0.21 & -44 & 0.24 & 0.87 & 0.69\\%done2020
%Sulphate_in_Air ugS/m3
\chem{SO_4^{2-}}, including sea salt (\ugS) % OBS CHOOSE CORRECT UNIT!!!!
  & 32 & 0.48 & 0.29 & -39 & 0.24 & 0.87 & 0.71\\%done2020
%NO3-_in_Air ugN/m3
\chem{NO_3^-} (\ugN) % OBS CHOOSE CORRECT UNIT!!!!
  & 25 & 0.27 & 0.31 &  14 & 0.12 & 0.78 & 0.87\\%done2020
%HNO3_in_Air ugN/m3
\chem{HNO_3} (\ugN)
  & 17 & 0.12 & 0.10 & -17 & 0.08 & 0.54 & 0.68\\%done2020
%Sum_of_HNO3,_NO3-_in_air ugN/m3
\chem{NO_3^-}+\chem{HNO_3} (\ugN)
  & 34 & 0.42 & 0.43 &   1 & 0.08 & 0.94 & 0.97\\%done2020
%Ammonia_in_Air ugN/m3
\chem{NH_3} (\ugN)
  & 20 & 0.64 & 0.68 &   7 & 0.29 & 0.91 & 0.95\\%done2020
%NH4+_in_Air ugN/m3
\chem{NH_4^+} (\ugN)
  & 26 & 0.50 & 0.42 & -16 & 0.20 & 0.78 & 0.86\\%done2020
%NH3+NH4+_in_Air ugN/m3
\chem{NH_3}+\chem{NH_4^+} (\ugN)
  & 31 & 1.28 & 1.59 &  24 & 1.47 & 0.70 & 0.69\\%done2020
%SO4_wet_dep. mgS/m2 (unit error removed in 2018)
\chem{SO_4^{2-}} wd (\mgSm)
  & 43 &  9587 & 6151 & -36 & 253 & 0.76 & 0.61\\%done2020
%SO4_conc._in_precip. mgS/l (unit error removed in 2018)
\chem{SO_4^{2-}} cp (\mgSl)
  & 43 & 0.30 & 0.19 & -38 & 0.27 & 0.60 & 0.55\\%done2020
%Ammonium_wet_dep. mgN/m2 (unit error removed in 2018)
\chem{NH_4^+} wd (\mgNm)
  & 42 & 11065 & 11624 &  5 & 185 & 0.62 & 0.78\\%done2020
%Ammonium_conc._in_precip. mgN/l (unit error removed in 2018)
\chem{NH_4^+} cp (\mgNl)
  & 42 & 0.38 & 0.38 &  0 & 0.20 & 0.62 & 0.78\\%done2020
%Nitrate_wet_dep. mgN/m2 (unit error removed in 2018)
\chem{NO_3^-} wd (\mgNm)
  & 43 &  9003 &  8363 & -7 & 156 & 0.58 & 0.73\\%done2020
%Nitrate_conc._in_precip. mgN/l (unit error removed in 2018)
\chem{NO_3^-} cp (\mgNl)
  & 43 & 0.30 & 0.27 & -12 & 0.24 & 0.46 & 0.58\\%done2020
Ozone daily max (ppb) 
  & 117 & 42.60 & 41.57 & -2 & 3.00 & 0.82 & 0.85\\%done2020
Ozone daily mean (ppb) 
  & 117 & 33.00 & 34.47 &  4 & 4.19 & 0.72 & 0.76\\%done2020
%PM10 ug/m3
\PM[10] (\ug)
  &  31 &  14.09 &  10.45 & -26 & 4.68 & 0.68 & 0.69\\%done2020
%PM25 ug/m3
\PM[2.5] (\ug)
  &  26 &   8.28 &   6.62 & -20 & 2.86 & 0.76 & 0.82\\%done2020
\hline\hline
\end{tabular}
\end{center}
\end{table}


\begin{table}\small
\centering
\begin{center}
\caption{As Table \ref{tab:EMEPrun}, but for EMEPwRef2C run with old PM speciation}
\label{tab:EMEPwRef2C_iiasaPM}
\begin{tabular}{c|ccccccc}
\hline\hline
Component  & N$_{stat}$ &Obs. & Mod. &Bias (\%) & RMSE & Corr.& IOA\\
\hline
%NO2_in_Air ugN/m3
\chem{NO_2} (\ugN)
  & 73 & 1.71 & 1.50 & -12 & 0.68 & 0.87 & 0.92\\%done2020
%SO2_in_Air ugS/m3
\chem{SO_2} (\ugS)
  & 57 & 0.30 & 0.26 & -13 & 0.21 & 0.63 & 0.78\\%done2020
%Sulphate_in_Air ugS/m3
\chem{SO_4^{2-}}, sea salt corrected (\ugS) % OBS CHOOSE CORRECT UNIT!!!!
  & 24 & 0.38 & 0.21 & -44 & 0.24 & 0.87 & 0.69\\%done2020
%Sulphate_in_Air ugS/m3
\chem{SO_4^{2-}}, including sea salt (\ugS) % OBS CHOOSE CORRECT UNIT!!!!
  & 32 & 0.48 & 0.29 & -39 & 0.24 & 0.87 & 0.71\\%done2020
%NO3-_in_Air ugN/m3
\chem{NO_3^-} (\ugN) % OBS CHOOSE CORRECT UNIT!!!!
  & 25 & 0.27 & 0.31 &  13 & 0.12 & 0.78 & 0.87\\%done2020
%HNO3_in_Air ugN/m3
\chem{HNO_3} (\ugN)
  & 17 & 0.12 & 0.10 & -18 & 0.08 & 0.54 & 0.68\\%done2020
%Sum_of_HNO3,_NO3-_in_air ugN/m3
\chem{NO_3^-}+\chem{HNO_3} (\ugN)
  & 34 & 0.42 & 0.42 &   0 & 0.08 & 0.94 & 0.97\\%done2020
%Ammonia_in_Air ugN/m3
\chem{NH_3} (\ugN)
  & 20 & 0.64 & 0.68 &   7 & 0.29 & 0.91 & 0.95\\%done2020
%NH4+_in_Air ugN/m3
\chem{NH_4^+} (\ugN)
  & 26 & 0.50 & 0.42 & -16 & 0.20 & 0.78 & 0.86\\%done2020
%NH3+NH4+_in_Air ugN/m3
\chem{NH_3}+\chem{NH_4^+} (\ugN)
  & 31 & 1.28 & 1.59 &  24 & 1.47 & 0.70 & 0.69\\%done2020
%SO4_wet_dep. mgS/m2 (unit error removed in 2018)
\chem{SO_4^{2-}} wd (\mgSm)
  & 43 &  9587 & 6149 & -36 & 253 & 0.76 & 0.61\\%done2020
%SO4_conc._in_precip. mgS/l (unit error removed in 2018)
\chem{SO_4^{2-}} cp (\mgSl)
  & 43 & 0.30 & 0.19 & -38 & 0.27 & 0.60 & 0.55\\%done2020
%Ammonium_wet_dep. mgN/m2 (unit error removed in 2018)
\chem{NH_4^+} wd (\mgNm)
  & 42 & 11065 & 11622 &  5 & 185 & 0.62 & 0.78\\%done2020
%Ammonium_conc._in_precip. mgN/l (unit error removed in 2018)
\chem{NH_4^+} cp (\mgNl)
  & 42 & 0.38 & 0.38 &  0 & 0.20 & 0.62 & 0.78\\%done2020
%Nitrate_wet_dep. mgN/m2 (unit error removed in 2018)
\chem{NO_3^-} wd (\mgNm)
  & 43 &  9003 &  8339 & -7 & 156 & 0.58 & 0.73\\%done2020
%Nitrate_conc._in_precip. mgN/l (unit error removed in 2018)
\chem{NO_3^-} cp (\mgNl)
  & 43 & 0.30 & 0.26 & -12 & 0.24 & 0.46 & 0.58\\%done2020
Ozone daily max (ppb) 
  & 117 & 42.60 & 41.52 & -3 & 3.03 & 0.82 & 0.85\\%done2020
Ozone daily mean (ppb) 
  & 117 & 33.00 & 34.36 &  4 & 4.14 & 0.72 & 0.76\\%done2020
\PM[10] (\ug)
  &  31 &  14.09 &  10.43 & -26 & 4.69 & 0.68 & 0.69\\%done2020
%PM25 ug/m3
\PM[2.5] (\ug)
  &  26 &   8.28 &   6.61 & -20 & 2.88 & 0.76 & 0.81\\%done2020
\hline\hline
\end{tabular}
\end{center}
\end{table}

\clearpage
\bibliographystyle{copernicus}         % change bibliography-name after each
\renewcommand\bibname{References}      % bibliographystyle command!
\addcontentsline{toc}{section}{References}
\bibliography{Refs,EMEP_Reports,uEMEP}
%\bibliography{myRefs,EMEP_Reports}
