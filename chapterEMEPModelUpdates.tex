\chapter[Model updates]{Updates to the EMEP MSC-W model, 2020--2021}
\label{ch:ModelUpdates}

%% authors? added all names seen in Log.changes

{\bf{David Simpson, Michael Gauss, Qing Mu, Svetlana Tsyro and Peter Wind}}
\vspace{30pt}

%\old{TODO}

This chapter summarises the changes made to the EMEP MSC-W  model
since \citet{R2020:ModDev}, and along with changes discussed in
\citet{R2013:ModDev,R2015:ModDev,R2016:ModDev,R2017:ModDev,R2019:ModDev,R2020:ModDev} and
\citet{R2014:ModDev},
updates the standard description given in \citet{Simpson:EMEP2012}. The
model version used for reporting this year is denoted rv4.42, which has
had some major updates (especially with regard to emissions) compared to
the rv4.35 reported in \citet{R2020:ModDev}.
Table~\ref{tab:Updates} summarises
the changes made in the EMEP model since the version documented in
\citet{Simpson:EMEP2012}, and these changes are discussed in
more detail in Sects.~\ref{sec:updateOverview}-\ref{sec:updateOther}.

%Versions:
%
%  rv4.36 - open source November 2020
%  rv4.35 used  for R2020
%  rv4.34 - open source January 2020
%  rv4.33 used  for R2019
%  rv4.33 - open source June 2019
%  rv4.32 used for EMEP course, April 2019
%  rv4.17a used for R2018 runs  (July-ish?)
%  rv4.17 released 26/2/2018
%  rv4.16 interim 21/12/2017 - used for N2O5 paper, wheat calculations
%  rv4.15 released 8/9/2017

\section{Overview of changes} 
\label{sec:updateOverview}

%%%%%%%%%%%%%%%%%%%%%%%%%%%%%%%%%%%%%%%%%%%%%%%%%%%%%%%%%%%%%%%%%%%%%%%%%%%%%
\begin{table}
\begin{footnotesize}
\caption{Summary of major EMEP MSC-W model versions from 2012--2020.
Extends Table S1 of \citealt{Simpson:EMEP2012}
  COMMENT: Note - recent changes and tags are a little confused. Dave will fix.
}
\label{tab:Updates}
\centering
\begin{tabular}{lp{11cm}l}
\hline
Version & Update                                        & Ref$^{(a)}$   \\
\hline
rv4.42  & Changed default Kz scheme &  This report        \\
        & Updated emissions from CAMS81 project: DMS, soil NO;  &  This report        \\
% 21st May 2021:
rv4.41  & Updated emissions from CAMS81 project: DMS, soil NO; % 16/5
          Modified various parameters concerning sea-salt and wet-radius calculations; % 16/5
   fracPM25 modified;
          Revised global monthly factor;  % 16/5
        & \\
% 13/5
rv4.40  & Shipping emissions now spread as 20\% below 20m, 80\% between 20-90m;  
          New CAMS81-based aircraft emissions; % 29-30/4
          New CAMS81-based DMS emissions; % 29/4
        & \\
%SKIP BUG fix on aircraft emissions  27/4
% 21/4
rv4.39 &
  GNFR-CAMS is now the default emissions system - see Sect.~\ref{ssec:gnfr};  % 21/4
          Uses global time-zone map;   % 19/3
          Revised global monthly factor?? or rv4.41?;  % 19/3
          New default Kz method introduced (Sect.~\ref{ssec:PBL});
          NWP model rh2m now used in place of earlier sub-grid calculation; % 9/2
          Upgrade of local fractions methodology;  % 7/11/2020
        & \\
% 23/10/2021:
rv4.35  & Various updates, including heavy   
          refactoring of local-fraction code, bug-fixes in MARS module,
          and updates in chemical mechanisms, default PM and NMVOC speciation and
          GenChem systems     & R2020            \\
rv4.34  & Public domain (Feb. 2020); EmChem19a, EmChem19p      & R2020            \\
rv4.33  & Public domain (June 2019);
         EmChem19, PAR bug-fix, EQSAM4clim    & R2019            \\
rv4.32  & Used for EMEP course, April 2019    &    \\
rv4.30  & Moved to new GenChem-based system  &   \\
%        &                                    &        \\
rv4.17a & Used for R2018. Small updates         & R2018      \\
rv4.17  & Public domain (Feb. 2018);
         Corrections in global land-cover/deserts; added
          'LOTOS' option for European \ce{NH3} emissions; corrections
          to snow cover & R2018 \\
rv4.16  & New radiation scheme (Weiss\&Norman); Added dry and wet deposition for \ce{N2O5};
         (Used for  \citealt{Stadtler2018,MillsGCB2018b}) & R2018   \\
rv4.15  & EmChem16 scheme & R2017 \\
%    Sect.\ref{sec:GNFR}--\ref{ss:Splits} & R2016  \\
%
rv4.14  & Updated chemical scheme & R2017       \\
%% rv4.13 + CRI was used in McFiggans. Difficult to describe combo
%        & \\
rv4.12  & New  global land-cover and BVOC & R2017       \\
%        & \\
rv4.10  &  Public domain (Oct. 2016)                 
         (Used for  \citealt{MillsGCB2018a}) &  R2016 \\
%        & \\
rv4.9   & Updates for GNFR sectors, DMS, sea-salt, dust, \ce{S_A} and  $\gamma$, \ce{N2O5} & \\ 
rv4.8   &  Public domain (Oct. 2015); ShipNOx introduced.                          
         Used for EMEP HTAP2 model calculations, see
         special issue:
         \url{www.atmos-chem-phys.net/special_issue390.html},
          and \citet{Jonson_et_al:2017}.              & R2015\\
rv4.7   & Used for reporting, summer 2015;
         New calculations of aerosol surface area; 
         New gas-aerosol uptake and \ce{N2O5} hydrolysis rates; 
         Added 3-D calculations pf aerosol extinction and AODs;
         Emissions - new flexible mechanisms for interpolation and merging sources;
         Global - monthly emissions from ECLIPSE project;
         Global -  LAI changes from LPJ-GUESS model;
         WRF meteorology \citep{SkamarockKlemp2008} can now
     be used directly in EMEP model. & R2015 \\
%        &                                                &\\
rv4.6   & Used for Euro-Delta SOA runs                   & R2015  \\
%QUERY        & Bug-fix for ammonium deposition & \\
       & Revised boundary condition treatments % & \\  % Vertical profiles
       ; ISORROPIA capability added & \\
%       &                                                &\\
rv4.5  & Sixth open-source (Sep 2014);                    
        Improved dust, sea-salt, SOA modelling          % &      \\
       ; AOD and extinction coefficient calculations  updated %& \\
       ; Data assimilation system added % & \\
       ; Hybrid vertical coordinates replace earlier sigma % & \\
       ; Flexibility of grid projection increased. & R2014\\
%SKIP        & ?? Point sources, plume rise, data-assimilation\\
%       &                                                &\\
rv4.4   & Fifth open-source (Sep 2013) %
       ; Improved dust and sea-salt modelling   %          &      \\
       ; AOD and extinction coefficient calculations added %  &\\
       ; gfortran compatibility improved            %      &      \\
                  & R2014, R2013\\
%       &                                                &\\
rv4.3   & Fourth public domain (Mar. 2013)  %
       ; Initial use of namelists           %            & \\
       ; Smoothing of MARS results         %            & \\
       ; Emergency module for volcanic ash and other events% & \\
       ; Dust and road-dust options added as defaults % & \\
       ; Advection algorithm changed  % & \\ % \citet{CLAPP98}    & \\
             & R2013\\ 
%        &                                                &\\
rv4.0   & Third public domain (Sep. 2012), as \citet{Simpson:EMEP2012}            & R2013\\ 
%        & As documented in \citet{Simpson:EMEP2012}    & \\
%v2011-06& Second public domain (Aug. 2011)                &\\ 
%rv3     & First public domain (Sep. 2008)                &\\ 
        &                                                &\\
\hline
\end{tabular}
Notes: (a) R2018 refers to EMEP Status report 1/2018, etc.
\end{footnotesize}
\end{table}
%%%%%%%%%%%%%%%%%%%%%%%%%%%%%%%%%%%%%%%%%%%%%%%%%%%%%%%%%%%%%%%%%%%%%%%%%%%%%

\begin{itemize}

\item Introduced new soil-NO, DMS and aircraft emissions, from CAMS-81 project
(Sect.~\ref{sec:updateEmis}). 

\item Revised methods for vertical diffusion (Kz) \QUERY{and Hmix?} (Sect.~\ref{sec:updateKz}). -- Qing

\item Modified fine/coarse split of sea-salt (Sect.~\ref{ssec:updateSS}).

\item
Upgraded Local Fractions capabilities (Sect.~\ref{sec:updateLF}).

\item  fracPM25 stuff (Sect.~\ref{sec:updateOther}). -- Dave
\item revised wet-radius calculations (Sect.~\ref{sec:updateOther}). --Dave

%\item The EMEP model's chemical pre-processing system and associated box-model (GenChem, boxChem) have been released as open-source.  See Sect.~\ref{sec:GChem}.

\item
Emissions speciation. New default and country-specific emission
speciations for NMVOC and \pmfine have been implemented.  See
Sect.~\ref{ssec:emissplits}.

\item
Numerous small changes to make the code more flexible and/or to
fix minor bugs.

\end{itemize}

In addition to these changes, articles on GenChem (EMEP's chemical pre-processing system)  and
the local-scale extension \COMMENT{Phrasing?} have now been published \citep{Simpson:GenChem,Denby:2020}

%%%%%%%%%%%%%%%%%%%%%%%%%%%%%%%%%%%%%%%%%%%%%%%%%%%%%%%%%%%%%%%%%%%%%%%%%%%%%%
\section{New emission inputs}
\label{sec:updateEmis}

From CAMS81...

\subsection{New basis for emission sectors: GNFR\_CAMS}
\label{ssec:gnfr}

\COMMENT{Text/Table from gmdPaper}
 
Gridded anthropogenic emissions from CEIP
were previously categorized into 11
SNAP sectors, but for many years now EMEP emission reporting has been conducted and prepared for
modelling using the 13-sector
GNFR system\footnote{
GNFR=Gridding nomenclature for reporting/UNECE nomenclature for reporting of emissions to air,
e.g. \citealt{CEIP2020:GNFR}}.
In 2020 a 19-sector emission system (`GNFR\_CAMS') was implemented in the EMEP model, to take care of
emissions provided by TNO as part of the Copernicus CAMS project \citep{Kuenen2021}. This extended emissions system
enables for example four road traffic sectors, F1--F4, with e.g. F1
 representing exhaust emissions from gasoline vehicles.
Such emission sectors are characterised in the model by
release heights, timefactors and species-splits (e.g. NOx to NO and
\ce{NO2}, or NMVOC to individual VOC surrogates) for each sector.

Table~\ref{tab:GNFRsectors} summarises the new GNFR\_CAMS sectors,
and the mapping indices used.
%\COMMENT{Add index explanation in SI table?}


\begin{table}
\caption{The `GNFR\_CAMS' 19-sector system, and mapping indices for release heights, timefactors and species-splits,
 which is now
 default in EMEP model. \label{tab:GNFRsectors}
}
\begin{tabular}{lrrrrlc}
\hline
code & Number  & \multicolumn{3}{c}{Index}      & sector & SNAP \\ \cline{3-5}
     &         & timefac  & height & emissplit  &        & equivalent \\
\hline
A &      01 &   1 &  1 &   1 &  Public Power  &  1 \\
B &      02 &   3 &  3 &   2 &  Industry  & 3, 4 \\
C &      03 &   2 &  2 &   3 &  Other Stationary Combustion & 2 \\
%D &      04 &   4 &  4 &   4 &  Fugitive & \QUERY{EXPLAIN/CHECK??}  \\
%dev:
D &      04 &   5 &  5 &   4 &  Fugitive & \QUERY{as in dev 16/8/CHECK??}  \\
E &      05 &   6 &  2 &   5 &  Solvents & 6 \\
F &      06 &   7 &  2 &   6 &  Road Transport & 7 \\
G &      07 &   8 &  8 &   7 &  Shipping & 8  \\ % dev 16/8
H &      08 &   8 &  7 &   8 &  Aviation & 8  \\
I &      09 &   8 &  2 &   9 &  Offroad & 8  \\
J &     10 &   9 &  6 &  10 &  Waste & \QUERY{EXPLAIN/CHECK?} \\
K &     11 &  10 &  2 &  11 &  Agri - Livestock & 10 \\
L &     12 &  10 &  2 &  12 &  Agri - Other & 10 \\
M &     13 &   5 &  5 &  13 &  Other & 11 \\
A1 &    14 &   1 &  1 &   1 &  PublicPower - Point & 1 \\
A2 &    15 &   1 &  3 &   1 &  PublicPower - Area & 1 \\
F1 &    16 &   7 &  2 &  16 &  Road Transport - Exhaust Gasoline & 7  \\
F2 &    17 &   7 &  2 &  17 &  Road Transport - Exhaust Diesel& 7   \\
F3 &    18 &   7 &  2 &  18 &  Road Transport - Exhaust LPGgas & 7  \\
F4 &    19 &   7 &  2 &  19 &  Road Transport - NonExhaust Other& 7   \\
\hline
\end{tabular}
\end{table}

\clearpage


\subsection{Soil NO emissions}
\label{ssec:soilNO}

The EMEP model now makes uses of a new global dataset for soil NO
emissions. This dataset, described in detail in \citet{SimpsonDarras:2021},
provides gridded monthly data and the corresponding 3-hourly
weight factors at \ResEq{0.5} degrees horizontal resolution, over
the period 2000-2018.  The basic methodology merges methods from
\citet{YiengerLevy:1995}, with various updates to reflect recent
literature \citep[especially][]{SteinkampLawrence2011}, and some
simplifications which reflect lack of  availability of some key data.

%In YL95 and SL11, background biome emission factors ($A_{biome}$)  were modified by estimates of locally available nitrogen ($N_{avail}$),
%which consists mainly of agricultural inputs of N (N from fertilizer, manure, hereafter \Nfert), or atmospheric deposition
%of reactive nitrogen (hereafter \Ndep), and a pulse factor, \Fp. For this work we prefer to calculate the contributions of
%\Nfert, \Ndep and \Fp separately, so we have:

%Emissions are provided as total values and also with separate data for soil
%NO emissions from background biome values, and  those induced by fertilizers/manure, pulsing effects, and atmospheric
%deposition, so that users can include, exclude or modify each component if wanted.

The CAMS-GLOB-SOIL v2.2 NOx inventory
provides estimate of soil NO emissions from four categories:

\begin{description}
  \item[Biome] - a background emission rate for each landcover type, modified by temperature
and soil moisture % (CRF?)
  \item[Fert] - emissions resulting from applied fertilizer and manure inputs to soils. (We assume
that  0.7\% of the N-inputs are re-emitted as NO.)
  \item[Ndep] - emissions resulting from atmospheric N deposition (also 0.7\%)
  \item[Pulsing] - emissions resulting from rain and/or soil moisture changes after
    a dry period
\end{description}

In developing CAMS-GLOB-SOIL , we  expressed the total soil emission fluxes as the sum
of these terms:
\begin{equation}
  \Fs = \Fb  + \Ff + \Fd + \Fp
  \label{eqn:Fs}
\end{equation}

Given that the inventories to be discussed seem to include a
component similar to our \Ff, we hereby introduce \Fn, such that:
\begin{equation}
  \Fs = \Fn  + \Ff
  \label{eqn:Fn}
\end{equation}
Where \Fn then is the sum of the \Fb, \Fd, and \Fp terms, and where the
flux terms have units \ngN. The calculations of \Fb, \Ff, \Fd and \Fp
are detailed in \citet{SimpsonDarras:2021}.
Figure~\ref{fig:SIemis} illustrates the contribution of these terms for
the European region.

\begin{figure}
  \centering
   %\includegraphics*[width=0.7*\textwidth]{\HFIGS/PlotRegionalTimeSeriesMonthly_EUR_YLnSMI_Mar2021.png}
   \includegraphics*[width=10cm]{FIGS_UPDATES/PlotRegionalTimeSeriesMonthly_EUR_YLnSMI_Mar2021.png}
  \parbox{10cm}{
  \caption{CAMS-GLOB-SOIL: monthly above canopy NO emissions (\ngN)  calculated for the year 2010
     for Europe.
    \label{fig:SIemis}}
  }
\end{figure}

% from  gmdSI from /home/davids/Documents/ARTICLES/2021/GmdSoilNO
% CHECK also gmdSI from /home/davids/Work/EU_Projects/CAMS/CAMS81/D81_3_3_1_jun2021

\subsubsection{Risk of double-counting?}

An important issue arose during the construction  of this inventory - the risk of double-counting
soil-NO emissions, since some anthropogenic  emission inventories include and 
some exclude these.
Indeed, within the LRTAP Convention,
most countries mainly report NOx emissions due to agricultural activities
using the EMEP/EEA Emissions Inventory Guidebook
\citep{Guidebook2019:3D}. The Guidebook provides methods for calculating
soil-NO data from fertilizer and other inputs.
\citet{SimpsonDarras:2021} present the main sources for which
 soil NO emissions areas covered by the Guidebook, and
compare the nationally
submitted emissions following this system with the CAMS estimates.
It is shown that for the vast majority of countries
the main emission categories are NFR  3Da1,3Da2a-c,and
3Da3, which together are roughly equivalent to
the `Fert' emissions from CAMS-GLOB-SOIL.

When using EMEP emissions derived from
officially reported data (with soil NO emissions as given in GNFR L),
for example in EMEP MSC-W reporting runs,
\citet{SimpsonDarras:2021} recommended to retain the
official GNFR L data, but but add biome, N-dep and pulse emissions from CAMS-GLOB-SOIL.

\citet{SimpsonDarras:2021} also give recommendations for
ECLIPSE v5 and v6 inventories, and for CAMS-GLOB and CAMS-REG. In
some cases the CAMS-GLOB-SOIL \Nfert emissions should be used, but
in other cases not. In the EMEP model this choice is governed
by the config\_emep.nml variable \verb|USES%SOILNOX_METHOD |, which can
be set to either `Total', or `NoFert'.


\COMMENT{Will add some words on impact of new soil emissions later.}

\subsection{DMS emissions}
\label{ssec:DMS}

A new option to calculate biogenic emissions of dimethyl sulphide (DMS) has been added to the EMEP model. It is based on the climatology of sea surface DMS concentrations published by \citet{Lana2011} and the meteorological data driving the EMEP model, and it has been used for this year's status run. 

At each location and timestep, the model calculates the Schmidt number ($Sc$) from the sea surface temperature ($SST$, given in \textdegree C) with the 4th-degree formula of \citet{Wanninkhof2014}:
\\
\\
$Sc=2855.7 -177.63\  SST + 6.0438\  SST^2 - 0.11645\  SST^3 + 0.00094743\ SST^4$ 
\\
\\
The transfer velocity $K_{w}$ (given in cm hour$^{-1}$) is then calculated  from the Schmidt number and the 10-meter wind speed $u_{10}$ (given in m~s$^{-1}$) with the formula of \citet{Nightingale2000}:
\\
\\
$K_w=(0.222\  u_{10}^2 +0.333\ u_{10}) \sqrt{\frac{600}{Sc}} $
\\
\\
Following \citet{Leonor:DMS2} we assume that 66\% of the DMS in air is converted into \ce{SO2}. The DMS emission is thus implemented in the model as an \ce{SO2} flux as:

\begin{verbatim}
flux(SO2)=0.66*O_DMS*Kw/3600*0.01*GRAV*roa/p*AVOG  
\end{verbatim}

where $O\_DMS$ is the sea surface DMS concentration (in nanomol L$^{-1}$),
$GRAV$ is standard gravity (9.807 m s$^{-2}$),
$roa$ is the ambient density of air (kg m$^{-3}$),
$p$ is ambient air pressure
and $AVOG$ is Avogradro's number (6.023e+23 mol$^{-1}$).

The options to calculate the Schmidt number and transfer velocity, and also to use the older climatology of \citet{Kettle1999}, are still available in the EMEP model.

\subsection{Aircraft emissions}
\label{ssec:Aircraft}

 LTO (landing/take-off) emission data are, as before, taken from CEIP, i.e. based on aircraft emissions officially reported by countries for up to 3000 feet above the surface. However, the high distribution in the EMEP model has been slightly revised. We assume that, in approximation, half of the LTO emissions (from CEIP) occur below 50 meters above the surface, while the other half is evenly distributed over the height range between 50 meters and about 3000 feet.
 Furthermore, aircraft emissions at higher altitudes than 3000 feet are now based on a new dataset (CAMS-GLOB-AIR, \citet{GranierCAMS2019}), given separately for all years from 2000 to 2019, with monthly variation and a spatial resolution of 0.5$^\circ$\,$\times$\,0.5$^\circ$\,$\times$\,610\,m. This dataset has been provided through the Copernicus Atmosphere Monitoring Service and can be downloaded from the Atmosphere Data Store of ECMWF (\url{https://ads.atmosphere.copernicus.eu/}).

In the EMEP model, CAMS-GLOB-AIR emissions between 0 and 610m (2000 ft) above the surface are not used, and half of the CAMS-GLOB-AIR emissions between 610 and 1220m are evenly distributed over the model layers corresponding to that height range. The assumption behind this approch is that emissions up to 3000 feet should already be fully included in the CEIP (LTO) emissions. Above 1220m (4000ft), CAMS-GLOB-AIR emissions are used in full, and interpolated to the relevant model grid during model runtime.

The option of using the old QUANTIFY data (\url{www.pa.op.dlr.de/quantify}) instead of CAMS-GLOB-AIR for non-LTO emissions is still available in the EMEP model.

\subsection{Revised fine/coarse splits of sea-salt emissions}
\label{ssec:updateSS}

The split of sea salt between the fine and coarse fractions have been revised based on the results of comparison of modelled \chem{Na+} in \PM[2.5] and \PM[10] with EMEP observations. Namely, the fraction of sea salt emitted in \PM[2.5] fraction has been slightly increased (while the total sea salt emissions in \PM[10] remain unchanged). 

More specifically, sea salt is emitted in seven size bins up to 10~\um in diameter, from which the smallest four bins were previously assigned to the \PM[2.5] fraction and the remaining three bins comprised coarse sea salt. Now, the fifth bin is split 50/50 \% between the \PM[2.5] and \PM[2.5-10] fractions.  


This modification has a minor effect on formation of coarse \ce{NO3-} as the surface area of coarse sea salt has slightly decreased.



%%%%%%%%%%%%%%%%%%%%%%%%%%%%%%%%%%%%%%%%%%%%%%%%%%%%%%%%%%%%%%%%%%%%%%%%%%%%%%
\subsection{Emission speciation and `rnr' splits}
\label{ssec:emissplits}.

The emissions speciation of primary PM (PPM) were updated
to reflect recent data available from
ECLIPSE v6b dataset
(\url{https://iiasa.ac.at/web/home/research/researchPrograms/air/ECLIPSEv6.html}).
These emissions were provided for the years 2000-2016 (plus 2020 and some future
scenarios), and give country-specific fine and coarse emissions of EC, OM and remaining
PPM (remPPM) in fine and coarse modes.

In many previous studies we have distinguished estimated wood-burning and
fossil-fuel emissions, but for the runs presented in this report we
have made the simpler distinction between GNFR category C (mainly
residential emissions), and other GNFR sectors.  We denote this
split the `rnr' (residential-non-residential) split, and have slighly
modified the EmChem19a chemical mechanism to cope with the addition
of `remPPM' from GNFR C.


%%%%%%%%%%%%%%%%%%%%%%%%%%%%%%%%%%%%%%%%%%%%%%%%%%%%%%%%%%%%%%%%%%%%%%%%%%%%%%
\section{Revised PBL parametrisations}
\label{sec:updateKz}

The new default calculation of eddy diffusivity coefficient \textit{Kz} (`TROENKz') follows the method described in \citet{troenkz}. Unlike the old \textit{Kz} method (JericevicKz) which does not consider the stability within the boundary layer, this new \textit{Kz} method differentiates \textit{Kz} formulation in the stable and unstable boundary layer conditions. Therefore, the new method predicts reasonably well in cases of weak surface heat flux and transitions between stable and unstable cases \citep{troenkz}. However, we have not seen significant changes in modelled concentrations. The formulation of \textit{Kz} is given as below. In unstable boundary layer:
\begin{equation}
    \phi_h = (1 - 16 \times \min(z, z_\mathit{surf}) \times \frac{1}{L})^{-\frac 1 2}
\end{equation}
while in stable boundary layer:
\begin{equation}
    \phi_h = 1 + 5 \times z \times \frac{1}{L}
\end{equation}
and \textit{Kz} is given as: 
\begin{equation}
    \mathit{Kz} = 0.4 \times \frac{u_*}{\phi_h} \times z \times (1 - \frac z h)^2
\end{equation}
in which $z$ is the height from surface, $h$ is boundary layer depth, $z_\mathit{surf}$ is the height of surface layer which is assumed to be 10\% of the boundary layer depth $h$,  $L$ is the Monin-bukhov length \citep{Garratt}, and $u_*$ is friction velocity.

The default calculation of PBL is updated to the HmixMethod JcRb\_t2m. Compared with the old HmixMethod JcRb, the new one uses temperature at 2 m instead of temperature at the middle of the lowest grid cell to formulate PBL in the stable conditions. For unstable conditions, the new formulation is the same as the JcRb method. The differences in concentrations are unnoticeable between the two methods, but this update is more accurate according to the original reference paper \citep{JcRb}.  


%%%%%%%%%%%%%%%%%%%%%%%%%%%%%%%%%%%%%%%%%%%%%%%%%%%%%%%%%%%%%%%%%%%%%%%%%%%%%%
\section{Local Fractions}
\label{sec:updateLF}

The Local Fractions method \citep{wind-2020} is a framework that allows to track thousands of sources in a single run at a low computational cost. It is fully implemented for primary particles, but also reduced nitrogen compounds can be treated. The tracking of other species is less accurate, but progress is being made to eventually include a correct description for all species.

Since last year, the range of applications of the Local Fractions has been extended.

The sources can be defined by either a list of countries or group of countries, as small squares surrounding the receptor grid (with user defined range and square sizes), or as regions defined as "masks" given in a NetCDF file.
Wet and dry deposition (separately) per source country can now also be computed.



%%%%%%%%%%%%%%%%%%%%%%%%%%%%%%%%%%%%%%%%%%%%%%%%%%%%%%%%%%%%%%%%%%%%%%%%%%%%%%
\section{Other}
\label{sec:updateOther}

A number of smaller changes have been made:

\begin{description}
  \item[fracPM25] A fraction of coarse nitrate particulate mass has been allocated to the \PM[2.5] fraction in order to represent the fact that some of the particles from the coarse fraction have diameters of less than 2.5~\um. The value of this fraction was previously 27\% \citep{Simpson:EMEP2012}. In the latest version this fraction was reduced to 13\%, in order to better account for the differences between mass-median and aerodynamic diameters.
  \item[IAM\_CROP] A new generic crop type has been introduced for integrated assessment modelling outputs. This is an updated version of the previously used IAM\_CR species \citep{Simpson:EMEP2012}, in which the phenology factor (\fphen) has been set to 1.0 across the ozone flux accumulation period. (The IAM\_CR parameters gave a reduction in \fphen towards the end of the period.)
  \item[Pollen emissions] Optional birch, olive, ragweed and grass pollen emissions based on \citet{Sofiev-2015,Sofiev-2017}.
\end{description}


revised wet-radius calculations --Dave




\clearpage
\bibliographystyle{copernicus}         % change bibliography-name after each
\renewcommand\bibname{References}      % bibliographystyle command!
\addcontentsline{toc}{section}{References}
\bibliography{Refs,EMEP_Reports,uEMEP,RefsUpdates}
