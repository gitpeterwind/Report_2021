\chapter[Trends]{Trends of......}
\label{ch:XXX}

{\bf{Authors}}\\

\old{TODO}

\section{\label{sec:Trends_introduction}Introduction and background}

Colocation: To be done

We use the same methodology as described by \cite{aas2019global, mortier2020} to derive the trends at the individual stations. The significance of the trends is tested with the Mann-Kendall test \citep{hamed1998modified}. The related p-value is used to determine if the trend is significant or not within a confidence interval of 95\%. The slope is calculated with the Theil-Sen estimator which is less sensitive to outliers than standard least-squares methods \citep{sen1968estimates}. At least least 7 valid yearly regional averages (50\% of time coverage) are required in the regional time series for the computation of a slope.

An uncertainty is provided for each trend by combining the error of the slope calculation itself to the error of the residuals:

\begin{equation}
 Uncertainty = \sqrt{{\left (\frac{\Delta m}{y(2000)}\right )}^{2} + {\left ( \frac{m \cdot \Delta r}{y(2000)^2}\right )}^{2} }
\end{equation}

where $\Delta m$ is the Theil-Sen estimator 95\% confidence interval, $y(2000)$ is the value of the regression line at the year 2000, $m$ is the value of the Theil-Sen slope and $\Delta r$ is the averaged error on the residuals computed based on the difference between the linear trend and the yearly mean values of the regional time series.

In order to allow for consistent comparisons, the trend is provided as a relative trend (\%/yr) with respect to the first year of the time period (2000).

\clearpage
\bibliographystyle{copernicus}         % change bibliography-name after each
\renewcommand\bibname{References}      % bibliographystyle command!
\addcontentsline{toc}{section}{References}
\bibliography{trends}

