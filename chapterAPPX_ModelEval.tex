% do this here for safety!!!
\setcounter{page}{1}

%Leave this here in order to start the whole on an odd-numbered page
\cleardoublepage
\chapter[2019 Model Evaluation]{Model Evaluation}
\label{ch:appx_modeleval}

\old{TODO}

%The purpose of this appendix is to make readers of this report better
%aware of our web resources (evaluation report/supplementary material)
%\vspace{30pt}

The EMEP MSC-W model is regularly evaluated against various kinds of measurements, including ground-based, airborne and satellite measurements. As the main application of the EMEP MSC-W model within the LRTAP Convention is to assess the status of air quality on regional scales and to quantify long-range transboundary air pollution, the focus of the evaluation performed for the EMEP status reports is on the EMEP measurement sites.

Only parts of this evaluation are included in the printed version of the EMEP status report. A more comprehensive collection of maps, graphs and statistical analyses, including a more detailed discussion of model performance, are freely available as supplementary material from the MSC-W report page on the EMEP website:\\ \url{https://emep.int/mscw/mscw_publications.html}

This year, the evaluation report is found under the link 'Supplementary material to EMEP Status Report 1/2020'. It contains a comprehensive evaluation of the EMEP MSC-W model for air concentrations and depositions in 2018. The report is divided into three chapters, dealing with pollutants responsible for eutrophication and acidification \citep{WEB2020:SN}, ground level ozone and nitrogen dioxide
\citep{WEB2020:O3}, and particulate matter \citep{WEB2020:PM}, respectively.\\

The agreement between model and measurements in 2018 is visualized as:
\begin{itemize}
\item scatter plots for the EMEP MSC-W model domain;
\item time series for individual EMEP stations;
\item horizontal maps combining model results and EMEP measurement data.
\end{itemize}

Tables summarize common statistical measures of model score, such as bias, root mean square error, temporal and spatial correlations and the index of agreement (see Chapter \ref{ch:Intro}).

This type of model evaluation is performed on an annual basis and can be downloaded from the same web page also for previous years.\\

\vspace{1cm}

A major effort this year has been put into the development of a web
interface that presents a detailed evaluation against measurements from
the European Environment Agency's (EEA) Air Quality e-Reporting Database:\\
\url{https://aerocom-evaluation.met.no/main.php?project=emep}\\
On that page the user can select the classification of measurement data
(rural, urban, non-traffic, or all stations) and view a large number of
statistical parameters (bias, correlation, root mean square error, etc.).

The web interface displays the co-located observational and model data sets
and contains:
\begin{itemize}

\item daily and monthly time series for each station, or averaged per
  country (or the whole area covered by the model and the measurement network);
\item seasonal- and annual-mean diurnal variation for each of the seven days of the week;
\item statistics and scatter plots calculated for each station and country;
\item an overall evaluation of the results using statistics calculated for
  each country or the whole area covered by the model and the measurement network.
\end{itemize}
In all cases the statistics are calculated using monthly resolution
by default. Daily statistics are available by adding \&stats=daily to the
site URL given above.

Evaluation is made for the following chemical species and indicators:
NO$_{2}$, O$_{3}$, PM$_{2.5}$ and PM$_{10}$, and O$_{3}$max (maximum daily
ozone).
Different types of visualization (bar charts, line charts, tables,
etc.) are available for viewing and for download. The measurement data
have been retrieved from the validated {\em E1a} stream of EEA and further
harmonized and quality controlled by the GHOST tool (Globally Harmonised
Observational Surface Treatment) developed at the Barcelona Supercomputing
Center (BSC).

For supplemental evaluation of Elemental Carbon (EC), the modelled
absorption coefficient (mainly due to EC) is compared to surface \textit{in-situ}
observations of the aerosol light absorption coefficient, accessed through
the Global Atmospheric Watch - WDCA database EBAS
(\url{http://ebas.nilu.no/}). More details about this can be found in Chapter~\ref{ch:EC}.

%\cleardoublepage

\bibliographystyle{copernicus}         % change bibliography-name after each
\renewcommand\bibname{References}      % bibliographystyle command!
\addcontentsline{toc}{section}{References}
\bibliography{Refs,EMEP_Reports}

