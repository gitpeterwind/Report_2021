\chapter*{Executive Summary}

\old{TODO}

This report presents the EMEP activities in 2020 and 2021 in relation to transboundary
fluxes of particulate matter, photo-oxidants, acidifying and
eutrophying components, with focus on results
for 2019. It presents major results of the activities related to
emission inventories, observations and modelling. The report also
introduces specific relevant research activities addressing EMEP key
challenges, as well as technical developments of the observation and
modelling capacities. This year special attention has been given to the trends in air pollution during the last decades, in support of the Gothenburg Protocol review.

\subsubsection*{Measurements and model results for 2019} %Wenche, Svetlana, Sverre, Hilde to write
In the first chapter, the status of air pollution in 2019 is presented, combining 
meteorological information and emissions with numerical simulations using the EMEP MSC-W model together with observed air concentration and deposition data.

Altogether 33 Parties reported measurement data for 2019, from 168 sites in total. 
Of these, 120 sites reported measurements of inorganic ions in precipitation and/or 
main components in air; 73 of these sites had co-located measurements in both air and 
precipitation. The ozone network consisted of 138 sites, particulate matter was measured at 
78 sites, of which 50 performed measurements of both \PM[10] and \PM[2.5]. 
In addition, 56 sites from 21 Parties reported at least one of the components required in the advanced EMEP measurement program (level 2). However, very few sites provided a complete level 2 program, i.e. only 12 sites have implemented all the required aerosol parameters. 

The mean daily max O$_3$, SOMO35 and AOT40 all show a distinct gradient with levels increasing from north to south, a well established feature for ozone reflecting the dependency of ozone on the photochemical conditions. The geographical pattern in the measured values is fairly well reflected by the model results for all these three metrics. Peak levels of surface O$_3$ were high in 2019 and this was linked to extreme heat waves in June and July. The national temperature records were broken in France, United Kingdom and Germany this year and associated with these episodes very high ozone levels were observed. Some stations registered the highest peak ozone levels since the mid 1990s. Many countries reported levels above EU's information threshold (180 \ug) and some even above the alert threshold (240 \ug). This confirms the strong link between weather conditions and surface ozone and is a signal that future climate change could have a substantial influence on the frequency and intensity of ozone episodes in Europe.

%Svetlana & Wenche update: Updated 31.8.2021
Overall, the year of 2019 was quite moderate with respect to PM air pollution in Europe, which was due to dominating meteorological conditions with a mild winter and excessive precipitation in the cold seasons (especially autumn) in many parts of Europe. In addition, the spring/summer period was (with some exceptions) relatively warm which favoured evaporation of semi-volatile inorganic and organic aerosols.
The results from the EMEP MSC-W model simulations and the observations show general increase of the annual mean levels of \PM[10] and \PM[2.5] over land from north to south. The regional background concentrations are below 2-5 \ug in Northern Europe, increasing to 5-15 \ug in the mid-latitudes and further south. The annual mean \PM[10] are in excess of 20 \ug in the Po Valley and on Cyprus. The observations also show \PM[10] concentrations above 20 \ug in Slovakia. \PM[2.5] concentrations are below 10 \ug over most of EMEP domain, or otherwise between 10 and 15 \ug in parts of Benelux, Poland, Hungary and some of Balkan countries. Furthermore, the model calculates high PM for the regions east of the Caspian Sea and over the southern Mediterranean, with annual mean concentrations in excess of 50 \ug due to windblown dust from the arid soils and deserts. There is a good general agreement between the modelled and observed distributions of annual mean \PM[10] and \PM[2.5], with correlation coefficients of 0.60 and 0.72, respectively. Overall, the model underestimates the observed annual mean of \PM[10] by 12\% and \PM[2.5] by 13\% .

Much rain in November-December combined with the mild winter temperatures resulted in the absence of major PM episodes in 2019 typical for Europe in winters. Model results and EMEP observational data show that the annual mean regional background \PM[10] concentrations were below the EU limit value of 40 \ug for all of Europe in 2019. The model calculates annual mean \PM[10] above the WHO recommended AQG of 20 \ug for only small regions in the Po Valley and western Turkey. The highest observed annual mean \PM[10], exceeding the AQG of 20 \ug, were only registered at two Slovakian and one Greek site. Further, the joint model and observational results show that annual mean regional background \PM[2.5] concentrations in 2019 were below the EU limit value of 25 \ug (except in the Po Valley according to the model). However, there were observed cases of exceedance of the WHO AQG value of 10 \ug at nine sites, with the highest values at the Hungarian site HU0004 with 16 \ug, followed by German DE0044 with 14 \ug and Italian IT0004 with 13 \ug. Overall, considerable fewer days with PM exceedances have been observed and modelled for EMEP sites for 2019. \PM[10] daily concentrations in excess of 50 \ug were observed at 31 out of 67 sites in 2019, but no violations of the \PM[10] EU limit value (more than 35 exceedance days) were registered. Still, 12 sites had more than 3 exceedance days recommended by WHO AQG’s (with the highest number of observed exceedance days being 7 at DE0001 and 6 at LV0010 and RS0005). Regional background \PM[2.5] concentrations exceeded the WHO AQG recommended limit of 25 \ug at 37 out of 51 stations in 2019. Among those, at 21 sites the number of exceedance days were more than 3 (the recommended limit according to WHO AQG).  The highest number of exceedance days was 40 at HU0002 and IT0004, followed by 30, 26 and 24 at DE0044, PL0005 and AT0002, respectively. Most of the exceedances registered at the central European sites occurred during the winter and in the spring, while rather few exceedances occurred in the wet autumn of 2019. By contrast, at the Mediterranean sites the exceedances were more frequent during summer. PM exceedances simulated by the EMEP MSC-W model correspond in general quite well with the EMEP observations, with some model’s tendency of underestimating the occurrence of exceedances for Central European sites, while overestimating those for some of the Mediterranean sites heavily influenced by desert dust.



%PM pollution was rather moderate in 2018, in particular in Central, Western and Southern Europe (10-30\% lower compared to the 18-year average) due to the relatively mild winter and more precipitation. Only in Turkey, the Caucasus region, Ukraine, the Baltic countries and southern parts of Finland and Sweden, \PM[10] and \PM[2.5] levels were higher. 
%EMEP MSC-W model calculations and EMEP observations show a general increase of annual mean \PM[10] and \PM[2.5] over land from north to south, with concentrations being below 2-5 \ug in Northern
%Europe and increasing to 5-15 \ug in the mid-latitudes and further south.
%, with \PM[2.5] levels being somewhat lower than those of \PM[10]. 
%The distribution of the regional background PM is fairly homogeneous over most of Central and Western Europe, with somewhat elevated \PM[10] levels of 15-20 \ug in the Po Valley and the Benelux. The observations also show elevated \PM[10] concentrations in Poland, Czechia and Hungary, while the model calculates high PM for the regions east of the Caspian Sea and over the southern Mediterranean (heavily influenced by Saharan dust). 

%There is good agreement between the modelled and observed \PM[10] and \PM[2.5], with annual mean model biases of -22\% and -14\% and correlation coefficients of 0.66 and 0.81, respectively.

%Model results and EMEP observational data show that the annual mean regional background \PM[10] concentrations were below the EU limit value of 40 \ug in Europe in 2018, but some exceedances of the WHO recommended AQG of 20 \ug occurred. 
%The annual mean \PM[2.5] concentrations in 2018 were below the EU limit value of 25 \ug (except modelled \PM[2.5] for the Po Valley). However, exceedances of the WHO AQG value of 10 \ug by annual mean \PM[2.5] were observed at seventeen sites.%, with the highest values registered in Hungary (16 \ug at HU0003 and 15.5 \ug at HU0002).

%Out of the 62 sites, days with \PM[10] exceedances of 50 \ug were observed at 36, but in all cases with fewer than the upper EU-limit of 35 days per year registered. Still, 18 sites had more than 3 exceedance days (the upper limit recommended by the WHO AQGs). %The highest numbers of days with observed exceedances of \PM[10] were 25 at CY0002 and SK0007, and 17 at CZ0003. 
%Daily mean \PM[2.5] concentrations exceeded the WHO AQG recommended of value 25 \ug at 31 out of 41 stations in 2018. Among those, at 21 sites the number of exceedance days were more than 3 (the recommended limit according to WHO AQG).  
%The modelled numbers of exceedance days in 2018 show in general a good correspondence with the observations, with somewhat better agreement for \PM[10] than for \PM[2.5]. The model has a tendency of underestimating the frequency of exceedance days at some Central European sites
%and overestimating at some Mediterranean sites. 
%The majority of PM exceedances occurred during the winter and autumn 2018 at Central European sites, whereas during the summer at the Mediterranean sites. Remarkably, the largest number of \PM[2.5] exceedances at three out of four German sites occurred in spring, while much fewer occurred during the cold seasons.





\QUERY{Something about wet dep, Hilde}


%\subsubsection*{The summer of 2018} OLD
%During April/May-August parts of Europe experienced a persistent heat wave. Northern Europe and in particular the Nordic countries and the UK were most heavily affected, but also central parts of the continent experienced long-lasting heat and drought in summer. The heat and drought affected the atmospheric level of pollutants in many ways. Tropospheric ozone is strongly tied to the meteorological conditions, and hot, sunny and dry weather conditions can lead to increased ozone levels in many ways. 

%The continent experienced a series of ozone episodes during April/May-August. The most pronounced lasted from the last part of July to the first days of August when areas from Southern Italy to Scandinavia were affected and peak values exceeding 100 ppb were seen at several sites. Another marked ozone episode was seen one month earlier, from the end of June to the first week of July, mainly affecting Central Europe.

%Long-term time series of EMEP ozone measurements show a downward trend in the metrics SOMO35 and AOT40 reflecting the reduced emission of precursors, whereas the levels in 2018 were clearly elevated. This is a clear message that an efficient abatement of surface ozone depends on future climate change as well as on the reduction of \nox and VOCs emissions. The high levels of AOT40 in 2018 should, however, not be used directly to assess ozone effects on vegetation. The extreme drought likely reduced the uptake of ozone in the plants, leading to increased near-surface ozone.

%\PM[10] and \PM[2.5] levels in Northern Europe were generally increased all of summer 2018. The increase was mainly attributed to fine (\PM[2.5]) OC (organic carbon) and mineral dust, which formation and emission were governed by the favourable meteorological conditions. Some sites experienced a substantial 250-400 \% increase in the mineral dust loading compared to the  2013-2017 period, with source regions both within and outside of Europe. Isoprene levels were elevated compared to previous years, as were 2-methyltetrols, which are oxidation products of isoprene partitioning to the particulate phase. Unchanged EC levels and reduced BB (biomass burning) and PBAP (primary biological aerosol particles) tracer levels point to BSOA (biogenic secondary organic aerosol) formation as the explanation of the increased level of fine OM in summer 2018.



\subsubsection*{Status of emission reporting}
In 2021, 48 out of 51 Parties (94$\%$) submitted emission inventories to the EMEP Centre on Emission Inventories and Projections (CEIP), and 42 Parties reported black carbon (BC) emissions. As 2021 is a reporting year for large point sources (LPS) and gridded emissions, 32 Parties reported information on LPS while 26 Parties reported gridded data. The quality of reported emission data differs significantly across countries, and the uncertainty of the data is considered to be relatively high.

After the first round of submissions in 2017, 2021 was the second year for which EMEP countries were obliged to report gridded emissions in  0.1{\degrees}$\times$0.1{\degrees} lon\-gi\-tude/la\-ti\-tude resolution. Until June 2021, 34 of the 48 countries which are considered to be part of the EMEP area reported sectoral gridded emissions in this resolution. For remaining areas, missing emissions are gap-filled and spatially distributed using expert estimates.

Estimates of PM emissions as currently provided by Parties have a number of major uncertainties, and there is a clear need for clarification and standardisation of the methods used to define and report PM emissions. 
Previous work has clearly shown that the definitions behind national emission estimates are inconsistent in their treatment of condensable organics: some countries explicitly do not include condensables in their PM inventories, some likely include condensables and for some it is mixed or unclear.

%Parties were asked to include a table with information on the condensable component in their reporting of PM emissions. This year, 23 Parties provided such information. 
The treatment of condensable organics in emission factors is best known for the emissions from the energy sector and road transport, while it is less clear for small-scale combustion, which is one of the sources where the largest impact on the emission factor occur.

%In March 2020, MSC-W hosted an expert workshop on condensables, which brought together experts in emissions, measurements, inventories, and policy from Europe and North America, and created a much better understanding of the issues and
%possible approaches for dealing with this important class of compounds.
%Discussions at the NMR workshop confirmed that even when countries did include condensables, there were significant differences in the methodologies used.
To improve the quality of the input data for air quality models, and following a decision of the EMEP Bureaux, the group of experts at the expert workshop on condensable organics hosted by MSC-W in 2020 agreed on the following approach: 1) in year one (2020) use the so-called REF2 emission data provided by TNO, which include condensable organics, as an initial estimate for residential combustion emissions and 2) in subsequent years these top-down estimates should be increasingly replaced by national estimates once procedures for quantifying condensable organics in a more harmonized way are agreed on and implemented. 

In 2021, CEIP in co-operation with TNO prepared a list of Parties where it could be assumed with a good degree of certainty that the condensable component is mostly included in PM emissions for GNFR sector C. For these Parties the reported PM emissions were used, while for other Parties the TNO REF2.1 data were used for GNFR sector C. The resulting GNFR C data set were combined with official EMEP emissions into the so-called EMEPwREF2.1C emission dataset. This emission data set has been used in the assessment of the air quality situation in Europe and the source receptor calculations for 2019 made this year. 

%The assessment of the air quality situation in Europe and the source receptor calculations for 2019 made this year have been conducted with the EMEPwREF2.1C emission dataset.
%official EMEP emissions combined with TNO REF2.1 emissions for the GNFR sector C for PM (EMEPwREF2.1C). 

%The 1999 Gothenburg Protocol lists emission reduction commitments of NO$_x$, SO$_x$, NH$_3$ and NMVOCs for most of the Parties to the LRTAP Convention for the year 2010\COMMENT{WHY DO THEY TALK ABOUT THE 2010 targets and not the 2020 targets?Should we include this at all in the Summary?}. These commitments should not be exceeded in 2010 nor in subsequent years. When considering only reported data, approved adjustments and fuel use data of the respective countries, it can be seen that in the year 2019 North Macedonia could not reduce their \sox emissions below their respective Gothenburg Protocol requirements, and that Croatia and Spain are above their 1999 Gothenburg Protocol ceilings concerning \nhiii. For \nox and NMVOC all countries were below their individual ceilings in year 2019.
%Emissions from international shipping in different European seas were estimated based on the CAMS global shipping emission dataset for the year 2019.

\subsubsection*{Trends in air pollution 2000-2019}
In December 2019, the Executive Body launched the review of the Gothenburg Protocol as amended in 2012. In order to support the review and assess the progress made towards achieving the environmental and health objectives of the Protocol, we present an assessment of the trends in air pollution in Europe for the period 2000-2019 based on long term observational data from the EMEP network as well as EMEP MSC-W model calculations.
We analyze trends in air concentrations for ozone, sulphur dioxide, particulate matter (and their species; sulphate, nitrate, ammonium, elemental carbon and organic carbon), oxidized and reduced nitrogen as well as wet deposition of sulphur and nitrogen species. In addition, we present trends in some indicators of health and vegetation risk (SOMO35, exceedances of WHO guideline values for PM$_{2.5}$, AOT40 for forests and crops, and exceedances of critical loads for acidification and eutrophication for every 5 year since 2000).
%Note on validity Hilde
Unfortunately, the EMEP observational network is dominated by sites in the western part of the EMEP domain and has hardly any coverage in the EECCA and western Balkan area. Therefore, the assessment discussed below is only valid for a part of the EMEP domain. Furthermore, the developments of emissions in the western and eastern part of the EMEP domain have followed different patterns, with clear decreases of most pollutants in the western countries but more stable (albeit gradually decreasing for most pollutants) in the eastern part of the domain over the 2000-2019 period. Thus, the trends in the eastern part of the EMEP domain are expected to be different than those presented here for the western part of the domain. Note also, that the uncertainties related to emissions and their trends in the eastern countries are large.

%SOX: Wenche
The SO$_x$ emissions in the EU27+UK+EFTA countries have declined by more than 80\% (-4.3\% yr$^{-1}$) the last two decades. This is reflected in both the observed and modelled trends for all the atmospheric sulfur components. The average total reductions in observations for the last 20 years (2000--2019) are 74\%, 61\% and 60\% for \soii, \soiv in aerosols and in wet deposition respectively, while the trends in model calculations are somewhat higher: 97\%, 72\% and 81\%, respectively.

%NOx Hilde
We find that oxidized nitrogen in precipitation and in air (NO$_2$, HNO$_3$, NO$_3^{-}$ aerosol and the sum of HNO$_3$ and NO$_3^{-}$ aerosol) has been decreasing since 2000. However, the decrease in the observations (-24\%, 1.2\% yr$^{-1}$) for NO$_2$ and for wet deposition of oxidized nitrogen (-26\%, -1.4\% yr$^{-1}$) are smaller than in model calculations (-42\%, -2.2\% yr$^{-1}$ for NO$_2$) and -40\% (-2.3\% yr$^{-1}$) for wet deposition of oxidized nitrogen and emissions (-48\% for EU27+UK+EFTA countries) for 2000--2019.

%redN Hilde 
For reduced nitrogen, the observations and the model calculations confirm that very small reductions of ammonia emissions have been achieved during the 2000--2019 period. 
Both observations and model calculations find very few significant trends in wet deposition of reduced nitrogen.
Total ammonium (NH$_3$ + NH$_4$) in air show larger negative changes of about -1.45\% yr$^{-1}$ (observations) and -1.33\% yr$^{-1}$ (model calculations) and with a larger fraction of the sites having significant trends. This trend is larger than the reduction in ammonia emissions for EU27+UK+EFTA countries (-12\%), in total 27\% (25\%) in observations (model calculations) for 2000-2019. The trend for ammonium aerosols is even more negative, -2.6 \% yr$^{-1}$ both in observations and model calculations. Very few sites have long term time series of ammonia in air (8 sites), and for very few of those sites the changes are statistically significant. However, on average, the changes observed and calculated are positive. 

These large differences between the changes found for the different reduced nitrogen components can be explained by the interaction of ammonia with the sulphur and nitrogen components. It can be noted that the results imply that the contribution of ammonia emission to aerosols have been largely reduced during the 2000--2019 period, due to the impact of SO$_x$ and NO$_x$ emission reductions. 

\QUERY{PM and exceedances of WHO guideline values for PM$_{2.5}$: Svetlana}


%OC and EC and 'do the organic fraction increase?': David & Karl Espen
For EC and OC we have calculated observed and modelled trends for 15 sites across Europe for the period 2010--2019. A reduction (of ca. 4.5~\%/yr) in observed EC was found, and a similar reduction was found for OC in the winter months (when OC is expected to be most sensitive to primary anthropogenic emissions rather than to OA associated with biogenic sources). These trends suggest that abatement measures are having some success in reducing both EC and OC in Europe, especially in wintertime.  The model reproduced modelled EC values and trends quite well, but underpredicted OC, both in terms of absolute values and trends. This underprediction is at least partly due to the omission of condensable organics in the reported emissions from many countries. 
%
The carbonaceous fraction appears particularly important for a
further reduction in the observed \pmfine mass concentration in Europe,
although effort is needed to separate its natural and anthropogenic
fraction to get a quantitative overview of the abatable fractions.


%Ozone Sverre
Trends in six annual percentiles of daily maximum O$_3$ as well as three aggregated metrics (AOT40 for crops, AOT40 for forests and SOMO35) were calculated for the period 2000-2019. For the high percentiles as well as for the three aggregated metrics both the model values and the observations show a decrease during this period when averaged over the stations. The observations show a mean decrease of 0.5-1.5 \% $yr^{-1}$ relative to 2000 over this 20 years period for the three aggregated metrics while the model calculates somewhat larger reductions, of the order of 0.9-2\% $yr^{-1}$. For the 99th percentile of the annual daily maximum values, corresponding to the 4th highest ozone concentration, the mean of the observed and modelled values agree very well for sites north of 49 \degrees N, both showing a reduction of the order of 0.5  \% $yr^{-1}$ relative to 2000. For sites south of 49 \degrees N the observations show a similar relative reduction on average whereas the model calculates stronger reductions. Comparisons of the observed and modelled absolute concentration levels of O$_3$ reveal that the model underpredicts the high percentile and overpredicts the AOT40 levels. The underprediction increases with higher percentiles and is strongest for sites south of 49 \degrees N. The modelled downward trend in the high percentiles is, however, of the same order if comparing absolute levels. For SOMO35, the trend in the model values agrees well with the observed levels.

Comparison of trends in measured and modelled VOCs is not included in this report. The main reasons being the lack of established procedures for doing such a comparison as well as the lack of proper monitoring sites with sufficient data capture and sufficient homogeneity in the monitoring. The lumped VOCs in the model could not be compared directly with measured species. The topic of VOC trends will be investigated in more detail in the near future.  



%Exceedances of critical loads Hilde












%Technical EMEP Developments
\subsubsection*{Model improvements} %DS updated 23/7/2021

The EMEP MSC-W model code has been upgraded in a number of ways. 
A 19-sector emissions system (GNFR-CAMS) was introduced into the code. Emissions for soil NO, DMS, and aircraft were updated using results from the CAMS81 project. 
The  fine/coarse fractions for sea-salt and nitrate were modified. Emissions and the chemical mechanism were adapted to explicitly track GNFR sector C emissions (residential). Revised global monthly emission factors were produced, and use of a global time-zone map introduced. The default Kz and Hmix schemes were changed. The local fraction methods were extended, and code for the many configuration options was simplified. 




\subsubsection*{Development in the monitoring programme} %Wenche/NILU to update
The two last chapters of the report present the implementation of the EMEP monitoring strategy and general development in the monitoring programme including data submission. There are large differences between Parties in the level of implementation, as well as significant changes in the national activities during the period 2000-2018. With respect to the requirement for level 1 monitoring, 40\% of the Parties have had an improvement since 2010, while 33\% have reduced the level of monitoring. For level 2 monitoring there has been a general positive development in recent years. However, only few sites have a complete measurement program.

The complexity of data reporting has increased in recent years, and it is therefore now mandatory for the data providers to use the submission and validation tool when submitting data to EMEP to improve the quality and timeliness in the data flow. 
There is a need for improvements in the reporting, as only half of the data 
providers use the submission tool, and less than 60\% report within the deadline of 31 July.

For the level 2 parameters, there have been large improvements in the data quality and measurement capabilities over the last decade resulting from development in ACTRIS (European Research Infrastructure for the observation of Aerosol, Clouds and Trace Gases) in co-operation with EMEP and the WMO Global Atmospheric Watch Programme (GAW). New routines for quality assurance for VOCs have been developed, including a digital tool (Atmospheric VOC Assessment Tool, @VOC@), which is presented briefly in this report. The tool will help streamline and harmonize the quality checks of VOC data in Europe.


