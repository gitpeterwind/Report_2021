\chapter[Emissions 2019]{Emissions for 2019}
\label{ch:emis2019}

\old{TODO}

{\bf{Bradley Matthews, Katarina Mareckova, Sabine Schindlbacher, Bernhard Ullrich, Robert Wankm\"uller and all CEIP/Umweltbundesamt Austria}}
\vspace{30pt}

In addition to meteorological variability, changes in the emissions
affect the inter-annual variability and trends of air pollution,
deposition and transboundary transport.  
The main changes in emissions in 2018 with respect to previous years
are documented in the following sections.


%\section{Emissions for 2018}
%\label{Emis_2018}

The EMEP Reporting guidelines \citep{UNECE2014} requests all Parties
to the LRTAP Convention to report annually emissions and activity data of air pollutants
(\sox\footnote{``Sulphur oxides (\sox)'' means all sulphur compounds,
  expressed as sulphur dioxide (\soii), including sulphur trioxide
  (\soiii), sulphuric acid (\sulacid), and reduced sulphur compounds,
  such as hydrogen sulphide (H${_2}$S), mercaptans and dimethyl
  sulphides, etc.}, \noii\footnote{``Nitrogen oxides (\nox)'' means
  nitric oxide and nitrogen dioxide, expressed as nitrogen dioxide
  (\noii).}, CO, NMVOCs\footnote{``Non-methane volatile organic
  compounds'' (NMVOCs) means all organic compounds of an anthropogenic
  nature, other than methane, that are capable of producing
  photochemical oxidants by reaction with nitrogen oxides in the
  presence of sunlight.}, \nhiii, HMs, POPs,
PM\footnote{``Particulate matter'' (PM) is an air pollutant
  consisting of a mixture of particles suspended in the air. These
  particles differ in their physical properties (such as size and
  shape) and chemical composition. Particulate matter refers to:\\  
(i) ``PM$_{2.5}$'', or particles with an aerodynamic diameter equal to or
  less than 2.5 micrometers ($\mu$m);\\ 
(ii) ``PM$_{10}$'', or particles with an aerodynamic diameter equal to or
  less than 10 ($\mu$m).} and voluntary BC). Further, every four years,
projection data, gridded data and information on large point sources (LPS)
have to be reported to the EMEP Centre on Emission Inventories and Projections (CEIP).

%\subsection{Reporting of emission inventories in 2020}
\section{Reporting of emission inventories in 2020}

Completeness and consistency of submitted data have improved significantly since EMEP started collecting information on emissions. Data of at least 45 Parties each year were submitted to CEIP for in years 2012-2019 (see Figure~\ref{fig:CEIP1}). As of 1 June 2020, 43 Parties (84\%) submitted inventories\footnote{The original submissions from the Parties can be accessed via the CEIP homepage on \url{https://www.ceip.at/status-of-reporting-and-review-results/2020-submissions}.} in 2020; eight Parties\footnote{Albania, Azerbaijan, Bosnia and Herzegovina, Georgia, Liechtenstein, Monaco, Russian Federation and USA} did not submit any data and 40 Parties reported black carbon (BC) emissions (see section~\ref{sec:bc}). Although 2020 was no reporting year for large point sources (LPS), gridded emissions and projections, seven Parties reported information on LPS, nine Parties reported gridded data, and five Parties reported projection data \citep{CEIP2020}.

\begin{figure}[h]
\centering
{\includegraphics*[viewport=55 290 550 500,clip,scale=0.7]{FIGS_CEIP/Figure_1.pdf}}
\caption{Parties reporting emission data to EMEP since 2002, as of 1 June 2020.}
\label{fig:CEIP1}
\end{figure}

The quality of the submitted data across countries differs quite significantly. By compiling the inventories, countries have to use the newest available version of the {\it EMEP/EEA air pollutant emission inventory guidebook}, which is the version of 2016 \citep{EmisInvGuide2016}. However, many countries still use the 2013 Guidebook \citep{EmisInvGuide2013} or older versions. Uncertainty of the reported data (national totals, sectoral data) is considered relatively high, the completeness of reported data has not turned out satisfactory for all pollutants and sectors either.

Detailed information on recalculations, completeness and key
categories, plus additional review findings can be found in the annual
CEIP technical country  reports{\footnote{\url{https:www.ceip.at/status-of-reporting-and-review-results/2020-submissions}}}. 

%\subsection{Black Carbon (BC) emissions}
\section{Black Carbon (BC) emissions}  
\label{sec:bc}

Over the last decade, black carbon (BC) has emerged as an important air pollutant in terms of both climate change and air quality.  

\begin{figure}[h]
\centering
{\includegraphics*[viewport=25 260 585 525,clip,scale=0.75]{FIGS_CEIP/Figure_2.pdf}}
\caption{Black carbon emissions of the year 2018 as reported by CLRTAP Parties.}
\label{fig:CEIP2}
\end{figure}


The emerging significance of BC is mirrored in developments in the international policy arena with respect to emissions reporting.
Since the Executive Body Decision 2013/04, Parties to the LRTAP Convention have been formally encouraged to
submit inventory estimates of their national BC emissions, and in 2015 the reporting templates were updated to include
BC data emissions.

In addition to reporting under CLRTAP, EU member states are also encouraged to submit BC emissions
estimates as part of their emissions reporting under the National Emissions Ceilings (NEC) Directive (2016/2284/EU).


While BC is not a mandatory pollutant to be reported under CLRTAP, CEIP continues to monitor and review the level of BC reporting by the Convention's Parties. A brief overview of BC emissions estimates submitted by EMEP countries in 2020 is given in Figure~\ref{fig:CEIP2} .

Since enabling the reporting of BC, a total of 44 CLRTAP Parties have reported BC emissions estimates\footnote{ As of 1 June 2020 Albania, Austria, Bosnia and Herzegovina, Liechtenstein, Luxembourg, Russia, and Turkey have yet to report estimates of national BC emissions.}. In this round of reporting, 25 CLRTAP Parties submitted a complete time series of national total BC emissions (1990-2018), while 33 CLRTAP Parties submitted a complete time series from 2000 onwards. Furthermore, 38 EMEP Parties have provided national total BC emissions estimates for the year 2018.

For more detailed information on BC consult the annual CEIP technical inventory review report \citep{CEIP2020}.

%\subsection{Inclusion of the condensable component in PM emissions}
% DAve added word reported:
\section{Inclusion of the condensable component in reported PM emissions}
\label{sec:EmisSVOC}

The condensable component of particulate matter is a class of compounds of low volatility that may exist in equilibrium between the gas and particle phase. It is probably the biggest single source of uncertainty in PM emissions.
%% The condensable component of particulate matter is released as a gas but forms particles when it is diluted and cools
%% down.
Currently the condensable component is not included or excluded consistently in PM emissions reported by Parties
of the LRTAP Convention. Also in the EMEP/EEA Guidebook \citep{EmisInvGuide2019} the condensable fraction is not consistently included or
excluded in the emission factors. Various EMEP centres and task forces and other stakeholders jointly discuss the topic and work on progress in this area. An important activity this year was the workshop organised by MSC-W that will result in a workshop report \citep{NMR-SVOC}. However, at the moment PM emissions reported by Parties to the LRTAP Convention are not directly
comparable, which has implications on the modeling of overall exposure to PM.



\begin{table}
  \caption{Information on the inclusion of the condensable component in PM$_{10}$ and PM$_{2.5}$ emission factors.}
\centering
%{\includegraphics*[width=0.95\textwidth]{FIGS_CEIP/CEIP_table1.png}}
%{\includegraphics*[viewport=100 335 545 765,clip,width=0.95\textwidth]{FIGS_CEIP/CEIP_table1.pdf}}
{\includegraphics*[viewport=70 190 525 775,clip,width=0.95\textwidth]{FIGS_CEIP/CEIP_table1_new.pdf}}
\label{tab:CEIP1}  
\end{table}


Parties were asked to include a table with information on the inclusion of the condensable component in PM$_{10}$ and PM$_{2.5}$
emission factors for the reporting under the CLRTAP convention in 2019 and 2020. This table has been added to the revised
recommended structure for informative inventory reports (IIRs)\footnote{\url{https://www.ceip.at/reporting-instructions/annexes-to-the-2014-reporting-guidelines}}. Twenty-one Parties
provided information on the inclusion of the condensable component in  PM$_{10}$ and PM$_{2.5}$ emission factors (Austria,
Belgium, Croatia, Denmark, Estonia, Germany, Finland, France, Latvia, Lithuania, the Netherlands, Poland, Portugal, Romania, Slovakia, Slovenia,
Spain, Sweden, Switzerland  and United Kingdom)\footnote{Status 18 May 2020}. This reporting is a first step towards a better understanding of the reported PM data. The information that Parties provided on whether the condensable component is included in PM emissions was quite heterogeneous. The status of inclusion or exclusion is best known for emissions from the energy sector and road transport, for which many Parties submitted information. For example for {\it ``1A3bi Road transport passenger cars''} 17 out of 18 Parties that provided condensable information for this source category report emissions to be included and only one Party states that the status of inclusion is unknown. For most other sectors, Parties either indicated that it is ''unknown'' whether the condensable component is included in the PM emissions or they did not provide any information. 


Small-scale combustion sources make a notable contribution to total PM emissions. For all Parties that reported PM$_{2.5}$
emissions for {\it ``1A4bi Residential: Stationary''} for the year 2018\footnote{ Status as of 7 May 2020.} emissions from this
source category contributed 47\% to the national total PM$_{2.5}$ emissions. Small-scale combustion is one of the sources
where the inclusion of the condensable component has the largest impact on the emission factor. For example, for
conventional woodstoves, one of the most important categories in Europe, the emission factors excluding and including
the condensable fractions may differ by up to a factor of five \citep{DeniervanderGon2015}.  Here the
status of the inclusion was less clear. Of the twenty-one Parties that provided information for ''1A4bi Residential:
Stationary'' four parties reported the condensable component to be included and five Parties to be excluded. The
other Parties reported ``unknown'', ``partially included'' or provided information on a more detailed level with
different status of inclusion (see Table~\ref{tab:CEIP1}).

% Added by Dave: 24/8
%
For the modelling work this year, and following a decision of \citet{EMEPBureaux2020}, EMEP MSC-W has made use of the so-called `Ref2' emissions provided by TNO, which include condensable organics. These data and their usage are described further in  Chapter~\ref{chap:TNO} and Chapter~\ref{ch:Condensables}.


%\subsection[Comparison of 2017 and 2018]{Comparison of 2017 data (reported in 2019) and 2018 data (reported in 2020)}
\section[Comparison of 2017 and 2018]{Comparison of 2017 data (reported in 2019) and 2018 data (reported in 2020)}
\label{emischanges}

The comparison of 2017 emissions (reported in 2019) and 2018 emissions
(reported in 2020) showed, that for 32 countries data changed by more
than 10\% for one or several pollutants (see Figure~\ref{fig:CEIP3} and Table~\ref{tab:emisdiffMAIN}-\ref{tab:emisdiffPM}).
These changes can be caused by real emission reductions or increases, or recalculations made by the respective country.

\begin{figure}[h]
\centering
{\includegraphics*[viewport=30 270 585 530,clip,scale=0.7]{FIGS_CEIP/Figure_3.pdf}}
\caption{Emission changes between 2017 and 2018 in reported data (only changes larger than 10\% are shown).
}
\label{fig:CEIP3}
\end{figure}

In five countries, both \nox and CO emissions changed by more than 10\%. For NMVOCs, emissions changed in two countries by more than 10\%. For \sox, emissions changed by more than 10\% in 14 countries, while for \nhiii the change of emission levels were less than 10\% in each country. Of the PMs, emissions changed by more than 10\% in ten countries for PM$_{2.5}$, in seven countries for PM$_{10}$ and in nine countries for PM$_{coarse}$\footnote{PM$_{coarse}$ emissions are not reported by Parties but calculated as difference between PM$_{10}$ and PM$_{2.5}$ emissions.} (see Figure~\ref{fig:CEIP5} and Table~\ref{tab:emisdiffMAIN}-\ref{tab:emisdiffPM}). The largest changes occurred in Belarus, Kyrgyzstan, Malta and Ukraine.

Reported emissions of \nox were relatively stable, with 2017 to 2018 changes of more than 10\% observed for three countries. For seven and ten countries respectively, \nhiii and NMVOC emissions changed by more than 10\%. For CO and \sox, 2017 to 2018 changes of more than 10\% were observed for 12 and 13 countries, respectively. For emissions of particulate matter, more countries reported changes higher than 10\%. In 16 countries, emissions of both  PM$_{2.5}$ and PM$_{10}$ changed by more than 10\% from 2017 to 2018, while for BC  14 countries reported changes higher than 10\%.


\begin{table} 
  \caption[Emission changes (main pollutants)]{Reported emission changes between 2017 (reported in 2019) and 2018 (reported in 2020) over 10\% for main pollutants.}
%\vspace{15pt}
\centering
{\includegraphics*[viewport=95 105 465 765,clip,width=0.85\textwidth]{FIGS_CEIP/CEIP_APPX1_tableMAIN.pdf}}
%{\includegraphics*[width=0.85\textwidth]{FIGS_CEIP/CEIP_APPX1_tableMAIN.png}} 
\label{tab:emisdiffMAIN}  
\end{table}

\begin{table} 
  \caption[Emission changes (PM)]{Reported emission changes between 2017 (reported in 2019) and 2018 (reported in 2020) over 10\% for PM and BC.}
%\vspace{15pt}
\centering
{\includegraphics*[viewport=105 110 500 745,clip,width=0.85\textwidth]{FIGS_CEIP/CEIP_APPX1_tablePM.pdf}}
%{\includegraphics*[width=0.85\textwidth]{FIGS_CEIP/CEIP_APPX1_tablePM.png}} 
\label{tab:emisdiffPM}  
\end{table}

Changes above 50\% were observed in Armenia, Croatia, Malta, Spain and Turkey. For Armenia, the 121\% and 1000\% increases in PM$_{2.5}$ and BC emissions were caused to a large extent by the increases in reported emissions from the sector {\it ''1A4bi -- Residential: Stationary''}. Such changes are likely  due to a revision of the calculations. However, only estimates for the year 2018 were reported by Armenia in 2020 and new estimates of the previously reported years were not submitted.

For Croatia, the 72\% increase in PM$_{2.5}$ was caused by an upward revision of the time series from 2005 onwards. This increase was mostly due  to the revision of emissions from the sector {\it ''1A4bi -- Residential: Stationary''}.

For Malta, the 69\% decrease in \sox emissions mainly originates in the NFR category {\it ''1A1a -- Public electricity and heat production''}. Malta explained in its IIR that the decline in \sox emissions mirrors the continued decrease in electricity generated from fuel combustion.

For Spain, the 56\% increase in BC emissions was caused by an upward revision of the time series from 2000 onwards. The increase was mostly due to the upward revision of emissions from the sector {\it ''5C2 -- Open burning of waste''}.

%For Turkey, PM$_{10}$ emissions decreased by 69\%, while PM$_{2.5}$ emissions increased by about 1000\%. The changes in both cases were due to revisions of the submitted emissions estimates. The change PM10 emissions was driven largely by a 95\% downward revision of industrial emissions from the sector 2B10a - Chemical industry: Other (please specify in the IIR). The increase in PM2.5 emissions on the other hand was due to the fact that in 2020 Turkey provided PM2.5 emissions estimates for a number of industrial-, stationary combustion- and transport sectors that were previously not estimated in the earlier submissions.

%\subsection{Gothenburg Protocol targets}
\section{Gothenburg Protocol targets}

The 1999 Gothenburg Protocol (GP) lists emission reduction commitments of  \nox,
\sox, NMVOCs and \nhiii for most of the Parties to the LRTAP Convention for the year 2010 (\cite{UNECE1999}). These commitments should not be exceeded in 2010 and in subsequent years either.

In 2012, the Executive Body of the LRTAP Convention decided that adjustments to inventories may be applied in some circumstances (\cite{UNECE2012}). From 2014 to 2020, adjustment applications of ten countries (Belgium, Denmark, Finland, France, Hungary, Germany, the Netherlands, Luxembourg, Spain and the United Kingdom) have been accepted and therefore these approved adjustments have to be subtracted for the respective countries when compared to the targets. In April 2020, Czechia submitted a new adjustment application, which will be approved most likely later this year.

Further, the reporting guidelines (\cite{UNECE2014}) specify that some Parties within the EMEP region (i.e. Austria, Belgium, Ireland, Lithuania, Luxembourg, the Netherlands, Switzerland, the United Kingdom of Great Britain and Northern Ireland) may choose to use the national emission total calculated on the basis of fuels used in the geographic area of the Party as a basis for compliance with their respective emission ceilings.

However, when considering only reported data, approved adjustments and fuel used data of the respective countries, Figure~\ref{fig:CEIP4} indicates that Czechia could not reduce its NMVOC emissions with regard to the Gothenburg Protocol requirements, and that Croatia, Denmark,  Germany, Norway and Spain are above their Gothenburg Protocol ceilings for \nhiii. In terms of \nox emissions, Norway exceeded their ceilings. For \sox all countries were below their individual ceilings.


\begin{figure}[h]
\centering
{\includegraphics*[viewport=25 275 585 515,clip,scale=0.75]{FIGS_CEIP/Figure_4.pdf}}
\caption{Distance to Gothenburg Protocol targets in 2020 (based on reported data).
  Only Parties that ratified the Gothenburg Protocol are included. 
  * Emission data based on fuels used for road transport.
Approved adjustments are considered for  Denmark (NMVOCs, \nhiii), Finland (\nhiii),  Germany (\nox, NMVOCs, \nhiii), Hungary (NMVOCs), Luxembourg (\nox, NMVOCs), the Netherlands (\nhiii, NMVOCs) and Spain (\nox).}
\label{fig:CEIP4}
\end{figure}


%\subsection{Emission trends in the EMEP area}
\section{Emission trends in the EMEP area}
\label{sec:EmisTrend}

To provide a picture as complete as possible of the emission trends in the EMEP area\footnote{The EMEP domain covers the geographic area between 30{\degrees} N-82{\degrees} N latitude and 30{\degrees} 
  W-90{\degrees} E longitude.}, data as used for EMEP models (i.e. gap-filled data) were used for the calculations (see Section~\ref{sec:modeldata}).

The trend indicates that in the EMEP area total emissions of three of the reported pollutants have decreased overall since 2000 (Figure~\ref{fig:CEIP5}). Please note that PM$_{coarse}$ is not reported but rather derived from the reported emissions of PM$_{10}$ and PM$_{2.5}$. Nonetheless it is important to note that PM$_{coarse}$ emissions have increased since 2000.
The presented emission trends are based on gap-filled data as used in the EMEP models, therefore there is a certain uncertainty in the magnitude of this development. The observed decrease is significant for \sox, CO, BC and \nox; 75\%, 91\%, 92\% and 94\% of the respective 2000 emissions. In contrast, the 2018 emissions of NMVOC, PM$_{2.5}$, PM$_{10}$, PM$_{coarse}$ and \nhiii have increased by 6\%, 5\%, 8\%, 14\% and 31\%, respectively. 

\begin{figure}[h]
\centering
{\includegraphics*[viewport=40 205 580 570,clip,scale=0.7]{FIGS_CEIP/Figure_5.pdf}}
\caption{Emission trends 2000--2018 in the EMEP area (based on gap-filled data as used in EMEP models)}
\label{fig:CEIP5}
\end{figure}

\begin{figure}[h]
\centering
     \subfigure[EMEP West]{\includegraphics*[viewport=40 205 580 575,clip,width=0.45\linewidth]{FIGS_CEIP/Figure_6a.pdf}}
    \subfigure[EMEP East]{\includegraphics*[viewport=40 205 580 575,clip,width=0.45\linewidth]{FIGS_CEIP/Figure_6b.pdf}}
\caption{Emission trends 2000-2018 in the EMEP area (based on gap-filled data as used in EMEP models) divided in 2 areas 'EMEP West' (left), 'EMEP East' (right).}
\label{fig:CEIP6}
\end{figure}


A more detailed assessment shows that emission developments in the eastern and western part of the EMEP area seem to follow strongly different patterns (see Figure~\ref{fig:CEIP6})\footnote{The split between the EMEP West region and the EMEP East region according to \url{https://www.ceip.at/countries}. 'North Africa' and sea areas are not included and 'Asian Areas' are included in the EMEP East region.}.

While emissions of all pollutants in the western part of the EMEP domain are slowly decreasing, emissions of all pollutants in the eastern part of the EMEP domain have increased since the year 2000. The emissions in the western parts of the EMEP area are mostly based on reported data, while the emissions in eastern parts are rather often expert estimates due to a lack of (plausible) reporting. One should thus keep this in mind when considering this comparison of emissions. The significant increase in emissions (of all pollutants) in the 'EMEP east' area is mainly influenced by emission estimates made for the remaining Asian Areas in the EMEP domain. These expert estimates are based on aggregated and interpolated gridded emissions from EDGAR \citep{JRC2016} for 2000, 2005 and 2010, extrapolated with the GDP trend for China.

\begin{table}
  \caption{Differences between emissions for 2000 and 2018 (based on gap--filled data as used in EMEP models). Negative values mean that 2018 emissions were lower than 2000 emissions. Red/blue coloured data indicates that 2018 emissions were higher/lower than 2000 emissions.}
\centering
{\includegraphics*[viewport=40 100 500 745,clip,width=0.95\textwidth]{FIGS_CEIP/CEIP_table2_new.pdf}}
\label{tab:CEIP2}  
\end{table}

\begin{table}
  \caption*{Table~\ref{tab:CEIP2} continued. Differences between emissions for 2000 and 2018 (based on gap--filled data as used in EMEP models).}
\centering
{\includegraphics*[viewport=40 525 500 745,clip,width=0.95\textwidth]{FIGS_CEIP/CEIP_table2_contd_new.pdf}}
\label{tab:CEIP2contd}  
\end{table}

%\subsubsection{Trend analysis}
\subsection{Trend analysis}

Emission levels in the EMEP domain for 2018 of individual countries and areas are compared to 2000 emission levels for \nox, NMVOCs, \sox, \nhiii, CO and PMs (see Tables~\ref{tab:CEIP2}-\ref{tab:CEIP2} continued). For this comparison, gap-filled data as used in the EMEP models were used (see Section~\ref{sec:modeldata}). Overview tables with reported emission trends for individual countries have been published on the CEIP website\footnote{\url{https://www.ceip.at/webdab-emission-database}}. Detailed information on the sectoral level can also be accessed in WebDab\footnote{\url{https://www.ceip.at/webdab-emission-database} and/or \url{https://www.ceip.at/webdab-emission-database/emissions-as-used-in-emep-models}}.

The assessment of emission levels in individual countries and areas show an increase of emissions in 2018 compared to 2000 emission levels in several countries or areas.

In case of PM emissions, 25 countries/areas have higher PM$_{coarse}$ emissions in 2018 then in 2000, while PM$_{10}$ and PM$_{2.5}$ emissions increased in 18 and 17 countries/areas, respectively. In case of \nox and NMVOC there are 17  countries/areas, \sox 14, \nhiii 20 and CO 15
countries/areas with higher emissions in 2018 than in year 2000. Detailed explanatory information on emission trends should be provided in the informative inventory reports (IIRs). 


\section{Data sets for modelers 2020}
\label{sec:modeldata}

Data used by CEIP were reported by the Parties to the LRTAP Convention as sectoral emissions (NFR14) and national total emissions according to the UNECE guidelines for reporting emissions and projections data under the LRTAP Convention, Annex I (\cite{UNECE2014}).

The sector data were aggregated to 13 GNFR sectors. In several cases, no data were submitted by the countries, or the reporting is not complete or contains errors. Before these emission data can be used by modelers, missing or erroneous information have to be filled in. To gap-fill those missing data, CEIP typically applies different gap-filling methods. The gap-filling procedure in 2020 is fully documented in a technical report (Technical report CEIP 01/2020), which can be downloaded from the CEIP website\footnote{\url{https://www.ceip.at/ceip-reports}}.\\

Where data were missing or deemed implausible, CEIP experts selected the most appropriate gap-filling/replacement option from a set of predefined methods:

\begin{itemize}
    \item {\it Replacement} -- where no data are available or no plausible data are available, the most appropriate option is to replace the time series with the respective estimates from the interpolated GAINS ECLIPSE v6b dataset. This option can be applied to either all GNFR sectors or in certain cases to a single GNFR sector. 
\item {\it Extrapolation}  -- where a significant portion of the data appears plausible, it is appropriate to extrapolate the missing/implausible years at the beginning and/or end of the time series. In this case the expert must decide the trend with which to extrapolate the national total:
\begin{itemize}
\item Constant emissions are assumed; or where several years need to be extrapolated
\item Using the respective trends from GAINS estimates or even reported national totals (where national totals seem plausible) 
\item Unlike replacement, this option can only be applied to all sectors. In this case, the national total is in fact extrapolated, and subsequently split between sectors based on a sector split of nearest year deemed plausible.
\end{itemize}
\item {\it Ratio} -- where the PM$_{2.5}$ emissions are plausible, yet BC has not been reported or appears implausible, this option is considered the most appropriate. For this option, the reported sector PM$_{2.5}$ emissions are multiplied by BC fractions (GNFR sector-specific) derived from respective GAINS estimates of PM$_{2.5}$ and BC emissions. In most cases, the ratios from the respective country are taken; however, for small countries which are not resolved by GAINS, another country’s BC fractions can be selected. Again this option can be applied to either all sectors or a single sector if e.g. there is a mass balance issue for the said GNFR sector.
\item {\it Interpolation} -- this option can be applied on its own or in combination with the extrapolation methods. In this case missing/implausible emissions for years in between periods of plausible data are simply replaced by linear interpolation.
\end{itemize}

The Parties where data were (partly) replaced, corrected or gap-filled in 2020 are Albania, Armenia, Austria, Azerbaijan, Belarus, Bosnia and Herzegovina, Denmark, Georgia, Kazakhstan, Kyrgyzstan, Liechtenstein, Lithuania, Luxembourg,  Malta, Monaco, the Republic of Moldova, the Russian Federation, Serbia, Turkey and Ukraine.

For the countries, regions and sea regions of the EMEP domain for which there is no reporting obligation, emissions estimates from independent sources %(mostly the GAINS ECLIPSE v6b emission dataset) 
have been used to complete the EMEP dataset.

For the gap filled time series, significant changes in emissions from 2017 to 2018 were due to a recent update of the gap-filling procedure. As of this year, if reported emissions time series have been replaced, they have been replaced with estimates from the most recent GAINS ECLIPSE v6b dataset only. In previous years, emissions time series have been replaced with estimates from an older GAINS dataset, as well as from the EDGAR dataset.\\

After the gap-filling, sector emissions are spatially distributed over the EMEP grid. In 2020, gridded emissions of the pollutants \nox, NMVOCs, \sox, \nhiii, CO, PM$_{2.5}$, PM$_{10}$, PM$_{coarse}$ and BC were provided to modelers for the year 2018\footnote{\url{https://www.ceip.at/webdab-emission-database/emissions-as-used-in-emep-models}}.


%\subsection{Contribution of individual sectors to total EMEP emissions} 
\subsection{Contribution of individual GNFR sectors to total EMEP emissions}

Figure \ref{fig:CEIP7} shows the contribution of each GNFR sector to the
total emissions of individual air pollutants (\sox, \nox, CO, NMVOC,
\nhiii, PM$_{2.5}$, PM$_{10}$, PM$_{coarse}$ and BC). To provide a picture as complete as possible of the situation of the individual sectors to total EMEP emissions, data as used for the EMEP models (i.e. gap-filled data) were used for the calculations. % (see Section~\ref{sec:modeldata}).
Sea regions, North Africa and the remaining Asian areas were excluded for this analysis, as sectoral distributions are better reflected when only using country data. 

\begin{figure}[h]
\centering
{\includegraphics*[viewport=45 150 570 645,clip,scale=0.6]{FIGS_CEIP/Figure_7.pdf}}
\caption{GNFR sector contribution to national total emissions in 2018
  for the EMEP domain without sea regions, North Africa and remaining Asian areas.}
\label{fig:CEIP7}
\end{figure}

\begin{figure}[h]
\centering
{\includegraphics*[viewport=45 135 570 645,clip,scale=0.6]{FIGS_CEIP/Figure_8.pdf}}
\caption{GNFR sector contribution to national total emissions in 2018
  for the EMEP West and EMEP East areas. Asian areas are not included in the EMEP East region.
}
\label{fig:CEIP8}
\end{figure}

It is evident that the combustion of fossil fuels is responsible for a significant part of all emissions. For \nox emissions,the largest contributions come from transport (sector F, 33\%) and from large power plants (sector A, 17\%).

NMVOC sources are distributed more evenly among the different sectors,
such as 'E -- Emissions from solvents' (25\%), 'F -- Road transport' (20\%), 'D -- Fugitive Emissions' (13\%), 'B -- Industry combustion' (10\%), 'K -- Manure management' (11\%) and 'C -- Other stationary combustion' (9\%).


The main source of  \sox emissions are large point sources from combustion in energy and transformation industries (sector A, 46\% and sector B, 32\%).

Ammonia arises mainly from agricultural activities; about 93\% combined contribution from sectors K and L. Emissions of CO  originate primarily from 'F -- Road transport' (40\%) and 'C -- Other stationary combustion'  (26\%). 

The main sources of primary PM$_{10}$ and PM$_{2.5}$ emissions  are industry (28\% and 25\%) and other stationary combustion processes (25\% and 35\%). Due to the higher agricultural emissions of PM$_{10}$ versus PM$_{2.5}$, sectors K and L make a much larger relative contribution to PM$_{coarse}$ emissions (39\% combined).

Finally, the most important contributors to BC emissions are 'F -- Road transport' (20\%) and 'C -- Other stationary combustion' (31\%).

Figure \ref{fig:CEIP8} illustrates the
sector contributions to the sum of total emissions in the EMEP West
region and the EMEP East region. The split between the EMEP West and EMEP East regions is according to  \url{ https://www.ceip.at/countries} (sea regions, North Africa and the remaining Asian areas are excluded). The comparison of both graphs highlights some significant differences between West and East. 

For \nox in both the EMEP West and EMEP East regions the most important sector is 'F -- Road transport emissions' (38\% and 34\%, respectively), although it is worth noting the higher contribution from  'A -- Public electricity and heat production'  in the East region (25\%).

For NMVOC in the EMEP West region the most relevant sector is 'E -- Emissions from solvents' with a share of 37\%. In the EMEP East region the same sector has a considerable lower share  (12\%), whilst the sector 'F -- Road transport' is of high importance (33\%).

The main source of \sox are 'A -- Public electricity and heat production'  and and 'B -- Industry combustion'. These two sectors together contribute to 78\% and 85\%  of the \sox emissions within the EMEP West and EMEP East areas, respectively.

The main sources of \nhiii emissions for both EMEP West and EMEP East are the
agricultural sectors (K and L) with 93\% and 95\%, respectively.

CO emissions arise mainly from 'F -- Road transport emissions' (58\%) in EMEP East. In the EMEP West region the main sector is 'C -- Other stationary combustion' (42\%).

For PM$_{2.5}$ and PM$_{10}$ 'C -- Other stationary combustion' holds a 
significant share of the total emissions in the EMEP West area (55\% and 38\%), compared to the EMEP East area (18\% and 13\%). For the
EMEP East area sector 'B -- Industry combustion' is of higher importance. For PM$_{coarse}$ it is worth mentioning the higher contributions from agriculture in the EMEP East area (45\%). Finally, it is interesting to note the significant contribution to BC emissions in the EMEP East area from fugitive emissions (13\% in EMEP East versus 1\% in EMEP West). 



\subsection{Reporting of gridded data}

\begin{table} 
  \caption{Gridded emissions reported until 2017 and 2020.}
%\vspace{15pt}
\centering
{\includegraphics*[viewport=95 160 510 765,clip,width=0.95\textwidth]{FIGS_CEIP/CEIP_table3.pdf}}
%{\includegraphics*[width=0.95\textwidth]{FIGS_CEIP/CEIP_table3.png}} 
\label{tab:emis01degreported}  
\end{table}

2017 was the first year with reporting obligation of gridded emissions in the grid resolution of  0.1{\degrees}$\times$0.1{\degrees} lon\-gi\-tude/la\-ti\-tude.
Until June 2020, thirty-two of the 48 countries which are considered to be part of the EMEP area reported sectoral gridded emissions in this resolution.

The majority of gridded sectoral emissions in 0.1{\degrees}$\times$0.1{\degrees} lon\-gi\-tude/la\-ti\-tude resolution have been reported for the year 2015 (30 countries). For the year 2016, gridded sectoral emissions have been reported by three countries and for the year 2017 and 2018 by four countries.

Thirteen countries reported gridded emissions additionally for previous years (one country for the whole time series
from 1980 to 2018; one country for the whole time series from 1990 to 2018; five countries for the years 1990, 1995,
2000, 2005 and 2010; one country for the years 1990, 2000, 2005 and 2010; one country for the years 2005 and 2010; one
country for the year 2005; one country for the year 2010; and three countries for the year 2014). 

Reported gridded sectoral data in 0.1{\degrees}$\times$0.1{{\degrees}} lon\-gi\-tude/la\-ti\-tude resolution, which can be used for the preparation of gridded emissions for modelers, covers less than 20\% of the cells within the geographic EMEP area. For the remaining areas missing emissions are gap-filled and spatially distributed by expert estimates. Reported grid data can be downloaded from the CEIP website\footnote{\url{https://www.ceip.at/status-of-reporting-and-review-results}}. The gap-filled gridded emissions are also available there\footnote{\url{https://www.ceip.at/webdab-emission-database/emissions-as-used-in-emep-models}}. 

An overview of reported gridded data available in 2017 and until 2020 is provided in Table~\ref{tab:emis01degreported}, while an example map of the gap-filled and gridded \nox emissions in 2018 in 0.1{\degrees}$\times$0.1{\degrees} longitude-latitude resolution is shown in Figure~\ref{fig:CEIP9}.  

\begin{figure}[h]
\centering
{\includegraphics*[viewport=25 200 585 585,clip,scale=0.7]{FIGS_CEIP/Figure_9.pdf}}
\caption{Visualized gap-filled and gridded \nox emissions in 0.1{\degrees}$\times$0.1{\degrees} long-lat resolution.
}
\label{fig:CEIP9}
\end{figure}


Reported gridded data in 0.1{\degrees}$\times$0.1{\degrees} longitude-latitude resolution was used from Austria, Belgium, Bulgaria, Croatia, Czechia, Denmark, Finland, France, Georgia, Germany, Greece, Hungary, Ireland, Latvia, Luxembourg, Malta, Monaco, Netherlands, North Macedonia, Norway, Poland, Portugal, Romania, Slovakia, Slovenia, Spain, Sweden, Switzerland and United Kingdom.


\subsection{International shipping}

Under this category emissions from international shipping occurring in different European seas are accounted (European part of the North Atlantic, Baltic Sea, Black Sea, Mediterranean Sea and North Sea). International shipping emissions are not reported by Parties. Gridded emissions for the sea regions were
calculated using the CAMS global shipping emission dataset \citep{CAMSemis2019} for the years 2000 to 2018 (Figure~\ref{fig:CEIP10a}), developed by the Finnish Meteorological Institute (FMI), and provided via ECCAD\footnote{\url{https://eccad.aeris-data.fr}} the dataset {\it CAMS\_GLOB\_SHIP} \citep{ECCAD}.

Due to the selective implementation of the Sulphur Emission Control Areas (SECAs) on the North Sea and Baltic Sea only,
the emission trends differ between those seas and the other seas.


\begin{figure}[h]
\centering
{\includegraphics*[viewport=25 280 565 500,clip,scale=0.7]{FIGS_CEIP/Figure_10a.pdf}}
\caption{International shipping emission trends in the EMEP area, extracted from the CAMS global shipping emission dataset developed by FMI, and provided via ECCAD ({\it CAMS\_GLOB\_SHIP}) in April 2019 (for the years 2000 to 2017) and in November 2019 (for the year 2018). These are the emissions which have been used for the most recent trend calculations with the EMEP model.
}
\label{fig:CEIP10a}
\end{figure}

\begin{figure}[h]
\centering
{\includegraphics*[viewport=25 280 585 510,clip,scale=0.7]{FIGS_CEIP/Figure_10.pdf}}
\caption{International shipping emission trends in the EMEP area, extracted from the CAMS global shipping emission dataset developed by FMI, and provided via ECCAD (\it{CAMS\_GLOB\_SHIP}) in November 2019.}
\label{fig:CEIP10}
\end{figure}

The significant change in shipping emissions between 2017 and 2018 in the graph is partly because shipping data from the {\it CAMS\_GLOB\_SHIP} dataset have been adjusted since the preparation of gridded data for the years 2000 to 2017 in summer 2019. In 2020 only the year 2018 was added, but no re-gridding of the years 2000 to 2017 was carried out. Figure~\ref{fig:CEIP10} shows the adjusted CAMS global shipping emissions for the full time series from 2000 to 2018, as of November 2019.




\clearpage
\bibliographystyle{copernicus}         % change bibliography-name after each
\renewcommand\bibname{References}      % bibliographystyle command!
\addcontentsline{toc}{section}{References}
\bibliography{Refs,EMEP_Reports}
