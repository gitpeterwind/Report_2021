\chapter[Emissions 2019]{Emissions for 2019}
\label{ch:emis2019}


{\bf{Bradley Matthews, Katarina Mareckova, Sabine Schindlbacher, Bernhard Ullrich, \\
Robert Wankm\"uller and all CEIP/Umweltbundesamt Austria, Jeroen Kuenen (TNO)}}
\vspace{30pt}

In addition to meteorological variability, changes in the emissions
affect the inter-annual variability and trends of air pollution,
deposition and transboundary transport.  
The main changes in emissions in 2019 with respect to previous years
are documented in the following sections.


The EMEP Reporting guidelines \citep{UNECE2014} requests all Parties
to the LRTAP Convention to report annually emissions of air pollutants
(\sox\footnote{``Sulphur oxides (\sox)'' means all sulphur compounds,
  expressed as sulphur dioxide (\soii), including sulphur trioxide
  (\soiii), sulphuric acid (\sulacid), and reduced sulphur compounds,
  such as hydrogen sulphide (H${_2}$S), mercaptans and dimethyl
  sulphides, etc.}, \noii\footnote{``Nitrogen oxides (\nox)'' means
  nitric oxide and nitrogen dioxide, expressed as nitrogen dioxide
  (\noii).}, CO, NMVOCs\footnote{``Non-methane volatile organic
  compounds'' (NMVOCs) means all organic compounds of an anthropogenic
  nature, other than methane, that are capable of producing
  photochemical oxidants by reaction with nitrogen oxides in the
  presence of sunlight.}, \nhiii, HMs, POPs,
PM\footnote{``Particulate matter'' (PM) is an air pollutant
  consisting of a mixture of particles suspended in the air. These
  particles differ in their physical properties (such as size and
  shape) and chemical composition. Particulate matter refers to:\\  
(i) ``PM$_{2.5}$'', or particles with an aerodynamic diameter equal to or
  less than 2.5 micrometers ($\mu$m);\\ 
(ii) ``PM$_{10}$'', or particles with an aerodynamic diameter equal to or
  less than 10 $\mu$m.} and voluntary BC) and associated activity data. Projection data, gridded data and information on large point sources (LPS) 
have to be reported to the EMEP Centre on Emission Inventories and Projections (CEIP) every four years.

\section{Reporting of emission inventories in 2021}

Completeness and consistency of submitted data have improved significantly since EMEP started collecting information on emissions. Data of at least 47 Parties each year were submitted to CEIP since 2017 (see Figure~\ref{fig:CEIP1}). In 2021 (as of 1 June 2021), 48 Parties (94\%) submitted inventories\footnote{The original submissions from the Parties can be accessed via the CEIP homepage on \url{https://www.ceip.at/status-of-reporting-and-review-results/2021-submissions}.}, three Parties\footnote{Azerbaijan, Bosnia and Herzegovina and Kyrgyzstan} did not submit any data and 42 Parties reported black carbon (BC) emissions (see section~\ref{sec:bc}). As 2021 is a reporting year for large point sources (LPS) and gridded emissions, 32 Parties reported information on LPS, 26 Parties reported gridded data  \citep{CEIP2021}.

\begin{figure}[h]
\centering
{\includegraphics*[viewport=60 295 550 500,clip,scale=0.75]{FIGS_CEIP/Fig1.pdf}}
\caption{Parties reporting emission data to EMEP since 2002, as of 1 June 2021.}
\label{fig:CEIP1}
\end{figure}

The quality of the submitted data across countries differs quite significantly. By compiling the inventories, countries have to use the newest available version of the {\it EMEP/EEA air pollutant emission inventory guidebook}, which is the version of 2019 \citep{EmisInvGuide2019}. However, many countries still use the 2016 Guidebook \citep{EmisInvGuide2016} or older versions (e.g. \cite{EmisInvGuide2013}). As analysed in a technical report \citep{CEIP2021b}, uncertainty of the reported data (national totals, sectoral data) is relatively high, e.g. the reported uncertainty estimates ranged from 6.9\% to 56\% for \nox emissions reported in 2020. Further, the completeness of reported data has not turned out satisfactory for all pollutants and sectors either.

More detailed information on recalculations, completeness and key
categories, plus additional review findings can be found in the annual
CEIP technical country  reports{\footnote{\url{https://www.ceip.at/review-of-emission-inventories/technical-review-reports}}}.

Indeed, the issue of recalculations is highly relevant to users of EMEP emissions datasets. The aforementioned CEIP report on uncertainties in reported emissions highlighted how time series of reported emissions can vary significantly over subsequent rounds of submissions due to inter alia revisions in activity data, updates of methods and emissions factors and/or inclusion of previously overlooked sources of emissions \citep{CEIP2021b}.
%As an indication of potential revisions in reported emissions,  Appendix 1 lists the EMEP country emissions where emissions for the year 2019 (reported in 2021) deviate from respective emissions for the year 2018 (reported in 2020) by more than 10\%. This Appendix furthermore indicates whether these reported emissions times have been included in the EMEP datasets for modellers, or whether they have been partially or completely replaced during the quality assessment and gap-filling procedure.}
The following subchapters summarise the inventory submissions in terms of three topics that are currently of high interest within the Convention:

\begin{itemize}
\item Reporting of black carbon emissions (\ref{sec:bc})
\item Inclusion of the condensable component in particulate matter emissions (\ref{sec:EmisSVOC}) 
\item Comparison of reported Party emissions to respective reduction targets set out in the Gothenburg Protocol  (\ref{sec:GP})
\end{itemize}
  
%\subsection{Black Carbon (BC) emissions}
\section{Black Carbon (BC) emissions}  
\label{sec:bc}

Over the last decade, black carbon (BC) has emerged as an important air pollutant in terms of both climate change and air quality.  

\begin{figure}[h]
\centering
%{\includegraphics*[viewport=30 260 585 525,clip,scale=0.7]{FIGS_CEIP/Fig2.pdf}}
{\includegraphics*[viewport=1 1 950 465,clip,scale=0.4]{FIGS_CEIP/Fig2.pdf}}
\caption{Black carbon emissions of the year 2019 as reported by CLRTAP Parties.}
\label{fig:CEIP2}
\end{figure}


The emerging significance of BC is mirrored in developments in the international policy arena with respect to emissions reporting.
Since the Executive Body Decision 2013/04, Parties to the LRTAP Convention have been formally encouraged to
submit inventory estimates of their national BC emissions, and in 2015 the reporting templates were updated to include
BC data emissions.

%% In addition to reporting under CLRTAP, EU member states are also encouraged to submit BC emissions
%% estimates as part of their emissions reporting under the National Emissions Ceilings (NEC) Directive (2016/2284/EU).


While BC is not a mandatory pollutant to be reported under the Convention, CEIP continues to monitor and review the level of BC reporting by the Convention's Parties. A brief overview of BC emissions estimates submitted by Parties in 2021 is given below.  %Figure~\ref{fig:CEIP2} 

Since enabling the reporting of BC, a total of 45 CLRTAP Parties have reported BC emissions estimates\footnote{ As of 1 June 2021  Austria, Bosnia and Herzegovina, Liechtenstein, Luxembourg, Russia, and Turkey have yet to report estimates of national BC emissions.}. In this round of reporting, 30 CLRTAP Parties submitted a complete time series of national total BC emissions (1990-2019), while 38 CLRTAP Parties submitted a complete time series from 2000 onwards. Furthermore, 42 EMEP Parties have provided national total BC emissions estimates for the year 2019 (see Figure~\ref{fig:CEIP2}).

For more detailed information on BC consult the annual CEIP technical inventory review report \citep{CEIP2020}.


\section{Inclusion of the condensable component in reported PM emissions}
\label{sec:EmisSVOC}

The condensable component of particulate matter is a class of organic compounds of low volatility that may exist in equilibrium between the gas and particle phase. It is probably the biggest single source of uncertainty in PM emissions. For more background see \citet{CONDws2020}.
%% The condensable component of particulate matter is released as a gas but forms particles when it is diluted and cools
%% down.
Currently the condensable component is not included or excluded consistently in PM emissions reported by Parties
of the LRTAP Convention. Also in the EMEP/EEA Guidebook \citep{EmisInvGuide2019} the condensable fraction is not consistently included or
excluded in the emission factors. Various EMEP centres and task forces and other stakeholders jointly discuss the topic and work on progress in this area. An important activity in 2020 was the workshop organised by MSC-W that resulted in a workshop report \citep{CONDws2020}. However, at the moment PM emissions reported by Parties to the LRTAP Convention are not directly
comparable, which has implications on the modeling of overall exposure to PM.

Small scale combustion sources make a notable contribution to total PM emissions. For all Parties that reported PM$_{2.5}$ emissions for ''1A4bi Residential: Stationar'' for the year 2019 the average contribution to the national total PM$_{2.5}$ emissions from this source category was 46\%. Small-scale combustion is one of the sources where the inclusion of the condensable component has the largest impact on the emission factor. For example, for conventional woodstoves, one of the most important categories in Europe, the emission factors excluding and including the condensable fractions may differ by up to a factor of five \citep{DeniervanderGon2015}.
To improve the quality of the input data for air quality models, and following a decision of \citet{EMEPBureaux2020}, the group of experts that met at the workshop organised by MSC-W agreed on the following approach (for more details see \citet{CONDws2020}): 

\begin{itemize}
\item In year one (2020) the so-called REF2 emission data provided by TNO, which include condensable organics,  is used in an initial estimate for residential combustion emissions. The REF2 data and their usage in the EMEP modeling work in 2020 are described  in  \citet{R2020:CAMSREF2} and \citet{R2020:SVOC}. 
\item  In subsequent years these top-down estimates should be increasingly replaced by national estimates once procedures for quantifying condensables in a more harmonized way are agreed on and implemented.
\end{itemize}

CEIP in co-operation with TNO prepared a list of Parties where it could be assumed with a good degree of certainty that the condensable component is mostly included in PM emissions for GNFR sector C. The analysis focused on small scale combustion. This analysis was based on (a) the calculation of Implied Emission Factors (emission divided by the reported activity data), (b) the fuel mix of the Party, (c) the information provided in the Informative Inventory Report and (d) in a few cases direct information from Parties (received via e-mail).

The analysis resulted in a list of Parties where the conclusion was that the PM emission data reported by the Party should be used as the condensable component seemed to be included. For other Parties the TNO REF2 data were used. In a few cases data that had been gap-filled by CEIP was used as for the respective Party no REF2 estimate was available (see table~\ref{tab:CEIP1}).



\begin{table}
  \caption{Data source for PM emission in GNFR C used in EMEP modelling in 2021.}
\centering
{\includegraphics*[viewport=120 275 550 760,clip,width=0.8\textwidth]{FIGS_CEIP/Table1.pdf}}
\label{tab:CEIP1}  
\end{table}

Parties were asked to include a table with information on the inclusion of the condensable component in PM$_{10}$ and PM$_{2.5}$ emission factors for reporting under the CLRTAP Convention in 2021. This table has been added to the revised
recommended structure for IIRs\footnote{\url{https://www.ceip.at/reporting-instructions } }. Twenty-three Parties provided information on the inclusion of the condensable component in PM$_{10}$ and PM$_{2.5}$ emission factors\footnote{Status as of 15 May 2021}. This reporting is a first step towards a better understanding of reported PM data. The information that Parties provided on whether the condensable component is included in PM emissions was quite heterogeneous. The status of inclusion or exclusion is best known for emissions from the energy sector and road transport, for which many Parties submitted information. 

\subsection{REF2 emissions and improvements compared to last year}


The REF2 emission inventory provides a bottom-up database of PM emissions (both PM$_{10}$ and PM$_{2.5}$) from small combustion activities (GNFR category C), taking into account activity data and consistent emission factors that include condensables, for both wood and solid fuel combustion. It was originally developed for the year 2010 \citep{DeniervanderGon2015}. Residential emissions vary from year to year, because of technological developments in the sector (replacement of stoves and boilers) but also because of the heating demand due to fluctuating temperatures. To take this into account, an alternative REF2 was developed for 2015 which was also used as an input to EMEP modelling in 2020. These REF2 emission data and their usage in the EMEP modelling work in 2020 are described in  \citet{R2020:CAMSREF2} and \citet{R2020:SVOC}. This version was developed by scaling the original REF2 for 2010 to the year 2015 by using the trend in the official reported data for PM$_{2.5}$ from GNFR~C for that particular year. The idea behind this approach was that by scaling in this way both the technological changes within each country as well as the annual fluctuation in heating demand are included. It is desirable but difficult to separate these effects because the detailed underlying country data are not available. Countries report emissions only by source sector and for the sum of all fuels; activity data or emissions by appliance type are not available.

This scaled REF2 for the year 2015 has been used both in EMEP and other modelling activities, and presented in various international expert group meetings such as CAMS, EMEP and UNECE Task Forces (in particular TFEIP and TFMM). This triggered several discussions with experts (including some specific countries), which resulted in the provision of additional information of the national circumstances on the types of stoves installed, etc. (see also \cite{CONDws2020}). 

With this information, as well as additional information made available by IIASA (Z. Klimont), for 5 specific countries (Austria, Germany, Finland, France, the Netherlands) the REF2 emission estimates were revised. The main reason to change the emission estimates is the new information that became available on the appliance types, especially regarding the split between different types of heating stoves (traditional vs. modern). For modern stoves, the PM emission factors are significantly lower compared to traditional stoves, and for the condensable component of PM this effect is even larger.
The 5 selected countries were chosen because of feedback received or discussions with national experts which suggested that REF2 emissions may be overestimating. It is planned to also revisit the REF2 emission estimates for the other countries, but this is ongoing work.

\begin{figure}[h]
\centering
{\includegraphics*[viewport=1 1 740 400,clip,scale=0.45]{FIGS_CEIP/Fig3.pdf}}
\caption{Results for REF2 for 2010 (original), 2015 (after scaling, used in EMEP 2020) and 2015 (after scaling + adjustment for 5 countries, used in EMEP 2021).}
\label{fig:CEIP3}
\end{figure}


Figure~\ref{fig:CEIP3} shows the different REF2 emission estimates. The year 2010 is the original REF2 estimate, while 2015\_scaled represents the scaled REF2 (REF2.1) to the year 2015 using the official reported data (used in the 2020 EMEP modelling). The estimate 2015\_scaled\_adjusted is the updated version described here, where for 5 selected countries the REF2 data have been revised based on new information.
From the figure, it can be seen that the 2015 emission is lower than 2010 for almost all countries. This is partly the result of technological advancements in countries (replacement of older stoves with new ones), but also largely related to the fact that 2010 was a relatively cold year in Europe, hence with higher emissions from the residential sector compared to other years. The figure also shows that for the five countries that have been adjusted in the latest update, in each case the REF2.1 emission including condensables is corrected downward. This adjusted REF2.1 is, however, still considerably higher than reported emission for each of these countries.

\section{Gothenburg Protocol targets}
\label{sec:GP}

The 1999 Gothenburg Protocol (GP) lists emission reduction commitments of  \nox,
\sox, NMVOCs and \nhiii for most of the Parties to the LRTAP Convention for the year 2010 (\cite{UNECE1999}). These commitments should not be exceeded in 2010 and in subsequent years either.

In 2012, the Executive Body of the LRTAP Convention decided that adjustments to inventories may be applied in some circumstances (\cite{UNECE2012}). From 2014 to 2021, adjustment applications of ten countries (Belgium, Czechia, Denmark, Finland, France, Hungary, Germany, the Netherlands, Luxembourg, Spain and the United Kingdom) have been accepted by expert review team and therefore these approved adjustments have to be subtracted for the respective countries when compared with the targets. In April 2021, Czechia and France  submitted  new adjustment applications; the \nhiii adjustment application  of Czechia was rejected by the review team and application from France has been accepted.     

Further, the reporting guidelines (\cite{UNECE2014}) specify that some Parties within the EMEP region (i.e. Austria, Belgium, Ireland, Lithuania, Luxembourg, the Netherlands, Switzerland, the United Kingdom of Great Britain and Northern Ireland) {\it may choose to use the national emission total calculated on the basis of fuels used} in the geographic area of the Party as a basis for compliance with their respective emission ceilings.

However, when considering only reported data, approved adjustments and fuel used data of the respective countries, Figure~\ref{fig:CEIP4} indicates that in the year 2019 North Macedonia could not reduce their \sox emissions below their respective Gothenburg Protocol requirements, and that Croatia and Spain are above their 1999 Gothenburg Protocol ceilings concerning \nhiii. For \nox and NMVOC all countries were below their individual ceilings in year 2019.


\begin{figure}[h]
\centering
{\includegraphics*[viewport=1 1 753 323,clip,scale=0.55]{FIGS_CEIP/Fig4.pdf}}
\caption{Distance to Gothenburg Protocol targets in 2019 (based on reported data in 2021).
  Only Parties that ratified the Gothenburg Protocol are included. 
  * Emission data based on fuels used for road transport.
Approved adjustments are considered for  Denmark (NMVOCs, \nhiii), Finland (\nhiii),  Germany (\nox, NMVOCs, \nhiii), Luxembourg (\nox, NMVOCs), the Netherlands (\nhiii, NMVOCs).}
\label{fig:CEIP4}
\end{figure}





\section{Datasets for modellers 2021}
\label{sec:modeldata}

Under the Convention, CEIP is responsible for synthesizing the reported emissions data of the EMEP countries into complete gridded emissions datasets for the EMEP domain (covering the geographic area between 30\degrees N-82\degrees N latitude and 30\degrees W-90\degrees E longitude. These data are mainly used for modelling of air pollutant concentrations and depositions.

To compile these datasets each year, CEIP synthesizes and evaluates the most recent national sectoral emissions estimates and national gridded emissions data reported by the EMEP countries. CEIP strives to include, to the largest possible extent, the reported emissions data it receives from EMEP countries. However, due to cases of non-reporting or identified quality issues in the reported data, emissions need to be gap-filled or replaced. Furthermore, it should be noted how gridded and sectoral emissions totals are combined in compiling these datasets. National gridded emissions data, even if reported annually, are not directly utilized but are rather used to map out relative emissions, with which national sector emission totals are spatially distributed. Of course if for a given year both national sector emissions totals and gridded estimates reported by a given country pass through the CEIP QA/QC checks, the generated gridded emissions will be identical to the gridded emissions reported by the country.
The following subchapters describe important aspects of the 2021 EMEP datasets, summarising:

\begin{itemize}
    \item The status of reporting of national gridded emissions data and the extent to which these are used to distribute emissions spatially (Section~\ref{sec:griddedemis})
    \item The extent to which sectoral emissions were gap-filled or replaced (Section~\ref{sec:gapfilling})
    \item  The sectoral contributions (Section~\ref{sec:GNFRsec}) and temporal trends (Section~\ref{sec:emistrends}) in the emissions of carbon monoxide, nitrogen oxides, sulphur oxides, ammonia, non-methane volatile organic carbons, and particulate matter including black carbon. Trends in shipping emissions are discussed separately (Section~\ref{sec:shiptrends}).     
\end{itemize}


\subsection{Reporting of gridded data}
\label{sec:griddedemis}

\begin{table} 
  \caption{Gridded emissions in 0.1{\degrees}$\times$0.1{\degrees} longitude/latitude resolution reported until 2017, 2020 and 2021.}
%\vspace{15pt}
\centering
{\includegraphics*[viewport=115 65 550 740 ,clip,width=0.95\textwidth]{FIGS_CEIP/Table2.pdf}}
\label{tab:emis01degreported}  
\end{table}


After the first round of submissions in 2017, 2021 was the second year for which EMEP countries were obliged to report gridded emissions in the grid resolution of 0.1{\degrees}$\times$0.1{\degrees} lon\-gi\-tude/la\-ti\-tude. As of June 2021, 34 of the 48 countries which are considered to be part of the EMEP area reported sectoral gridded emissions in this resolution.


The majority of gridded sectoral emissions in 0.1{\degrees}$\times$0.1{\degrees} lon\-gi\-tude/la\-ti\-tude resolution have been reported for the year 2015 (32 countries). For 2019 gridded sectoral emissions have been reported by 29 countries, for 2016, 2017 by five countries and for 2018 by four countries. Comparing to reporting in 2017, reported gridded data are available for 11 more countries in 2021.

Fifteen countries reported gridded emissions additionally for previous years (one country for the whole time series from 1980 to 2019; one country for the whole time series from 1990 to 2019; seven countries for the years 1990, 1995, 2000, 2005 and 2010; one country for the years 1990, 2000, 2005 and 2010; one country for the years 2000, 2005 and 2010; one country for the year 2005; one country for the year 2010; and two countries for the year 2014).


Reported gridded sectoral data in 0.1{\degrees}$\times$0.1{{\degrees}} lon\-gi\-tude/la\-ti\-tude resolution, which can be used for the preparation of gridded emissions for modelers, covers less than 25\% of the cells within the geographic EMEP area. For the remaining areas (or for EMEP countries that have no reported gridded data) missing emissions are gap-filled and spatially distributed by expert estimates. Reported grid data can be downloaded from the CEIP website\footnote{\url{https://www.ceip.at/status-of-reporting-and-review-results}}. The gap-filled gridded emissions are also available there\footnote{\url{https://www.ceip.at/webdab-emission-database/emissions-as-used-in-emep-models}}. 

An overview of gridded data in 0.1{\degrees}$\times$0.1{{\degrees}} lon\-gi\-tude/la\-ti\-tude resolution reported in 2017, 2020 and 2021 is provided in Table~\ref{tab:emis01degreported}.

%while an example map of the gap-filled and gridded \nox emissions in 2018 in 0.1{\degrees}$\times$0.1{\degrees} longitude-latitude resolution is shown in Figure~\ref{fig:CEIP5}.


%% \begin{figure}[h]
%% \centering
%% {\includegraphics*[viewport=25 200 585 585,clip,scale=0.7]{FIGS_CEIP/Fig5.pdf}}
%% \caption{Visualized gap-filled and gridded \nox emissions in 0.1{\degrees}$\times$0.1{\degrees} long-lat resolution.
%% }
%% \label{fig:CEIP5}
%% \end{figure}


For compiling the 2021 EMEP emisisons dataset, reported gridded data in 0.1{\degrees}$\times$0.1{\degrees} longitude-latitude resolution was used from Austria, Belgium, Bulgaria, Croatia, Cyprus, Czechia, Denmark, Estonia, Finland, France, Georgia, Germany, Greece, Hungary, Ireland, Latvia, Luxembourg, Malta, Monaco, Netherlands, North Macedonia, Norway, Poland, Portugal, Romania, Slovakia, Slovenia, Spain, Sweden, Switzerland and United Kingdom.


\subsection{Gap-filling of reported data in 2021}
\label{sec:gapfilling}

As described above, sectoral emissions reported by the EMEP countries are used, to the largest extent possible, to compile the gridded EMEP datasets. Each year the reported source-sector level data of the (NFR level) are aggregated into the 13 GNFR sectors and are then evaluated to identify countries for which emissions have not been reported or appear to exhibit implausible emission levels and/or trends. Based on this assessment, a procedure is then implemented to gap-fill missing emissions data and to replace data that have been identified as implausible. The sectoral emissions are then distributed spatially using, where available (and appropriate), the reported national gridded emissions as relative spatial proxies, or other independent datasets of spatial proxies.

Given the end of May deadline for compiling EMEP datasets, a cut-off date for incorporating reported emissions has to be set to allow necessary time for evaluating the reported emissions and implementing the gap-filling procedure. This year, the sectoral emissions data reported by 14 April 2021 were evaluated and considered for use in the compilation of the 2021 EMEP datasets of gridded emissions.

The Parties, where data were (partly) replaced, corrected or gap-filled in 2021 are Albania, Armenia, Austria, Azerbaijan, Belarus, Bosnia and Herzegovina, Cyprus, Denmark, Georgia, Kazakhstan, Kyrgyzstan, Liechtenstein, Lithuania, Luxembourg,  Montenegro, the Republic of Moldova, Montenegro, the Russian Federation, Serbia, Turkey and the Ukraine. The results of the quality control and gap-filling procedures are described in detail in CEIP gap-filling report \citep{CEIP2021c}. % and in Appendix 2, a table is provided that documents for which EMEP countries and specific pollutants gap-filling/replacement was required and which methods were implemented. The reader is nonetheless reminded that this Table does not indicate if a subsequent adjustment of reported PM emissions for GNFR Sector C was undertaken to account for the potentially missing condensable component. This information is provided in Table 1.

Finally, it should be noted that the gap-filling and replacement procedure has been updated since 2020. The gap-filling/replacement of EMEP country emissions remains based on the independent estimates from the ECLIPSE v6b\footnote{\url{https://iiasa.ac.at/web/home/research/researchPrograms/air/ECLIPSEv6.html}} dataset that has been compiled by IIASA using their GAINS model \citep{Amann_et_al:2011}. However, the emissions for areas North Africa, remaining Asian areas, the Aral Lake and the part of Russia within the EMEP domain for which Russia does not report emissions (referred to as 'Russian Federation Asian part' further in this chapter), are now based on the updated EDGAR v5.0\footnote{\url{https://edgar.jrc.ec.europa.eu/dataset_ap50}}  dataset \citep{EDGARv50} that was generated by the European Commission's Joint Research Centre (JRC). Previously, the emissions for these areas were based on a previous version (EDGAR v4.3.2) of the dataset \citep{EDGARv432}.


\subsection{Contribution of GNFR sectors to total EMEP emissions}
\label{sec:GNFRsec}

Figure \ref{fig:CEIP6} shows the contribution of each GNFR sector to the
total emissions of individual air pollutants (\sox, \nox, CO, NMVOC,
\nhiii, PM$_{2.5}$, PM$_{10}$, PM$_{coarse}$ and BC).
%To provide a picture as complete as possible of the situation of the individual sectors to total EMEP emissions, data as used for the EMEP models (i.e. gap-filled data) were used for the calculations. 
To clarify, the reader is reminded that these analyses are based on the emission data in the EMEP datasets for modellers i.e. data based largely on reported emissions, but also compiled with independent emissions estimates for countries and regions where data are not reported or the reported data have been omitted due to quality issues. The sea regions were excluded for this sectoral analysis.


\begin{figure}[h]
\centering
{\includegraphics*[viewport=1 1 565 565,clip,width=0.65\linewidth]{FIGS_CEIP/Fig6.pdf}}
\caption{GNFR sector contribution to national total emissions in 2019
  for the EMEP domain apart from the  sea regions.}
\label{fig:CEIP6}
\end{figure}

\begin{figure}[h]
\centering
{\includegraphics*[viewport=1 1 565 565,clip,width=0.65\linewidth]{FIGS_CEIP/Fig7.pdf}}
\caption{GNFR sector contribution to national total emissions in 2019
  for the EMEP West and EMEP East areas. Asian areas, North Africa and the sea regions are not included.}
\label{fig:CEIP7}
\end{figure}

It is evident that the combustion of fossil fuels is responsible for a significant part of all emissions. For \nox emissions, the largest contributions come from transport (sector F, 40\%) and from large power plants (sector A, 21\%).

NMVOC sources are distributed more evenly among the different sectors,
such as 'E - Emissions from solvents' (21\%), 'F - Road transport' (30\%), 'D - Fugitive Emissions' (14\%), 'B - Industry combustion' (7\%), 'K - Manure management' (9\%) and 'C - Other stationary combustion' (12\%).


The main source of  \sox emissions are large point sources from combustion in energy and transformation industries (sector A, 56\% and sector B, 23\%).

Ammonia arises mainly from agricultural activities; about 92\% combined contribution from sectors K and L. Emissions of CO  originate primarily from 'F - Road transport' (51\%) and 'C - Other stationary combustion'  (26\%). 

The main sources of primary PM$_{10}$ and PM$_{2.5}$ emissions  are industry (23\% and 20\%) and other stationary combustion processes (39\% and 55\%). Due to the higher agricultural emissions of PM$_{10}$ versus PM$_{2.5}$, sectors K and L make a much larger relative contribution to PM$_{coarse}$ emissions (29\% combined) together with significant contributions from 'B - industry combustion' (26\%) and 'C - Other stationary combustion' (29\%). 

Finally, the most important contributors to BC emissions are 'F - Road transport' (18\%) and 'C - Other stationary combustion' (57\%).

Figure \ref{fig:CEIP7} illustrates the
sector contributions to the sum of total emissions in the EMEP West
region and the EMEP East region. The split between the EMEP West and EMEP East regions is according to  \url{https://www.ceip.at/countries} (sea regions, North Africa and the remaining Asian areas are excluded, with the Aral Lake area assigned to EMEP East). The comparison of both graphs highlights some significant differences between West and East. 

For \nox in both the EMEP West and EMEP East regions the most important sector is 'F - Road transport emissions' (38\% and 34\%, respectively), although it is worth noting the higher contribution from  'A - Public electricity and heat production'  in the East region (23\%).

For NMVOC in the EMEP West region the most relevant sector is 'E - Emissions from solvents' with a share of 35\%. In the EMEP East region the same sector has a considerable lower share  (14\%), whilst the sector 'F - Road transport' is of high importance (32\%).

The main source of \sox are 'A - Public electricity and heat production'  and and 'B - Industry combustion'. These two sectors together contribute to 78\% and 86\%  of the \sox emissions within the EMEP West and EMEP East areas, respectively.

The main sources of \nhiii emissions for both EMEP West and EMEP East are the
agricultural sectors (K and L) with 93\% and 94\%, respectively.

CO emissions arise mainly from 'F - Road transport emissions' (60\%) in EMEP East. In the EMEP West region the main sector is 'C - Other stationary combustion' (40\%).

For PM$_{2.5}$ and PM$_{10}$ 'C - Other stationary combustion' holds a 
significant share of the total emissions in the EMEP West area (54\% and 37\%), compared to the EMEP East area (18\% and 14\%). For the
EMEP East area sector 'B - Industry combustion' is of higher importance. For PM$_{coarse}$ it is worth mentioning the higher contributions from agriculture in the EMEP East area (44\%). Finally, it is interesting to note the significant contribution to BC emissions in the EMEP East area from fugitive emissions (13\% in EMEP East versus 1\% in EMEP West). 


\subsection{Trends in emissions in the geographic EMEP domain}
\label{sec:emistrends}

\begin{figure}[b]
\centering
{\includegraphics*[viewport=1 1 565 370,clip,width=0.75\linewidth]{FIGS_CEIP/Fig8.pdf}}
\caption{Emission trends 2000--2019 in the geographic EMEP area (emissions from international shipping in the sea regions are excluded)}
\label{fig:CEIP8}
\end{figure}

\begin{figure}
\centering
     \subfigure[EMEP West]{\includegraphics*[viewport=1 1 567 375,clip,width=0.6\linewidth]{FIGS_CEIP/Fig9i.pdf}}\\
     \subfigure[EMEP East]{\includegraphics*[viewport=1 1 567 375,clip,width=0.6\linewidth]{FIGS_CEIP/Fig9ii.pdf}}\\
     \subfigure[Other Land Areas]{\includegraphics*[viewport=1 1 567 375,clip,width=0.6\linewidth]{FIGS_CEIP/Fig9iii.pdf}}     
\caption{Emission trends 2000-2019 in the geographic EMEP domain (emissions from international shipping in the sea regions are excluded) divided into three areas: 'EMEP West' (top), 'EMEP East' (middle) and 'Other Land Areas' (bottom), that include the emissions from North Africa and the remaining Asian areas. }
\label{fig:CEIP9}
\end{figure}

The following trend analyses are based on the emissions data in the EMEP datasets for modellers, i.e. data based largely on reported emissions, but also compiled with independent emissions estimates for countries and regions where data are not reported or the reported data have been omitted due to quality issues.

Excluding shipping emissions in the sea regions (these are summarised in the following subchapter), the trend analyses of total emissions from the non-sea areas in the EMEP domain\footnote{The EMEP domain covers the geographic area between 30{\degrees} N-82{\degrees} N latitude and 30{\degrees} 
  W-90{\degrees} E longitude.} show that emissions of seven of the nine pollutants have decreased overall since 2000 Figure~\ref{fig:CEIP8}). Only the 2019 PM$_{coarse}$  and \nhiii emissions have increased (by 6 and 12\%, respectively) since 2000. The 2019 emissions of \sox are 69\% of the respective 2000 emissions. While the 2019 emissions of CO, NMVOC, \nox, PM$_{2.5}$, PM$_{10}$ and BC are all lower than respective emissions in 2000 (1-10\% lower), it is interesting to note that emissions of these pollutants have been increasing since ca. 2014.

Despite these overall trends, assessment shows that regional emission developments seem to follow strongly different patterns (Figure~\ref{fig:CEIP9}). While emissions of all of the pollutants in the EMEP West countries are clearly decreasing, emissions of all pollutants in the EMEP East countries of the EMEP domain have been somewhat stable (albeit gradually decreasing for most pollutants) over the 2000-2019 period. For the Other Land Areas (North Africa and the remaining Asian areas, emissions are clearly on the rise.

Of course it is not just the emission trends that separate the three land regions. Whereas the emission trends of the EMEP West countries are based to a very large extent on the official national inventories reported to CEIP, the countries within the Other Land Areas within the EMEP domain (North Africa, remaining Asian areas) are not Parties to the Convention and thus are not obliged to report their emissions. For these regions, emissions are based completely on the independent gridded emission estimates of the EDGAR v5.0 dataset \citep{EDGARv50} that was generated by the European Commission's Joint Research Centre (JRC). For the EMEP East region, again not all countries are Parties to the Convention (Turkmenistan, Tajikistan and Uzbekistan) and reported Russian emissions do not cover the region of Russia within the EMEP domain that is ca. east of the Urals. Emissions for the area of the Aral Lake are also not reported by any Convention country. Note that the emissions for the eastern part of Russia and the Aral Lake have also been gap-filled using the independent gridded emission estimates of the EDGAR v5.0 dataset. Finally, it should be noted that many of the emissions time series for the EMEP East countries that are Parties have been partially or fully replaced with independent estimates from the ECLIPSE v6b\footnote{\url{https://iiasa.ac.at/web/home/research/researchPrograms/air/ECLIPSEv6.html}} dataset that has been compiled by IIASA using their GAINS model  \citep{Amann_et_al:2011}.
%An overview of the Parties and pollutants for which replacement and gap-filling was required is given in Appendix 2.


Non-sea emission levels in the geographic EMEP domain for 2019 of the individual countries and areas are compared to 2000 emission levels for each pollutant (see Tables~\ref{tab:CEIP3}-\ref{tab:CEIP3} continued). Again, the reader is reminded that the following trend analyses are based on the emissions data in the EMEP datasets for modellers i.e. data based largely on reported emissions, but also compiled with independent emissions estimates for countries and regions where data are not reported or the reported data have been omitted due to quality issues. Overview tables with reported emission trends for individual countries have been published on the CEIP website\footnote{\url{https://www.ceip.at/webdab-emission-database/reported-emissiondata}}. Detailed information on the sectoral level can also be accessed in WebDab\footnote{\url{https://www.ceip.at/webdab-emission-database/emissions-as-used-in-emep-models} and/or \url{https://www.ceip.at/webdab-emission-database/reported-emissiondata}}.

The assessment of emission levels in individual countries and areas show an increase of emissions in 2019 compared to 2000 emission levels in several countries or areas.


In case of PM emissions, 24 countries/areas have higher PM$_{coarse}$ emissions in 2019 than in 2000, while PM$_{10}$ and PM$_{2.5}$ emissions increased in 17 and 16 countries/areas, respectively. In case of \nox and \sox there are 14  countries/areas, NMVOC 13, \nhiii 18 and CO 12
countries/areas with higher emissions in 2019 than in year 2000. Detailed explanatory information on emission trends for the reporting countries should be provided in the respective informative inventory reports (IIRs). Tables~\ref{tab:CEIP3}-\ref{tab:CEIP3} continued indicates whether the emissions were based completely  (R) or partially (r) on reported data. 

\begin{table}
  \caption{Differences between emissions for 2000 and 2019 (based on gap--filled data as used in EMEP models). Negative values mean that 2019 emissions were lower than 2000 emissions. Red/blue coloured data indicates that 2019 emissions were higher/lower than 2000 emissions. Furthermore, the symbol in parentheses indicate whether the emissions times series are completely based on reported data (R), are partially based on reported data (r), or have been completely replaced/gap-filled (-).}
\centering
{\includegraphics*[viewport=45 100 535 735,clip,width=0.99\textwidth]{FIGS_CEIP/Table3_page1.pdf}}
\label{tab:CEIP3}  
\end{table}

\begin{table}
  \caption*{Table~\ref{tab:CEIP3} continued. Differences between emissions for 2000 and 2019 (based on gap--filled data as used in EMEP models).}
\centering
{\includegraphics*[viewport=45 520 535 735,clip,width=0.99\textwidth]{FIGS_CEIP/Table3_page2.pdf}}
\label{tab:CEIP3contd}  
\end{table}

%% CEIP included again a summary of increase/decrease by pollutant here, which I (Agnes) did not include because it has no additional value
%% whatsoever. All info can be found in the table and the text before. I thought we agreed not to write this part anymore. 

\subsection{Trends in emissions from international shipping}
\label{sec:shiptrends}

International shipping emissions are not reported by Parties. Gridded emissions for the sea regions (European part of the North Atlantic, Baltic Sea, Black Sea, Mediterranean Sea and North Sea) were calculated using the CAMS global shipping dataset \citep{CAMSemis2019} for the years 2000 to 2019 (Figure~\ref{fig:CEIP10}), developed by the Finnish Meteorological Institute (FMI), and provided via ECCAD\footnote{\url{https://eccad.aeris-data.fr}} the dataset {\it CAMS\_GLOB\_SHIP} \citep{ECCAD}.

\begin{figure}[h]
\centering
{\includegraphics*[viewport=1 1 553 230,clip,width=0.75\linewidth]{FIGS_CEIP/Fig10.pdf}}
\caption{International shipping emission trends in the EMEP area, extracted from the CAMS global shipping emission dataset developed by FMI, and provided via ECCAD ({\it CAMS\_GLOB\_SHIP}) in April 2019 (for the years 2000 to 2017) and in November 2019 (for the year 2018). These are the emissions which have been used for the most recent trend calculations with the EMEP model.
}
\label{fig:CEIP10}
\end{figure}


According to FMI the high increase in shipping emissions from 2018 to 2019 is because much more small vessels are using AIS than in previous years, which means that emissions from this small vessels are included in the shipping emissions for 2019, but not in previous years.
Due to the selective implementation of the Sulphur Emission Control Areas (SECAs) on the North Sea and Baltic Sea only, the emission trends differ between those seas and the other seas.



\section{Summary}

This chapter summarises the status of emissions reporting by LRTAP Convention Parties and the extent to which these data have been incorporated into the 2021 EMEP emissions datasets for modellers. The chapter documents the historical improvement in reporting over time, noting the increasing extent of reporting of emissions inventories for the mandatory pollutants and black carbon, as well as increased reporting of gridded emissions in 2021 compared to 2017. Despite these positive trends in terms of reporting, reporting is not yet complete. For some parties, emissions inventories and gridded data are not reported (or are reported late and/or incomplete). There is further room for improvement on the reporting of particulate matter emissions with respect to whether the condensable component has been included in the reported estimates.

The 2021 EMEP emissions datasets for modellers therefore need to be complied carefully and this chapter documents for which countries and pollutants the time series have been based fully or partially on reported inventories and gridded data, and for which countries and regions the datasets have been built using independent emissions data products.

Based on the complied datasets in 2021, it is worth noting that the 2000 to 2019 trends in emissions from the land areas have decreased for most pollutants except for PM$_{coarse}$  and \nhiii. This trend appears to be driven by the trends of the EMEP West countries, for which the time series are based almost completely on reported data. In contrast, EMEP East as whole shows a rather stable trend in terms of emissions (emissions based partially on reported data), with notable emissions increases shown for the 'Other areas' (based completely on independent estimates). International shipping emissions had been shown to be decreasing up to 2018; however, a notable jump in 2018  to 2019 emissions of all pollutants likely stems from a methodological artefact, whereby the number of smaller vessels using AIS systems (on which the CAMS global shipping dataset is based) seems to have increased.

\clearpage
\bibliographystyle{copernicus}         % change bibliography-name after each
\renewcommand\bibname{References}      % bibliographystyle command!
\addcontentsline{toc}{section}{References}
\bibliography{Refs,EMEP_Reports,RefsEmis}
