\chapter[Introduction]{Introduction}
\label{ch:Intro}

\old{TODO}

\section{Purpose and structure of this report}

The mandate of the European Monitoring and Evaluation Programme (EMEP)
is to provide sound scientific support to the Convention on Long-range
Transboundary Air Pollution (LRTAP), particularly in the areas of
atmospheric monitoring and modelling, emission inventories, emission
projections and integrated assessment. Each year EMEP provides
information on transboundary pollution fluxes inside the
EMEP area, relying on information on emission sources and
monitoring results provided by the Parties to the LRTAP Convention.

The purpose of the annual EMEP status reports is to provide an
overview of the status of transboundary air pollution in Europe,
tracing progress towards existing emission control Protocols and
supporting the design of new protocols, when necessary. An additional
purpose of these reports is to identify problem areas, new aspects
and findings that are relevant to the Convention.
%The progress according to the EMEP Workplan \citep{EMEP:WP2016} is also reported here. Table \ref{Tab:WP} give an overview of which items in the workplan that the different chapters report on.


%2016:
%\begin{table}[!ht]
%\caption{Overview of items from the EMEP workplan 2016-2017 that chapters report progress on. Other chapters report results for the mandatory work.}
%\label{Tab:WP}
%\centering
%\begin{tabular}{l|l}
%\hline\hline
%Chapter & Workplan item \\
%\hline
%\ref{ch:LocalFraction} & 1.1.1.4\\
%\ref{ch:ACTRIS} & XXX\\
%\ref{ch:ShipEmis} & 1.3.3\\
%\ref{ch:slcf}& 1.1.1.20\\
%\ref{ch:ObsDevel} & 1.2.1\\


%2015:
%\begin{table}[!ht]
%\caption{Overview of which items from the EMEP workplan that the chapters report progress on.}
%\label{Tab:WP}
%\centering
%\begin{tabular}{l|l}
%\hline\hline
%Chapter & Workplan item \\
%\hline
%\ref{ch:chapterStatus} & 1.1.4, 1.1.6, 1.1.7, 1.3.2, 1.3.3\\
%\ref{ch:emis2013}      & 1.4.1, 1.4.3, 1.4.4\\
%\ref{ch:Finegrid} & 1.3.4\\
%\ref{ch:EC} & 1.3.8\\
%\ref{ch:condensables} & 1.3.8\\
%\ref{ch:MineralDust} & 1.3.8\\
%\ref{ch:EMEP2014} & 1.1.4, 1.3.8    \\
%\ref{ch:O3Bias} & 1.3.8, 1.6.3\\
%\ref{ch:ModelUpdates}& 1.3.8\\
%\ref{ch:ObsDevel} & 1.1.1, 1.1.2, 1.1.3, 1.1.4, 1.1.5, 1.1.7, 1.1.8\\


%2014:
%\ref{ch:chapterStatus} & 1.1.4, 1.1.6, 1.1.7, 1.3.2\\
%\ref{ch:emis2012}      & 1.4.1, 1.4.3, 1.4.4\\
%\ref{ch:ClimAQ} & 1.3.8 \\
%\ref{ch:Sources}& 1.3.8, 1.1.5, 1.1.9\\
%\ref{ch:slcf} & 1.3.8 \\
%\ref{ch:newgridemi}& 1.3.1\\
%\ref{sc:EMEP grid} & 1.3.4\\
%\ref{ch:Global}& 1.3.10, 1.6.2, 1.6.3\\
%\ref{ch:ModelUpdates}& 1.3.8\\
%\ref{ch:ESX}& 1.3.8\\
%\ref{ch:ObsDevel} & 1.1.1, 1.1.2, 1.1.3, 1.1.4, 1.1.5, 1.1.7, 1.1.8\\

%&\\
%Appendix & Workplan item \\
%\hline
%\ref{ch:appx_emis_2014} & 1.3.2\\
%\ref{ch:appx_sr2014} & 1.3.2\\
%\ref{ch:appx_countryrep_2014} &1.3.2\\
%\ref{ch:appx_modeleval}&1.3.2\\
%\hline\hline
%\end{tabular}
%\vspace{0.05in}
%\end{table}

The present report is divided into four parts. Part I presents the status
of transboundary air pollution with respect to acidification, eutrophication,
ground level ozone and particulate matter in Europe in 2018.
Part II summarizes research activities of relevance to the
EMEP programme, while Part III deals with technical developments going on within the centres.

Appendix~\ref{ch:appx_emis_2018} in Part IV contains information on the national total emissions of main pollutants and  primary particles for 2018, while Appendix~\ref{ch:appx_emis_trends} shows the  emission trends for the period of 2000-2018. Country-to-count\-ry source-receptor matrices with calculations of
the transboundary contributions to pollution in different countries
for 2018 are presented in Appendix~\ref{ch:appx_sr2018}.

Appendix~\ref{ch:appx_countryrep_2018} describes the country
reports which are  issued as a supplement to the EMEP status reports.

Appendix~\ref{ch:appx_modeleval} introduces the model evaluation
report for 2018 \citep{WEB2020:Eval} which is available online and contains time series plots
of acidifying and eutrophying components
\citep{WEB2020:SN}, ozone \citep{WEB2020:O3} and particulate matter \citep{WEB2020:PM}. These plots are provided for all stations reporting to
EMEP (with just a few exclusions due to data-capture or technical problems).
This online information is complemented by numerical fields and other
information on the EMEP website. The reader is encouraged to visit the
website, \url{http://www.emep.int}, to access this additional information.



\section{Definitions, statistics used}
\label{DEFS}

For sulphur and nitrogen compounds, the basic units used throughout
this report are $\mu$g (S or N)/m$^{3}$ for air concentrations and
mg (S or N)/m$^{2}$ for depositions. Emission data, in particular in
some of the Appendices, is given in Gg (SO$_2$)  and Gg (NO$_2$) in
order to keep consistency with reported values.

For ozone, the basic units used throughout this report are ppb (1 ppb
= 1 part per billion by volume) or ppm (1 ppm = 1000 ppb).  At
20\degrees C and 1013 mb pressure, 1 ppb ozone is equivalent to
2.00~\ug.  \vspace{0.5cm}

\noindent

A number of statistics have been used to describe the distribution of
ozone within each grid square:\\
\begin{description}

     \item[Mean of Daily Max. Ozone] - First we evaluate the maximum
modelled concentration for each day, then we take either 6-monthly
(1 April - 30 September) or annual averages of these values.

     \item[SOMO35] - The Sum of Ozone Means Over 35 ppb is the
     indicator for health impact assessment recommended by WHO. It is
     defined as the yearly sum of the daily maximum of 8-hour running
     average over 35 ppb. For each day the maximum of the running
     8-hours average for O$_3$ is selected and the values over 35 ppb
     are summed over the whole year.

     If we let $A^d_8$ denote the
     maximum 8-hourly average ozone on day $d$, during a year with
     $N_y$ days ($N_y$ = 365 or 366), then SOMO35 can be defined as:

\begin{math}
SOMO35 = \sum_{d=1}^{d=N_y} \max\bigl(A^d_8 - 35 \mbox{\, ppb}, 0.0\bigr)
\end{math}
%SOMO35 = \sum_{days} max\bigl(dailymax(running\_8h\_average\_O_3) - 35 ppb, 0.0\bigr)

where the {\tt max} function evaluates $\max(A-B,0)$ to $A-B$ for $A > B$, or
zero if $A \leq B$, ensuring that only $A^d_8$ values exceeding 35
ppb are included.  The corresponding unit is ppb.days.


\item[POD$_Y$] - Phyto-toxic ozone dose, is the accumulated stomatal ozone flux over a threshold Y, i.e.:

\begin{equation}
   %OLD \mbox{AFstY}_{gen} = \int \max(\Fst - Y, 0)\   dt
   \mbox{POD}_{Y} = \int \max(\Fst - Y, 0)\   dt
\end{equation}

where stomatal flux \Fst, and threshold, $Y$, are in \nmole.
This integral is evaluated over time, from the
start of the growing season (SGS), to the end (EGS).

For the generic crop and forest species, the suffix $gen$ can be
applied, e.g. POD$_{Y, gen}$ (or \AFSTDF) is used for forests.
POD was introduced in 2009 as an easier and  more descriptive term for the
accumulated ozone flux. The definitions of AFst and POD are identical
however, and are discussed further in \citet{R2010:Fluxes}. See
also \citet{MillsGCB2011,MillsAE2011} and \citet{MillsGCB2018a}.


     \item[AOT40] -
is the accumulated amount of ozone over the threshold value of 40 ppb, i.e..

\begin{math}
AOT40 = \int \max(O_3 - 40 \mbox{\, ppb}, 0.0) \, dt
\end{math}


where the {\tt max} function ensures that only ozone values exceeding 40
ppb are included.  The integral is taken over time, namely the
relevant growing season for the vegetation concerned. The
corresponding unit are ppb.hours (abbreviated to ppb.h).  The usage
and definitions of AOT40 have changed over the years though, and also
differ between UNECE and the EU.
\cite{MappingManual:Veg} give the latest definitions for UNECE work, and
describes carefully how AOT40 values are best estimated for local
conditions (using information on real growing seasons for example),
and specific types of vegetation.  Further, since O$_3$ concentrations
can have strong vertical gradients, it is important to specify the
height of the O$_3$ concentrations used. In previous EMEP work we have
made use of modelled O$_3$ from 1~m or 3~m height, the former being
assumed close to the top of the vegetation, and the latter being
closer to the height of O$_3$ observations.  In the Mapping Manual
\cite[]{MappingManual:Veg} there is an increased emphasis on
estimating AOT40 using ozone levels at the top of the vegetation
canopy.

Although the EMEP MSC-W model now generates a number of AOT-related outputs,
in accordance with the recommendations of \cite{MappingManual:Veg}
we will concentrate in this report on two definitions:

\begin{description}
   \item[\aotucf] %\item[AOT40$_f^{uc}$]
     - AOT40 calculated for
   forests using estimates of O$_3$ at forest-top ($uc$:
   upper-canopy). This AOT40 is that defined for forests by
   \cite{MappingManual:Veg}, but using a default growing season of
   April-September.
   \item[\aotucc] %\item[AOT40$_c^{uc}$]
     - AOT40
   calculated for agricultural crops using estimates of O$_3$ at the
   top of the crop. This AOT40 is close to that defined for
   agricultural crops by \cite{MappingManual:Veg}, but using a default
   growing season of May-July, and a default crop-height of 1~m.
\end{description}

In all cases only daylight hours are included, and for practical
reasons we define daylight for the model outputs as the time when the
solar zenith angle is equal to or less than 89\degrees. (The proper
UNECE definition uses clear-sky global radiation exceeding 50 W
m$^{-2}$ to define daylight, whereas the EU AOT definitions use day
hours from 08:00-20:00.).
In the comparison of modelled and observed \aotucf in chapter \ref{ch:chapterStatus}, we have used the EU AOT definitions of day
hours from 08:00-20:00.

The AOT40 levels reflect interest in long-term ozone exposure which is
considered important for vegetation - critical levels of 3~000~ppb.h
have been suggested for agricultural crops and natural vegetation, and
5~000~ppb.h for forests \cite[]{MappingManual:Veg}.
Note that recent  UNECE workshops have recommended that AOT40 concepts
are replaced by ozone flux estimates for crops and forests.
\citep[See also][]{R2010:Fluxes}.
\end{description}



This report includes also concentrations of particulate matter
(PM). The basic units \linebreak throughout this report
are \ug for PM
concentrations and the following acronyms are used for different
components to PM:

\begin{description}

\item[POA] - primary organic aerosol - which is the organic component of
the PPM emissions (defined below). (POA is in this report assumed to
be entirely in the particle phase, see Chap.~\ref{ch:Condensables}.)
  
\item[SOA] - secondary organic aerosol, defined as the aerosol mass
  arising from the oxidation products of gas-phase organic species.

\item[SIA]- secondary inorganic aerosols, defined as the sum of
  sulphate (SO$^{2-}_4$), nitrate (NO$^-_3$) and ammonium (NH$^+_4$).
  In the EMEP MSC-W model SIA is calculated as the sum: SIA= SO$^{2-}_4$
  + NO$^-_3$(fine) + NO$^-_3$(coarse) + NH$^+_4$.

\item[SS] - sea salt.

\item[MinDust] - mineral dust.

\item[PPM] - primary particulate matter, originating directly from
  anthropogenic emissions. One usually distinguishes between fine
  primary particulate matter, PPM$_{2.5}$, with aerosol diameters
  below 2.5 $\mu$m and coarse primary particulate matter, PPM$_{coarse}$
  with aerosol diameters between 2.5 $\mu$m and 10 $\mu$m.

\item[PM$_{2.5}$] - particulate matter with aerodynamic diameter
  up to 2.5 $\mu$m. In the EMEP
  MSC-W model PM$_{2.5}$ is calculated as PM$_{2.5}$ = SO$^{2-}_4$
  + NO$^-_3$(fine) + NH$^+_4$ + SS(fine) + MinDust(fine)
  + SOA(fine) + PPM$_{2.5}$ + 0.27 NO$^-_3$(coarse) + PM25water.
  (PM25water = PM associated water).

\item[PM$_{\text{coarse}}$] - coarse particulate matter with aerodynamic
  diameter between 2.5$\mu$m 
  and 10$\mu$m. In the EMEP MSC-W model PM$_{\text{coarse}}$ is calculated
  as PM$_{\text{coarse}}$ = 0.73 NO$^-_3$(coarse)+ SS(coarse)
  + MinDust(coarse) + PPM$_{coarse}$.

\item[PM$_{10}$] - particulate matter with aerodynamic diameter
  up to 10 $\mu$m. In the EMEP
   MSC-W model PM$_{10}$ is calculated as PM$_{10}$ = PM$_{2.5}$
  + PM$_{\text{coarse}}$.

\end{description}

In addition to bias, correlation and root mean square the statistical
parameter, index of agreement, are used to judge the model's agreement
with measurements:\\
\begin{description}
%     \item[Bias] - $\frac{\overline{Mod}-\overline{Obs}}{\overline{Obs}}\times
%     100\%$ measured yearly average (Obs), modelled yearly average (Mod)
%     \item[Corr] - Correlation between observation and model for station
%     yearly averages.
     \item[IOA]  - The index of agreement (IOA) is defined as follows
     \citep{Willmott1981, Willmott1982}:
%     \vspace{0.3in}
%     \begin{math}
\begin{equation}
IOA=1-\frac{\sum_{i=1}^{N}(m_i-o_i)^2}{\sum_{i=1}^{N}(|m_i-\bar{o}|+|o_i-\bar{o}|)^2}
\label{eq:IOA}
\end{equation}
%     \end{math}\\
     where $\overline{o}$ is the average observed value. Similarly to
     correlation, IOA can be used to assess agreement either
     spatially or temporally.
     When IOA is used in a spatial sense, N denotes the number of stations
     with measurements at one specific point in time, and $m_i$ and $o_i$
     are the modelled and observed values at station $i$.
     For temporal IOA, N denotes the number of time steps with measurements,
     while $m_i$ and $o_i$ are the modelled and observed value at time step $i$.
     IOA varies between 0 and 1. A value of 1 corresponds to perfect agreement
     between model and observations, and 0 is the theoretical minimum.

\end{description}


\section{The EMEP grid}
\label{EMEPgrid}

At the 36$^{th}$ session of the EMEP Steering Body the EMEP Centres suggested 
to increase spatial resolution and projection of reported emissions from 50$\times$50~km$^{2}$ polar stereographic grid to {0.1\degrees $\times$0.1\degrees} longitude-latitude grid in a geographic coordinate system 
(WGS84). The EMEP domain shown in Figure~\ref{fig:lonlatgrid} covers 
the geographic area between 30\degrees N-82\degrees N latitude and 30\degrees 
W-90\degrees E longitude. This domain 
represents a balance between political needs, scientific needs and technical 
feasibility. Parties are obliged to report gridded emissions in this grid resolution from year 2017.


\begin{figure}[h]
\centering
\includegraphics*[viewport=25 150 570 680,clip,scale=0.5]{FIGS_INTRO/lonlatgrid.pdf}
\caption{The EMEP domain covering the geographic area between 30\degrees N-82\degrees N latitude and 30\degrees 
W-90\degrees E longitude.}
\label{fig:lonlatgrid}
\end{figure}


The higher resolution means an increase of grid cells from approximately 
21500 cells in the 50$\times$50 km$^2$ grid to 624000 cells in the  {0.1\degrees $\times$0.1\degrees} longitude-latitude grid.

\subsection{The reduced grid: EMEP0302}

For practical purposes, a coarser grid has also been defined. The EMEP0302 grid covers the same region as the  {0.1\degrees $\times$0.1\degrees} longitude-latitude EMEP domain (Figure~\ref{fig:lonlatgrid}), but the spatial resolution is 0.3{\degrees} in the longitude direction and  0.2{\degrees} in the latitude direction. Each gridcell from the EMEP0302 grid covers exactly 6 gridcells from the {0.1\degrees $\times$0.1\degrees} official grid. %A gridcell is approximatively square at 48.2 \degrees latitude = acos(2/3) 


\begin{table}[!ht]
\begin{center}
\begin{small}
\begin{tabular}{|l|l|c|l|l|}
\cline{1-2} \cline{4-5}
{\rule[-3mm]{0mm}{8mm}\textbf{Code}}&\textbf{Country/Region/Source}&&\textbf{Code}&\textbf{Country/Region/Source}\\  \cline{1-2} \cline{4-5}
AL & Albania & & IS & Iceland \\  \cline{1-2} \cline{4-5}
AM & Armenia & &  IT & Italy\\  \cline{1-2} \cline{4-5}
AST & Asian areas & &  KG & Kyrgyzstan \\  \cline{1-2} \cline{4-5}
AT & Austria & &  KZ & Kazakhstan \\  \cline{1-2} \cline{4-5}
ATL & N.-E. Atlantic Ocean & & LI & Liechtenstein \\  \cline{1-2} \cline{4-5}
AZ & Azerbaijan & & LT & Lithuania\\  \cline{1-2} \cline{4-5}
BA & Bosnia and Herzegovina & & LU & Luxembourg \\  \cline{1-2} \cline{4-5}
BAS & Baltic Sea & & LV & Latvia \\  \cline{1-2} \cline{4-5}
BE & Belgium & &  MC & Monaco\\  \cline{1-2} \cline{4-5}
BG & Bulgaria & & MD & Moldova \\  \cline{1-2} \cline{4-5}
BIC & Boundary/Initial Conditions & & ME & Montenegro \\  \cline{1-2} \cline{4-5}
BLS & Black Sea & & MED & Mediterranean Sea \\  \cline{1-2} \cline{4-5}
BY & Belarus & & MK & North Macedonia \\  \cline{1-2} \cline{4-5}
CH & Switzerland & & MT & Malta \\  \cline{1-2} \cline{4-5}
CY & Cyprus & & NL & Netherlands \\  \cline{1-2} \cline{4-5}
CZ & Czechia & & NO & Norway \\  \cline{1-2} \cline{4-5}
DE & Germany & & NOA & North Africa \\  \cline{1-2} \cline{4-5}
DK & Denmark & & NOS & North Sea \\  \cline{1-2} \cline{4-5}
DMS & Dimethyl sulfate (marine) & & PL & Poland \\  \cline{1-2} \cline{4-5}
EE & Estonia & & PT & Portugal \\  \cline{1-2} \cline{4-5}
ES & Spain & & RO & Romania \\  \cline{1-2} \cline{4-5}
EU & European Union (EU28) & & RS & Serbia \\  \cline{1-2} \cline{4-5}
EXC & EMEP land areas & & RU & Russian Federation  \\  \cline{1-2} \cline{4-5}
FI & Finland & & SE & Sweden \\  \cline{1-2} \cline{4-5}
FR & France & &  SI & Slovenia\\  \cline{1-2} \cline{4-5}
GB & United Kingdom & & SK & Slovakia\\  \cline{1-2} \cline{4-5}
GE & Georgia & & TJ & Tajikistan\\  \cline{1-2} \cline{4-5}
GL & Greenland & & TM & Turkmenistan \\  \cline{1-2} \cline{4-5}
GR & Greece & & TR & Turkey \\  \cline{1-2} \cline{4-5}
HR & Croatia & & UA & Ukraine \\  \cline{1-2} \cline{4-5}
HU & Hungary & & UZ & Uzbekistan \\  \cline{1-2} \cline{4-5}
IE & Ireland & & VOL & Volcanic emissions \\  \cline{1-2} \cline{4-5}
\end{tabular}
\end{small}
\caption{Country/region codes used throughout this report.}
\label{tab:countries}
\vspace{0.05in}
\end{center}


\end{table}

\section{Country codes}

Several tables and graphs in this report make use of codes to denote
countries and regions in the EMEP area. Table~\ref{tab:countries}
provides an overview of these codes and lists the countries and
regions included.

All 51 Parties to the LRTAP Convention, except two, are included in the
analysis presented in this report. The Parties that are excluded of the
analysis are Canada and the United States of America, because
they lie outside the EMEP domain.

%% Monaco and Liechtenstein are
%% excluded because their emissions and geographical extents are
%% below the accuracy of the present source-receptor calculations in
%% 50$\times$50km$^{2}$.

%% Malta, Monaco and Liechtenstein are introduced as a receptor country. However,
%% the estimated emissions
%% from Malta are below the accuracy limit of the source-receptor calculations
%% and do not justify a separate study of Malta as an emitter country.




\section{Other publications}
\label{sec:publ}
This report is complemented by a report on EMEP MSC-W model performance for acidifying and eutrophying components, photo-oxidants and particulate matter in 2018 \citep{WEB2020:Eval}, made available online, at \url{www.emep.int}.

%% This report is complemented by the country specific reports on the
%% 2014 status of transboundary acidification, eutrophication, ground level
%% ozone and PM (see Apendix~\ref{ch:appx_countryrep_2014}). Both English
%% and Russian versions of the country reports are available to the twelve EECCA
%% countries.

%As noted above, time series plots of acidifying and eutrophying components
%\citep{WEB2017:SN}, and ozone and NO$_2$ \citep{WEB2017:O3} have been made
%available online, at \url{www.emep.int}.% along with much other material.

A list of all associated technical reports and notes  by the EMEP
centres in 2020 (relevant for transboundary acidification, eutrophication,
ozone and particulate matter) follows at the end of this section.

%\newpage
\subsection*{Peer-reviewed publications in 2019}

The following scientific papers of relevance to transboundary acidification, eutrophication, ground level ozone and particulate matter, involving EMEP/MSC-W and EMEP/CCC staff, have become available in 2019:

\enlargethispage{\baselineskip}
\begin{list}{}{\setlength{\leftmargin}{15pt}\setlength{\itemindent}{-\leftmargin}}\small
%%%%%%%%%%%%%%%%%%%%%%%%%%%%%%%%%%%%%%%%%%%%%%%%%%%%%%%%%%%%%%%%%%%%%%%%
\item[]
Aas, Wenche; Mortier, Augustin; Bowersox, Van C.; Cherian, Ribu; Faluvegi, Greg; Fagerli, Hilde; Hand, Jenny; Klimont, Zbigniew; Galy-Lacaux, Corinne; Lehmann, Christopher M. B.; Myhre, Cathrine Lund; Myhre, Gunnar; Oliviè, Dirk Jan Leo; Sato, Keiichi; Quaas, Johannes; Rao, P.S.P.; Schulz, Michael; Shindell, Drew; Skeie, Ragnhild Bieltvedt; Stein, Ariel; Takemura, Toshihiko; Tsyro, Svetlana; Vet, Robert; Xu, Xiaobin.
Global and regional trends of atmospheric sulfur.
Scientific Reports , 9 , 2019.
DOI: 10.1038/s41598-018-37304-0

\item[]
Barregård, Lars; Molnar, Peter; Jonson, Jan Eiof; Stockfelt, Leo.
Impact on Population Health of Baltic Shipping Emissions.
International Journal of Environmental Research and Public Health 2019 ; 16.(11)
DOI: 10.3390/ijerph16111954

\item[]
Boy, Michael; Thomson, Erik S.; Acosta Navarro, Juan-Camilo; Arnalds, Olafur; Batchvarova, Ekaterina; Bäck, Jaana; Berninger, Frank; Bilde, Merete; Brasseur, Zoe; Dagsson-Waldhauserova, Pavla; Castarede, Dimitri; Dalirian, Maryam; de Leeuw, Gerrit; Dragosics, Monika; Duplissy, Ella-Maria; Duplissy, Jonathan; Ekman, Annica; Fang, Keyan; Gallet, Jean-Charles; Glasius, Marianne; Gryning, Sven-Erik; Grythe, Henrik; Hansson, Hans-Christen; Hansson, Margareta; Isaksson, Elisabeth; Iversen, Trond; Jónsdóttir, Ingibjörg Svala; Kasurinen, Ville; Kirkevåg, Alf. ; Korhola, Atte; Krejci, Radovan; Kristjansson, Jon Egill; Lappalainen, Hanna K.; Lauri, Antti; Leppäranta, Matti; Lihavainen, Heikki; Makkonen, Risto; Massling, Andreas; Meinander, Outi; Nilsson, E. Douglas; Ólafsson, Haraldur; Pettersson, Jan B. C.; Prisle, Nønne L.; Riipinen, Ilona; Roldin, Pontus; Ruppel, Meri; Salter, Matthew E.; Sand, Maria; Seland, Øyvind; Seppä, Heikki; Skov, Henrik; Soares, Joana; Stohl, Andreas; Ström, Johan; Svensson, Jonas; Swietlicki, Erik; Tabakova, Ksenia; Thorsteinsson, Throstur; Virkkula, Aki; Weyhenmeyer, Gesa A.; Wu, Yusheng; Zieger, Paul; Kulmala, Markku.
Interactions between the atmosphere, cryosphere,and ecosystems at northern high latitudes. .
Atmospheric Chemistry and Physics , 2019, p. 2015-2061.
DOI: 10.5194/acp-19-2015-2019

\item[]
Brasseur, Guy P.; Xie, Ying; Petersen, Anna Katinka; Bouarar, Idir; Flemming, Johannes; Gauss, Michael; Jiang, Fei; Kouznetsov, Rostislav; Kranenburg, Richard; Mijling, Bas; Peuch, Vincent-Henri; Pommier, Matthieu; Segers, Arjo; Sofiev, Mikhail; Timmermans, Renske; van der A, Ronald; Walters, Stacy; Xu, Jianming; Zhou, Guangqiang.
Ensemble forecasts of air quality in eastern China-Part 1: Model description and implementation of the MarcoPolo-Panda prediction system, version 1.
Geoscientific Model Development 2019 ; 12.(1) p. 33-67
DOI: 10.5194/gmd-12-33-2019

\item[]
Christensen, Torben R.; Arora, Vivek K; Gauss, Michael; Hoglund-Isaksson, Lena; Parmentier, Frans-Jan W.
Tracing the climate signal: mitigation of anthropogenic methane emissions can outweigh a large Arctic natural emission increase.
Scientific Reports , 9 (1146) , 2019.
DOI: 10.1038/s41598-018-37719-9

\item[]
Ciarelli, Giancarlo; Colette, Augustin; Schucht, Simone; Beekmann, Matthias; Andersson, Camilla; Manders-Groot, Astrid; Mircea, Mihaela; Tsyro, Svetlana; Fagerli, Hilde; Ortiz, Alberto González; Adani, Mario; Briganti, Gino; Cappelletti, Andrea; D'Isidoro, Massimo; Cuvelier, Cornelis; Couvidat, Florian; Meleux, Frédérik; Bessagnet, Bertrand.
Long-term health impact assessment of total PM2.5 in Europe during the 1990–2015 period.
Atmospheric Environment: X 2019 ; 3.
DOI: 10.1016/j.aeaoa.2019.100032

\item[]
Ciarelli, Giancarlo; Theobald, Mark, R.; Vivanco, Marta García; Beekmann, Matthias; Aas, Wenche; Andersson, Camilla; Bergström, Robert; Manders-Groot, Astrid; Couvidat, Florian; Mircea, Mihaela; Tsyro, Svetlana; Fagerli, Hilde; Mar, Kathleen; Raffort, Valentin; Roustan, Yelva; Pay, Maria-Teresa; Schaap, Martijn; Kranenburg, Richard; Adani, Mario; Briganti, Gino; Cappelletti, Andrea; D'Isidoro, Massimo; Cuvelier, Cornelis; Cholakian, Arineh; Bessagnet, Bertrand; Wind, Peter; Colette, Augustin.
Trends of inorganic and organic aerosols and precursor gases in Europe: insights from the EURODELTA multi-model experiment over the 1990–2010 period.
Geoscientific Model Development 2019 ; 12.(12) p. 4923-4954
DOI: 10.5194/gmd-12-4923-2019

\item[]
Chatoutsidou, S.E., Kopanakis, I., Lagouvardos, K., Mihalopoulos N., Tørseth. K., and Lazaridis M.:
PM10 levels at urban, suburban, and background locations in the eastern Mediterranean: local versus regional sources with emphasis on African dust. 
Air Qual Atmos Health 12, 1359–1371 (2019). 
https://doi.org/10.1007/s11869-019-00750-4

\item[]
Evangeliou, N., Kylling, A., Eckhardt, S., Myroniuk, V., Stebel, K., Paugam, R., Zibtsev, S., and Stohl, A.: 
Open fires in Greenland in summer 2017: transport, deposition and radiative effects of BC, OC and BrC emissions, 
Atmos. Chem. Phys., 19, 1393–1411, https://doi.org/10.5194/acp-19-1393-2019, 2019.

\item[]
Fanourgakis, George S.; Kanakidou, Maria; Nenes, Athanasios; Bauer, Susanne E.; Bergman, Tommi; Carslaw, Kenneth S.; Grini, Alf; Hamilton, Douglas S.; Johnson, Jill S.; Karydis, Vlassis A.; Kirkevåg, Alf; Kodros, John K.; Lohmann, Ulrike; Luo, Gan; Makkonen, Risto; Matsui, Hitoshi; Neubauer, David; Pierce, Jeffrey Robert; Schmale, Julia; Stier, Philip; Tsigaridis, Kostas; van Noije, Twan P.C.; Wang, Hailong; Watson-Parris, Duncan; Westerveld, Daniel M.; Yang, Yang; Yoshioka, Masaru; Daskalakis, Nikos; Decesari, Stefano; Gysel-Beer, Martin; Kalivitis, Nikos; Liu, Xiaohong; Mahowald, Natalie M.; Myriokefalitakis, Stelios; Schrödner, Roland; Sfakianaki, Maria; Tsimpidi, Alexandra P.; Wu, Mingxuan; Yu, Fangqun. Evaluation of global simulations of aerosol particle and cloud condensation nuclei number, with implications for cloud droplet formation.
Atmospheric Chemistry and Physics 2019
DOI: 10.5194/acp-19-8591-2019

\item[]
Graff, Lise Seland; Iversen, Trond; Bethke, Ingo; Debernard, Jens Boldingh; Seland, Øyvind; Bentsen, Mats; Kirkevåg, Alf; Li, Camille; Oliviè, Dirk Jan Leo.
Arctic amplification under global warming of 1.5 and 2\degC in NorESM1-Happi.
Earth System Dynamics 2019 ; 10.(3) p. 569-598
DOI: 10.5194/esd-10-569-2019

\item[]
Grennfelt, Peringe; Engleryd, Anna; Forsius, Martin; Hov, Øystein; Rodhe, Henning; Cowling, Ellis.
Acid rain and air pollution: 50 years of progress in environmental science and policy.
Ambio 2019 p. 1-16
DOI: 10.1007/s13280-019-01244-4

\item[]
Guarin, Jose Rafael; Emberson, Lisa; Simpson, David; Hernandez-Ochoa, Ixchel M; Rowland, Diane; Asseng, Senthold.
Impacts of tropospheric ozone and climate change on Mexico wheat production.
Climatic Change 2019 ; 155.(2) p. 157-174
DOI: 10.1007/s10584-019-02451-4

\item[]
Heinze, Christoph; Eyring, Veronika; Friedlingstein, Pierre; Jones, Colin; Balkanski, Yves; Collins, William; Fichefet, Thierry; Gao, Shuang; Hall, Alex; Ivanova, Detelina; Knorr, Wolfgang; Knutti, Reto; Löw, Alexander; Ponater, Michael; Schultz, Martin G.; Schulz, Michael; Siebesma, Pier; Teixeira, Joao; Tselioudis, George; Vancoppenolle, Martin.
ESD Reviews: Climate feedbacks in the Earth system and prospects for their evaluation.
Earth System Dynamics 2019 ; 10.(3) p. 379-452
DOI: 10.5194/esd-10-379-2019

\item[]
Hodnebrog, Øivind; Myhre, Gunnar; Samset, Bjørn Hallvard; Alterskjær, Kari; Andrews, Timothy; Boucher, Oliviér; Faluvegi, Gregory S.; Fläschner, Dagmar; Forster, Piers M.; Kasoar, Matthew; Kirkevåg, Alf; Lamarque, Jean-François; Oliviè, Dirk Jan Leo; Richardson, Thomas Benjamin; Shawki, Dilshad; Shindell, Drew; Shine, Keith P; Stier, Philip; Takemura, Toshihiko; Voulgarakis, Apostolos; Watson-Parris, Duncan.
Water vapour adjustments and responses differ between climate drivers.
Atmospheric Chemistry and Physics 2019 ; 19.(20) p. 12887-12899
DOI: 10.5194/acp-19-12887-2019

\item[]
Inness, Antje; Ades, Melanie; Agustí-Panareda, Anna; Barré, Jérôme; Benedictow, Anna Maria Katarina; Blechschmidt, Anne-Marlene; Dominguez, Juan Jose; Engelen, Richard J.; Eskes, Henk; Flemming, Johannes; Huijnen, Vincent; Jones, Luke; Kipling, Zak; Massart, Sebastien; Parrington, Mark; Peuch, Vincent-Henri; Razinger, Miha; Remy, Samuel; Schulz, Michael; Suttie, Martin.
The CAMS reanalysis of atmospheric composition.
Atmospheric Chemistry and Physics 2019 ; 19.(6) p. 3515-3556
DOI: 10.5194/acp-19-3515-2019

\item[]
Jacobi, H.-W., Obleitner, F., Da Costa, S., Ginot, P., Eleftheriadis, K., Aas, W., and Zanatta, M.: Deposition of ionic species and black carbon to the Arctic snowpack: combining snow pit observations with modeling, 
Atmos. Chem. Phys., 19, 10361–10377, 
https://doi.org/10.5194/acp-19-10361-2019, 2019.

\item[]
Jenkin, Michael E.; Khan, Md Anwar Hossain; Shallcross, Dudley E; Bergström, Robert; Simpson, David; Murphy, KLC; Rickard, Andrew R.
The CRI v2.2 reduced degradation scheme for isoprene.
Atmospheric Environment 2019 ; 212. p. 172-182
DOI: 10.5194/acp-19-7019-2019

\item[]
Jonson, Jan Eiof; Gauss, Michael; Jalkanen, Jukka Pekka; Johansson, Lasse.
Effects of strengthening the Baltic Sea ECA regulations.
Atmospheric Chemistry and Physics 2019 ; 19.(21) p. 13469-13487
DOI: 10.5194/acp-19-13469-2019

\item[]
Kang, Sarah M.; Hawcroft, Matt; Xiang, Baoqiang; Hwang, Yen-Ting; Cazes, Gabriel; Codron, Francis; Crueger, Traute; Deser, Clara; Hodnebrog, Øivind; Kim, Hanjun; Kim, Jiyeong; Kosaka, Yu; Losada, Teresa; Mechoso, Carlos R.; Myhre, Gunnar; Seland, Øyvind; Stevens, Björn; Watanabe, Masahiro; Yu, Sungduk.
Extratropical–Tropical Interaction Model Intercomparison Project (ETIN-MIP): Protocol and Initial results.
Bulletin of The American Meteorological Society - (BAMS) 2019 ; 100.(12) p. 2589-2605
DOI: 10.1175/BAMS-D-18-0301.1

\item[]
Karl, Matthias; Jonson, Jan Eiof; Uppstu, Andreas; Aulinger, Armin; Prank, Marje; Sofiev, Mikhail; Jalkanen, Jukka Pekka; Johansson, Lasse; Quante, Markus; Matthias, Volker.
Effects of ship emissions on air quality in the Baltic Sea region simulated with three different chemistry transport models.
Atmospheric Chemistry and Physics 2019 ; 19.(10) p. 7019-7053
DOI: 10.5194/acp-19-7019-2019

\item[]
Lin, Yan; Couture, Raoul-Marie; Klein, Heiko; Ytre-Eide, Martin; Dyve, Jan Erik; Lind, Ole Christian; Bartnicki, Jerzy; Nizzetto, Luca; Butterfield, Daniel; Larssen, Thorjørn; Salbu, Brit.
Modelling environmental impacts of Cesium-137 under a hypothetical release of radioactive waste.
Bulletin of Environmental Contamination and Toxicology 2019 ; 103.(1) p. 69-74
DOI: 10.1007/s00128-019-02601-5

\item[]
Madakumbura, Gavin D.; Kim, Hyungjun; Utsumi, Nobuyuki; Shiogama, Hideo; Fischer, Erich M.; Seland, Øyvind; Scinocca, John F.; Mitchell, Daniel M.; Hirabayashi, Yukiko; Oki, Taikan.
Event-to-event intensification of the hydrologic cycle from 1.5 °C to a 2 °C warmer world.
Scientific Reports 2019 ; 9.
DOI: 10.1038/s41598-019-39936-2

\item[]
McFiggans, Gordon; Mentel, Thomas F.; Wildt, Jürgen; Pullinen, Iida; Kang, Sungah; Kleist, Einhard; Schmitt, Sebastian; Springer, Monika; Tillmann, Ralf; Wu, Cheng; Zhao, Defeng; Hallquist, Mattias; Faxon, Cameron; Le Breton, Michael; Hallquist, Åsa M; Simpson, David; Bergström, Robert; Jenkin, Michael E.; Ehn, Mikael; Thornton, Joel A.; Alfarra, M. Rami; Bannan, Thomas J.; Percival, Carl J.; Priestley, Michael; Topping, David; Kiendler-Scharr, Astrid.
Secondary organic aerosol reduced by mixture of atmospheric vapours.
Nature 2019 ; 565. p. 587-593
DOI: 10.1038/s41586-018-0871-y

\item[]
McClintock, MA; McDowell, William H; Gonzalez, Grizelle; Schulz, Michael; Pett-Ridge, Julie C.
African dust deposition in Puerto Rico: Analysis of a 20-year rainfall chemistry record and comparison with models.
Atmospheric Environment 2019 ; 216. p. 1-11
DOI: 10.1016/j.atmosenv.2019.\hfill\break 116907

\item[]
Mircea, Mihaela; Bessagnet, Bertrand; D'Isidoro, Massimo; Pirovano, Guido; Aksoyoglu, Sebnem; Ciarelli, Giancarlo; Tsyro, Svetlana; Manders, Astrid; Bieser, Johannes; Stern, Rainer; Vivanco, Marta García; Cuvelier, Cornelius; Aas, Wenche; Prévôt, André S.H.; Aulinger, Armin; Briganti, Gino; Calori, Giuseppe; Cappelletti, Andrea; Colette, Augustin; Couvidat, Florian; Fagerli, Hilde; Finardi, Sandro; Kranenburg, Richard; Rouil, Laurence; Silibello, Camillo; Spindler, Gerald; Poulain, Laurent; Herrmann, Hartmut; Jimenez, Jose L.; Day, Douglas A.; Tiitta, Petri; Carbone, Samara.
EURODELTA III exercise: An evaluation of air quality models' capacity to reproduce the carbonaceous aerosol.
Atmospheric Environment: X 2019 ; 2.
DOI: 10.1016/j.aeaoa.2019.100018

\item[]
Myhre, Gunnar; Alterskjær, Kari; Stjern, Camilla Weum; Hodnebrog, Øivind; Marelle, Louis; Samset, Bjørn Hallvard; Sillmann, Jana; Schaller, Nathalie; Fischer, Erich; Schulz, Michael; Stohl, Andreas.
Frequency of extreme precipitation increases extensively with event rareness under global warming.
Scientific Reports 2019 ; 9.(16063)
DOI: 10.1038/s41598-019-52277-4

\item[]
Petersen, Anna Katinka; Brasseur, Guy P.; Bouarar, Idir; Flemming, Johannes; Gauss, Michael; Jiang, Fei; Kouznetsov, Rostislav; Kranenburg, Richard; Mijling, Bas; Peuch, Vincent-Henry; Pommier, Matthieu; Segers, Arjo; Sofiev, Mikhail; Timmermans, Renske; van der A, Ronald; Walters, Stacy; Xie, Ying; Xu, Jianming; Zhou, Guangqiang.
Ensemble forecasts of air quality in eastern China – Part 2: Evaluation of the MarcoPolo–Panda prediction system, version 1.
Geoscientific Model Development 2019; 12. p. 1241-1266
DOI: 10.5194/gmd-12-1241-2019

\item[]
Richardson, Thomas B.; Forster, Piers M.; Smith, Christopher J.; Maycock, Amanda C.; Wood, Tom; Andrews, Timothy; Boucher, Oliviér; Faluvegi, Gregory; Fläschner, Dagmar; Hodnebrog, Øivind; Kasoar, Matthew; Kirkevåg, Alf; Lamarque, Jean-François; Mülmenstädt, Johannes; Myhre, Gunnar; Oliviè, Dirk Jan Leo; Portmann, Robert W.; Samset, Bjørn Hallvard; Shawki, Dilshad; Shindell, Drew T.; Stier, Philip; Takemura, Toshihiko; Voulgarakis, Apostolos; Watson-Parris, Duncan.
Efficacy of climate forcings in PDRMIP models.
Journal of Geophysical Research (JGR): Space Physics 2019 ; 124.(23) p. 12824-12844
DOI: 10.1029/2019JD030581

\item[]
Scannel, Claire; Booth, Ben B. B.; Dunstone, Nick J.; Rowell, David P.; Bernie, Dan J.; Kasoar, Matthew; Voulgarakis, Apostolos; Wilcox, Laura J.; Acosta Navarro, Juan C.; Seland, Øyvind; Paynterj, David J.
The Influence of Remote Aerosol Forcing from Industrialized Economies on the Future Evolution of East and West African Rainfall.
Journal of Climate 2019 ; 32.(23) p. 8335-8354
DOI: 10.1175/JCLI-D-18-0716.1

\item[]
Sillmann, Jana; Stjern, Camilla Weum; Myhre, Gunnar; Samset, Bjørn Hallvard; Hodnebrog, Øivind; Andrews, Timothy; Boucher, Oliviér; Faluvegi, Gregory; Forster, Piers; Kasoar, Matthew R.; Kharin, Viatcheslav V.; Kirkevåg, Alf; Lamarque, Jean-Francois; Oliviè, Dirk Jan Leo; Richardson, Thomas B.; Shindell, Drew; Takemura, Toshihiko; Voulgarakis, Apostolos; Zwiers, Francis W..
Extreme wet and dry conditions affected differently by greenhouse gases and aerosols.
npj Climate and Atmospheric Science 2019 ; 2.
DOI: 10.1038/s41612-019-0079-3

\item[]
Stjern, Camilla Weum; Lund, Marianne Tronstad; Samset, Bjørn Hallvard; Myhre, Gunnar; Forster, Piers M.; Andrews, Timothy; Boucher, Olivier; Faluvegi, Gregory; Fläschner, Dagmar; Iversen, Trond; Kasoar, Matthew; Kharin, Viatcheslav; Kirkevåg, Alf; Lamarque, Jean-François; Oliviè, Dirk Jan Leo; Richardson, Thomas; Sand, Maria; Shawki, Dilshad; Shindell, Drew; Smith, Christopher J.; Takemura, Toshihiko; Voulgarakis, Apostolos.
Arctic Amplification Response to Individual Climate Drivers.
Journal of Geophysical Research (JGR): Space Physics 2019 ; 124.(13) p. 6698-6717
DOI: 10.1029/2018JD029726

\item[]
Stojiljkovic, Ana; Kauhaniemi, Mari; Kukkonen, Jaakko; Kupiainen, Kaarle; Karppinen, Ari; Denby, Bruce; Kousa, Anu; Niemi, Jarkko V; Ketzel, Matthias.
The impact of measures to reduce ambient air PM10 concentrations originating from road dust, evaluated for a street canyon in Helsinki.
Atmospheric Chemistry and Physics 2019 ; 19. (17) p. 11199-11212
DOI: 10.5194/acp-19-11199-2019

\item[]
Tang, Tao; Shindell, D; Faluvegi, G.; Myhre, Gunnar; Oliviè, Dirk Jan Leo; Voulgarakis, A.; Kasoar, Matthew; Andrews, T.; Boucher, O.; Forster, Piers M.; Hodnebrog, Øivind; Iversen, Trond; Kirkevåg, Alf; Lamarque, Jean-François; Richardson, T.; Samset, Bjørn Hallvard; Stjern, Camilla Weum; Takemura, Toshihiko; Smith, C..
Comparison of Effective Radiative Forcing Calculations Using
Multiple Methods, Drivers, and Models. Journal of Geophysical Research (JGR): Space Physics 2019 ; 124 .(8) p. 4382-4394
DOI: 10.1029/2018JD030188

\item[]
Theobald, Mark R.; Vivanco, Marta G.; Aas, Wenche; Andersson, Camilla; Ciarelli, Giancarlo; Couvidat, Florian; Cuvelier, Kees; Manders, Astrid; Mircea, Mihaela; Pay, Maria-Teresa; Tsyro, Svetlana; Adani, Mario; Bergström, Robert; Bessagnet, Bertrand; Briganti, Gino; Cappelletti, Andrea; D'Isidoro, Massimo; Fagerli, Hilde; Mar, Kathleen; Otero, Noelia; Raffort, Valentin; Roustan, Yelva; Schaap, Martijn; Wind, Peter; Colette, Augustin.
An evaluation of European nitrogen and sulfur wet deposition and their trends estimated by six chemistry transport models for the period 1990–2010.
Atmospheric Chemistry and Physics , 19 . p. 379-405, 2019.
DOI: 10.5194/acp-19-379-2019

\item[]
Yttri, Karl Espen; Simpson, David; Bergström, Robert; Kiss, Gyula; Szidat, Sönke; Ceburnis, Darius; Eckhardt, Sabine; Hueglin, Christoph; Nøjgaard, Jacob Klenø; Perrino, Cinzia; Pisso, Ignacio; Prévôt, André Stephan Henry; Putaud, Jean-Philippe; Spindler, Gerald; Vána, Milan; Zhang, Yan-Lin; Aas, Wenche.
The EMEP Intensive Measurement Period campaign, 2008–2009: characterizing carbonaceous aerosol at nine rural sites in Europe.
Atmospheric Chemistry and Physics 2019 ; 19 . p. 4211-4233
DOI: 10.5194/acp-19-4211-2019

\item[]
Winiger, P.; Barrett, T. E.; Sheesley, R. J.; Huang, L.; Sharma, S.; Barrie, L. A.; K.E. Yttri, N. Evangeliou, S. Eckhardt, A. Stohl, Klimont, Z.; Heyes, C.; Semiletov, I. P.; Dudarev, O. V.; Charkin, A.; Shakhova, N.; Holmstrand, H.; Andersson, A.; Gustafsson, Ö.
Source apportionment of circum-Arctic atmospheric black carbon from isotopes and modeling
Sci. Adv., 5 (2) (2019), Article eaau8052, doi:10.1126/sciadv.aau8052

\end{list}


\subsection*{Associated EMEP reports and notes in 2020}

\leftline{\bf Joint reports}
\vspace{0.5cm}

\enlargethispage{\baselineskip}
\begin{list}{}{\setlength{\leftmargin}{15pt}\setlength{\itemindent}{-\leftmargin}}\small
% +
\item[]
Transboundary particulate matter, photo-oxidants, acidification and eutrophication components. Joint MSC-W \& CCC \& CEIP Report. EMEP Status Report 1/2020

\item[] EMEP MSC-W model performance for acidifying and
  eutrophying components, photo-oxidants and particulate matter in
  2018. Supplementary material to EMEP Status Report 1/2020

\item[]
 Assessment of transboundary pollution by toxic substances: Heavy metals and POPs.  Joint MSC-E  \& CCC  \& CEIP  \& INERIS Report. EMEP Status Report 2/2020

\end{list}


%% \leftline{\bf CIAM Technical and Data reports}

%% \enlargethispage{\baselineskip}
%% \begin{list}{}{\setlength{\leftmargin}{15pt}\setlength{\itemindent}{-\leftmargin}}\small

%% \item[] Wagner F., Winiwarter W., Klimont, Z., Amann, M., Sutton, M.
%% Ammonia reductions and costs implied by the three ambition levels
%% proposed in the Draft Annex IX to the Gothenburg protocol.
%% CIAM 5/2011–2 May, 2012


%% \item[] Amann, M., Bertok, I., Borken-Kleefeld, J., Cofala, J., Heyes, C.,
%%  H\"oglund-Isaksson, L., Klimont, Z., Rafaj, P., Sch\"opp, W., and Wagner, F.
%% Environmental Improvements of the Revision of the Gothenburg Protocol.
%% CIAM 1/2012.

%% \end{list}


\leftline{\bf CCC Technical and Data reports}
%NB Check if it is correct for 2020! E.g. authors and titles
\enlargethispage{\baselineskip}
\begin{list}{}{\setlength{\leftmargin}{15pt}\setlength{\itemindent}{-\leftmargin}}\small
\item[]
Anne-Gunn Hjellbrekke. 
Data Report 2018. Particulate matter, carbonaceous and inorganic compounds.
EMEP/CCC-Report 1/2020

\item[]
Anne-Gunn Hjellbrekke and Sverre Solberg. 
Ozone measurements 2018. 
EMEP/CCC-Report 2/2020

\item[]
Wenche Aas and Pernilla Bohlin Nizzetto. 
Heavy metals and POP measurements 2018.
EMEP/CCC-Report 3/2020

\item[]
Sverre Solberg, Anja Claude and Stefan Reimann. 
VOC measurements 2018. 
EMEP/CCC-Report 4/2020

\item[]
Katrine Aspmo Pfaffhuber, Anne-Gunn Hjellbrekke, Hilde Uggerud  and Wenche Aas. 
Analytical intercomparison of inorganic components and heavy metals from 2000 to 2019.
EMEP/CCC-Report 5/2020
\end{list}



\leftline{\bf CEIP Technical and Data reports}
\enlargethispage{\baselineskip}
\begin{list}{}{\setlength{\leftmargin}{15pt}\setlength{\itemindent}{-\leftmargin}}\small

\item[]  
  Bradley Matthews and Robert Wankm\"uller.
  Methodologies applied to the CEIP GNFR gap-filling 2020 for the years 2010-2018. Part I: Main pollutants and Particulate Matter (\nox, NMVOCs, \sox, \nhiii, CO, PM$_{2.5}$, PM$_{10}$, PM$_{coarse}$ and BC),  Technical Report CEIP 1/2020

\item[]
  Stephan Poupa and Robert Wankm\"uller.
  Methodologies applied to the CEIP GNFR gap-filling 2020 for the years 2010-2018. Part II: Heavy Metals (Pb, Cd, Hg),  Technical Report CEIP 1/2020

\item[]
  Stephan Poupa and Robert Wankm\"uller. 
Methodologies applied to the CEIP GNFR gap-filling 2020 for the years 2010-2018. Part III: Persistent organic pollutants (Benzo(a)pyrene, Benzo(b)fluoranthene, Benzo(k)fluoranthene, Indeno(1,2,3-cd)pyrene, Total polycyclic aromatic hydrocarbons, Dioxin and Furan, Hexachlorobenzene, Polychlorinated biphenyls), Technical Report CEIP 1/2020
  
  
\item[]
  Marion Pinterits, Bernhard Ullrich, Katarina Mareckova, Robert  Wankm\"uller and Markus Anys. 
  Inventory review 2020. Review of emission data reported under the LRTAP
Convention and NEC Directive.  Stage 1 and 2 review. Status of gridded
and LPS data. Technical Report CEIP 2/2020

\item[]
  Marion Pinterits, Katarina Mareckova, Robert Wankm\"uller  and Sabine Schindlbacher.
  Methodology report, Technical Report CEIP 3/2020

  

\end{list}


 \leftline{\bf MSC-W Technical and Data reports}

 \enlargethispage{\baselineskip}
 \begin{list}{}{\setlength{\leftmargin}{15pt}\setlength{\itemindent}{-\leftmargin}}\small

 \item[]
 Heiko Klein, Michael Gauss, \'Agnes  Ny\'{\i}ri, Svetlana Tsyro, Hilde Fagerli and Peter Wind. 
Transboundary air pollution by sulphur, nitrogen, ozone and particulate matter in 2018, Country Reports. EMEP/MSC-W Data Note 1/2020


 \end{list}


% ny bok:
%% \leftline{\bf Other associated reports, notes and books in 2012/2013}

%% \begin{list}{}{\setlength{\leftmargin}{15pt}\setlength{\itemindent}{-\leftmargin}}\small

%% \item[]

%% \end{list}


\newpage
\bibliographystyle{copernicus}         % change bibliography-name after each
\renewcommand\bibname{References}      % bibliographystyle command!
\addcontentsline{toc}{section}{References}
\bibliography{Refs,EMEP_Reports}
