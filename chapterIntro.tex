\chapter[Introduction]{Introduction}
\label{ch:Intro}
%DS changed to Ch X.X notation

%\old{TODO}

\section{Purpose and structure of this report}

The mandate of the European Monitoring and Evaluation Programme (EMEP)
is to provide sound scientific support to the Convention on Long-range
Transboundary Air Pollution (LRTAP), particularly in the areas of
atmospheric monitoring and modelling, emission inventories, emission
projections and integrated assessment. Each year EMEP provides
information on transboundary pollution fluxes inside the
EMEP area, relying on information on emission sources and
monitoring results provided by the Parties to the LRTAP Convention.

The purpose of the annual EMEP status reports is to provide an
overview of the status of transboundary air pollution in Europe,
tracing progress towards existing emission control Protocols and
supporting the design of new protocols, when necessary. An additional
purpose of these reports is to identify problem areas, new aspects
and findings that are relevant to the Convention. This year, special attention has been given to the trends of air pollution, supporting the review of the Gothenburg Protocol.
%The progress according to the EMEP Workplan \citep{EMEP:WP2016} is also reported here. Table \ref{Tab:WP} give an overview of which items in the workplan that the different chapters report on.


%2016:
%\begin{table}[!ht]
%\caption{Overview of items from the EMEP workplan 2016-2017 that chapters report progress on. Other chapters report results for the mandatory work.}
%\label{Tab:WP}
%\centering
%\begin{tabular}{l|l}
%\hline\hline
%Chapter & Workplan item \\
%\hline
%\ref{ch:LocalFraction} & 1.1.1.4\\
%\ref{ch:ACTRIS} & XXX\\
%\ref{ch:ShipEmis} & 1.3.3\\
%\ref{ch:slcf}& 1.1.1.20\\
%\ref{ch:ObsDevel} & 1.2.1\\


%2015:
%\begin{table}[!ht]
%\caption{Overview of which items from the EMEP workplan that the chapters report progress on.}
%\label{Tab:WP}
%\centering
%\begin{tabular}{l|l}
%\hline\hline
%Chapter & Workplan item \\
%\hline
%\ref{ch:chapterStatus} & 1.1.4, 1.1.6, 1.1.7, 1.3.2, 1.3.3\\
%\ref{ch:emis2013}      & 1.4.1, 1.4.3, 1.4.4\\
%\ref{ch:Finegrid} & 1.3.4\\
%\ref{ch:EC} & 1.3.8\\
%\ref{ch:condensables} & 1.3.8\\
%\ref{ch:MineralDust} & 1.3.8\\
%\ref{ch:EMEP2014} & 1.1.4, 1.3.8    \\
%\ref{ch:O3Bias} & 1.3.8, 1.6.3\\
%\ref{ch:ModelUpdates}& 1.3.8\\
%\ref{ch:ObsDevel} & 1.1.1, 1.1.2, 1.1.3, 1.1.4, 1.1.5, 1.1.7, 1.1.8\\


%2014:
%\ref{ch:chapterStatus} & 1.1.4, 1.1.6, 1.1.7, 1.3.2\\
%\ref{ch:emis2012}      & 1.4.1, 1.4.3, 1.4.4\\
%\ref{ch:ClimAQ} & 1.3.8 \\
%\ref{ch:Sources}& 1.3.8, 1.1.5, 1.1.9\\
%\ref{ch:slcf} & 1.3.8 \\
%\ref{ch:newgridemi}& 1.3.1\\
%\ref{sc:EMEP grid} & 1.3.4\\
%\ref{ch:Global}& 1.3.10, 1.6.2, 1.6.3\\
%\ref{ch:ModelUpdates}& 1.3.8\\
%\ref{ch:ESX}& 1.3.8\\
%\ref{ch:ObsDevel} & 1.1.1, 1.1.2, 1.1.3, 1.1.4, 1.1.5, 1.1.7, 1.1.8\\

%&\\
%Appendix & Workplan item \\
%\hline
%\ref{ch:appx_emis_2014} & 1.3.2\\
%\ref{ch:appx_sr2014} & 1.3.2\\
%\ref{ch:appx_countryrep_2014} &1.3.2\\
%\ref{ch:appx_modeleval}&1.3.2\\
%\hline\hline
%\end{tabular}
%\vspace{0.05in}
%\end{table}

The present report is divided into four parts. Part I presents the status
of transboundary air pollution with respect to acidification, eutrophication,
ground level ozone and particulate matter in Europe in 2019.
Part II summarizes the work on trends performed to support the review of the Gothenburg Protocol, while Part III deals with technical developments going on within the centres.

Appendix~\ref{ch:appx_emis_2019} in Part IV contains information on the national total emissions of main pollutants and  primary particles for 2019, while Appendix~\ref{ch:appx_emis_trends} shows the  emission trends for the period of 2000-2019. Country-to-country source-receptor matrices with calculations of
the transboundary contributions to pollution in different countries
for 2019 are presented in Appendix~\ref{ch:appx_sr2019}.

Appendix~\ref{ch:appx_countryrep_2019} describes the country
reports which are  issued as a supplement to the EMEP status reports.

%Appendix~\ref{ch:appx_modeleval} introduces the model evaluation
%report for 2018 \citep{WEB2020:Eval} which is available online and contains time series plots
%of acidifying and eutrophying components
%\citep{WEB2020:SN}, ozone \citep{WEB2020:O3} and particulate matter \citep{WEB2020:PM}. These plots are provided for %all stations reporting to
%EMEP (with just a few exclusions due to data-capture or technical problems).
Model evaluation against all EMEP observations are visualized online at XXXXX web address
This online information is complemented by numerical fields and other
information on the EMEP website. The reader is encouraged to visit the
website, \url{http://www.emep.int}, to access this additional information.



\section{Definitions, statistics used}
\label{DEFS}

For sulfur and nitrogen compounds, the basic units used throughout
this report are $\mu$g (S or N)/m$^{3}$ for air concentrations and
mg (S or N)/m$^{2}$ for depositions. Emission data, in particular in
some of the Appendices, is given in Gg (SO$_2$)  and Gg (NO$_2$) in
order to keep consistency with reported values.

For ozone, the basic units used throughout this report are ppb (1 ppb
= 1 part per billion by volume) or ppm (1 ppm = 1000 ppb).  At
20\degrees C and 1013 mb pressure, 1 ppb ozone is equivalent to
2.00~\ug.  \vspace{0.5cm}

\noindent

A number of statistics have been used to describe the distribution of
ozone within each grid square:\\
\begin{description}

     \item[Mean of Daily Max. Ozone] - First we evaluate the maximum
modelled concentration for each day, then we take either 6-monthly
(1 April - 30 September) or annual averages of these values.

     \item[SOMO35] - The Sum of Ozone Means Over 35 ppb is the
     indicator for health impact assessment recommended by WHO. It is
     defined as the yearly sum of the daily maximum of 8-hour running
     average over 35 ppb. For each day the maximum of the running
     8-hours average for O$_3$ is selected and the values over 35 ppb
     are summed over the whole year.

     If we let $A^d_8$ denote the
     maximum 8-hourly average ozone on day $d$, during a year with
     $N_y$ days ($N_y$ = 365 or 366), then SOMO35 can be defined as:

\begin{math}
SOMO35 = \sum_{d=1}^{d=N_y} \max\bigl(A^d_8 - 35 \mbox{\, ppb}, 0.0\bigr)
\end{math}
%SOMO35 = \sum_{days} max\bigl(dailymax(running\_8h\_average\_O_3) - 35 ppb, 0.0\bigr)

where the {\tt max} function evaluates $\max(A-B,0)$ to $A-B$ for $A > B$, or
zero if $A \leq B$, ensuring that only $A^d_8$ values exceeding 35
ppb are included.  The corresponding unit is ppb.days.


\item[POD$_Y$] - Phyto-toxic ozone dose, is the accumulated stomatal ozone flux over a threshold Y, i.e.:

\begin{equation}
   %OLD \mbox{AFstY}_{gen} = \int \max(\Fst - Y, 0)\   dt
   \mbox{POD}_{Y} = \int \max(\Fst - Y, 0)\   dt
\end{equation}

where stomatal flux \Fst, and threshold, $Y$, are in \nmole.
This integral is evaluated over time, from the
start of the growing season (SGS), to the end (EGS).

For the generic crop and forest species, the suffix $gen$ can be
applied, e.g. POD$_{Y, gen}$ is used for forests.
POD was introduced in 2009 as an easier and  more descriptive term for the
accumulated ozone flux (\AFstY was used previously). 
%The definitions of POD are discussed further in \citet{R2010:Fluxes}. 
See also \citet{MillsGCB2011,MillsAE2011,MillsGCB2018a} and \citet{ICP2017}.


     \item[AOT40] -
is the accumulated amount of ozone over the threshold value of 40 ppb, i.e..

\begin{math}
AOT40 = \int \max(O_3 - 40 \mbox{\, ppb}, 0.0) \, dt
\end{math}


where the {\tt max} function ensures that only ozone values exceeding 40
ppb are included.  The integral is taken over time, namely the
relevant growing season for the vegetation concerned, and in some daytime period. The
corresponding unit are ppb.hours (abbreviated to ppb.h).  The usage
and definitions of AOT40 have changed over the years though, and also
differ between UNECE and the EU.
%\cite{MappingManual:Veg} 
\citet{ICP2017}
give the latest definitions for UNECE work, and
describes carefully how AOT40 values are best estimated for local
conditions (using information on real growing seasons for example),
and specific types of vegetation.  In the EU approaches, \ce{O3} concentrations are taken directly from observations (at typically ca. 3~m height), or grid-average 3~m modelled values. In the Mapping Manual
\cite[]{MappingManual:Veg} approaches, there is a strong emphasis on
estimating AOT40 using ozone levels at the top of the vegetation
canopy. Since O$_3$ concentrations can have strong vertical gradients, this approach leads to lower AOT40 estimates than with the EU approach.

The EMEP MSC-W model now generates a number of AOT-related outputs,
in accordance with the recommendations of \cite{ICP2017}, but in this report we will concentrate in this report on four definitions, derived from either `EU' approach or `UNECE' approaches:

\begin{description}
   \item[AOT40c] - AOT40 calculated using EU criteria, from modelled (3~m) or observed ozone, for the assumed crop growing season of May--July. Here we use the EU definitions of day hours as 08:00--20:00.
   
   \item[AT40f] - AOT40 calculated  using EU criteria from modelled 3~m ozone, or observed ozone, for the assumed forest growing season of April--September. Here we use the EU definitions of day hours as 08:00--20:00.
   
   \item[\aotucf] %\item[AOT40$_f^{uc}$]
     - AOT40 calculated for
   forests using estimates of O$_3$ at forest-top ($uc$:
   upper-canopy). This AOT40 is that defined for forests by
   \cite{MappingManual:Veg}, but using a default growing season of
   April-September.
   \item[\aotucc] %\item[AOT40$_c^{uc}$]
     - AOT40
   calculated for agricultural crops using estimates of O$_3$ at the
   top of the crop. This AOT40 is close to that defined for
   agricultural crops by \cite{MappingManual:Veg}, but using a default
   growing season of May-July, and a default crop-height of 1~m.
\end{description}

For \aotucf and \aotucc only daylight hours are included, and for practical
reasons we define daylight in the  model outputs as the time when the
solar zenith angle is equal to or less than 89\degrees. (The proper
UNECE definition uses clear-sky global radiation exceeding 50 W
m$^{-2}$ to define daylight, whereas the EU AOT definitions use day
hours from 08:00-20:00.).

In practice, it is very difficult to convert measured \ce{O3} from an EMEP observation site to the UNECE AOT40 values, since there are no data with which is to estimate the vertical gradient to get to upper-canopy \ce{O3}. Therefore, in the comparison of modelled and observed AOT40s in Ch \ref{ch:chapterStatus} and Ch \ref{ch:Trends}, we have used the EU AOT definitions, since this approach is readily applicable to observed as well as modelled values. We do, however, present source-receptor calculations for the UNECE metrics \aotucf and \aotucc in Appendix~\ref{ch:appx_sr2019}.

The AOT40 levels reflect interest in long-term ozone exposure which is
considered important for vegetation - critical levels of 3~000~ppb.h
have been suggested for agricultural crops and natural vegetation, and
5~000~ppb.h for forests \cite[]{MappingManual:Veg}.
Note that recent  UNECE workshops have recommended that AOT40 concepts
are replaced by ozone flux estimates for crops and forests
\citep[see also][]{ICP2017}.
\end{description}



Furthermore, this report includes concentrations of particulate matter (PM). The basic units throughout this report
are \ug for PM concentrations and the following acronyms are used for different components to PM:

\begin{description}

\item[POA] - primary organic aerosol - which is the organic component of
the PPM emissions (defined below). (POA is in this report assumed to
be entirely in the particle phase, see \citet{R2020:SVOC}.)
  
\item[SOA] - secondary organic aerosol, defined as the aerosol mass
  arising from the oxidation products of gas-phase organic species.

\item[SIA]- secondary inorganic aerosols, defined as the sum of
  sulfate (SO$^{2-}_4$), nitrate (NO$^-_3$) and ammonium (NH$^+_4$).
  In the EMEP MSC-W model SIA is calculated as the sum: SIA= SO$^{2-}_4$
  + NO$^-_3$(fine) + NO$^-_3$(coarse) + NH$^+_4$.

\item[SS] - sea salt.

\item[MinDust] - mineral dust.

\item[PPM] - primary particulate matter, originating directly from
  anthropogenic emissions. One usually distinguishes between fine
  primary particulate matter, PPM$_{2.5}$, with aerosol diameters
  below 2.5 $\mu$m and coarse primary particulate matter, PPM$_{coarse}$
  with aerosol diameters between 2.5 $\mu$m and 10 $\mu$m.

\item[PM$_{2.5}$] - particulate matter with aerodynamic diameter
  up to 2.5 $\mu$m. In the EMEP
  MSC-W model, PM$_{2.5}$ is calculated as PM$_{2.5}$ = SO$^{2-}_4$
  + NO$^-_3$(fine) + NH$^+_4$ + SS$_{2.5}$ + MinDust(fine)
  + SOA(fine) + PPM$_{2.5}$ + 0.13 $\cdot$ NO$^-_3$(coarse) + PM25water.
  (PM25water = PM associated water).

\item[PM$_{\text{coarse}}$] - coarse particulate matter with aerodynamic
  diameter between 2.5$\mu$m 
  and 10$\mu$m. In the EMEP MSC-W model PM$_{\text{coarse}}$ is calculated
  as PM$_{\text{coarse}}$ = 0.87 $\cdot$ NO$^-_3$(coarse)+ SS(coarse)
  + MinDust(coarse) + PPM$_{coarse}$.

\item[PM$_{10}$] - particulate matter with aerodynamic diameter
  up to 10 $\mu$m. In the EMEP
   MSC-W model PM$_{10}$ is calculated as PM$_{10}$ = PM$_{2.5}$
  + PM$_{\text{coarse}}$.

\item[SS$_{10}$]  - sea salt aerosol with diameter
  up to 10 $\mu$m.
  
\item[SS$_{2.5}$]  - sea salt aerosol with diameter
  up to 2.5 $\mu$m.

\end{description}

In addition to bias, correlation and root mean square the statistical
parameter, index of agreement, are used to judge the model's agreement
with measurements:\\
\begin{description}
%     \item[Bias] - $\frac{\overline{Mod}-\overline{Obs}}{\overline{Obs}}\times
%     100\%$ measured yearly average (Obs), modelled yearly average (Mod)
%     \item[Corr] - Correlation between observation and model for station
%     yearly averages.
     \item[IOA]  - The index of agreement (IOA) is defined as follows
     \citep{Willmott1981, Willmott1982}:
%     \vspace{0.3in}
%     \begin{math}
\begin{equation}
IOA=1-\frac{\sum_{i=1}^{N}(m_i-o_i)^2}{\sum_{i=1}^{N}(|m_i-\bar{o}|+|o_i-\bar{o}|)^2}
\label{eq:IOA}
\end{equation}
%     \end{math}\\
     where $\overline{o}$ is the average observed value. Similarly to
     correlation, IOA can be used to assess agreement either
     spatially or temporally.
     When IOA is used in a spatial sense, N denotes the number of stations
     with measurements at one specific point in time, and $m_i$ and $o_i$
     are the modelled and observed values at station $i$.
     For temporal IOA, N denotes the number of time steps with measurements,
     while $m_i$ and $o_i$ are the modelled and observed value at time step $i$.
     IOA varies between 0 and 1. A value of 1 corresponds to perfect agreement
     between model and observations, and 0 is the theoretical minimum.

\end{description}


\section{The EMEP grid}
\label{EMEPgrid}

At the 36$^{th}$ session of the EMEP Steering Body the EMEP Centres suggested 
to increase spatial resolution and projection of reported emissions from 50$\times$50~km$^{2}$ polar stereographic grid to {0.1\degrees $\times$0.1\degrees} longitude-latitude grid in a geographic coordinate system 
(WGS84). The EMEP domain shown in Figure~\ref{fig:lonlatgrid} covers 
the geographic area between 30\degrees N-82\degrees N latitude and 30\degrees 
W-90\degrees E longitude. This domain 
represents a balance between political needs, scientific needs and technical 
feasibility. Parties are obliged to report gridded emissions in this grid resolution from year 2017.


\begin{figure}[h]
\centering
\includegraphics*[viewport=25 150 570 680,clip,scale=0.5]{FIGS_INTRO/lonlatgrid.pdf}
\caption{The EMEP domain covering the geographic area between 30\degrees N-82\degrees N latitude and 30\degrees 
W-90\degrees E longitude.}
\label{fig:lonlatgrid}
\end{figure}


The higher resolution means an increase of grid cells from approximately 
21500 cells in the 50$\times$50 km$^2$ grid to 624000 cells in the  {0.1\degrees $\times$0.1\degrees} longitude-latitude grid.

\subsection{The reduced grid: EMEP0302}

For practical purposes, a coarser grid has also been defined. The EMEP0302 grid covers the same region as the  {0.1\degrees $\times$0.1\degrees} longitude-latitude EMEP domain (Figure~\ref{fig:lonlatgrid}), but the spatial resolution is 0.3{\degrees} in the longitude direction and  0.2{\degrees} in the latitude direction. Each gridcell from the EMEP0302 grid covers exactly 6 gridcells from the {0.1\degrees $\times$0.1\degrees} official grid. %A gridcell is approximatively square at 48.2 \degrees latitude = acos(2/3) 


\begin{table}[!ht]
\begin{center}
\begin{small}
\begin{tabular}{|l|l|c|l|l|}
\cline{1-2} \cline{4-5}
{\rule[-3mm]{0mm}{8mm}\textbf{Code}}&\textbf{Country/Region/Source}&&\textbf{Code}&\textbf{Country/Region/Source}\\  \cline{1-2} \cline{4-5}
AL & Albania & & IS & Iceland \\  \cline{1-2} \cline{4-5}
AM & Armenia & &  IT & Italy\\  \cline{1-2} \cline{4-5}
AST & Asian areas & &  KG & Kyrgyzstan \\  \cline{1-2} \cline{4-5}
AT & Austria & &  KZ & Kazakhstan \\  \cline{1-2} \cline{4-5}
ATL & N.-E. Atlantic Ocean & & LI & Liechtenstein \\  \cline{1-2} \cline{4-5}
AZ & Azerbaijan & & LT & Lithuania\\  \cline{1-2} \cline{4-5}
BA & Bosnia and Herzegovina & & LU & Luxembourg \\  \cline{1-2} \cline{4-5}
BAS & Baltic Sea & & LV & Latvia \\  \cline{1-2} \cline{4-5}
BE & Belgium & &  MC & Monaco\\  \cline{1-2} \cline{4-5}
BG & Bulgaria & & MD & Moldova \\  \cline{1-2} \cline{4-5}
BIC & Boundary/Initial Conditions & & ME & Montenegro \\  \cline{1-2} \cline{4-5}
BLS & Black Sea & & MED & Mediterranean Sea \\  \cline{1-2} \cline{4-5}
BY & Belarus & & MK & North Macedonia \\  \cline{1-2} \cline{4-5}
CH & Switzerland & & MT & Malta \\  \cline{1-2} \cline{4-5}
CY & Cyprus & & NL & Netherlands \\  \cline{1-2} \cline{4-5}
CZ & Czechia & & NO & Norway \\  \cline{1-2} \cline{4-5}
DE & Germany & & NOA & North Africa \\  \cline{1-2} \cline{4-5}
DK & Denmark & & NOS & North Sea \\  \cline{1-2} \cline{4-5}
DMS & Dimethyl sulfate (marine) & & PL & Poland \\  \cline{1-2} \cline{4-5}
EE & Estonia & & PT & Portugal \\  \cline{1-2} \cline{4-5}
ES & Spain & & RO & Romania \\  \cline{1-2} \cline{4-5}
EU & European Union (EU28) & & RS & Serbia \\  \cline{1-2} \cline{4-5}
EXC & EMEP land areas & & RU & Russian Federation  \\  \cline{1-2} \cline{4-5}
FI & Finland & & SE & Sweden \\  \cline{1-2} \cline{4-5}
FR & France & &  SI & Slovenia\\  \cline{1-2} \cline{4-5}
GB & United Kingdom & & SK & Slovakia\\  \cline{1-2} \cline{4-5}
GE & Georgia & & TJ & Tajikistan\\  \cline{1-2} \cline{4-5}
GL & Greenland & & TM & Turkmenistan \\  \cline{1-2} \cline{4-5}
GR & Greece & & TR & Turkey \\  \cline{1-2} \cline{4-5}
HR & Croatia & & UA & Ukraine \\  \cline{1-2} \cline{4-5}
HU & Hungary & & UZ & Uzbekistan \\  \cline{1-2} \cline{4-5}
IE & Ireland & & VOL & Volcanic emissions \\  \cline{1-2} \cline{4-5}
\end{tabular}
\end{small}
\caption{Country/region codes used throughout this report.}
\label{tab:countries}
\vspace{0.05in}
\end{center}


\end{table}

\section{Country codes}

Several tables and graphs in this report make use of codes to denote
countries and regions in the EMEP area. Table~\ref{tab:countries}
provides an overview of these codes and lists the countries and
regions included.

All 51 Parties to the LRTAP Convention, except two, are included in the
analysis presented in this report. The Parties that are excluded of the
analysis are Canada and the United States of America, because
they lie outside the EMEP domain.

%% Monaco and Liechtenstein are
%% excluded because their emissions and geographical extents are
%% below the accuracy of the present source-receptor calculations in
%% 50$\times$50km$^{2}$.

%% Malta, Monaco and Liechtenstein are introduced as a receptor country. However,
%% the estimated emissions
%% from Malta are below the accuracy limit of the source-receptor calculations
%% and do not justify a separate study of Malta as an emitter country.




\section{Other publications}
\label{sec:publ}
This report is complemented by a report on EMEP MSC-W model performance for acidifying and eutrophying components, photo-oxidants and particulate matter in 2018 \citep{WEB2020:Eval}, made available online, at \url{www.emep.int}.

%% This report is complemented by the country specific reports on the
%% 2014 status of transboundary acidification, eutrophication, ground level
%% ozone and PM (see Apendix~\ref{ch:appx_countryrep_2014}). Both English
%% and Russian versions of the country reports are available to the twelve EECCA
%% countries.

%As noted above, time series plots of acidifying and eutrophying components
%\citep{WEB2017:SN}, and ozone and NO$_2$ \citep{WEB2017:O3} have been made
%available online, at \url{www.emep.int}.% along with much other material.

A list of all associated technical reports and notes  by the EMEP
centres in 2020 (relevant for transboundary acidification, eutrophication,
ozone and particulate matter) follows at the end of this section.

%\newpage
\subsection*{Peer-reviewed publications in 2020}

The following scientific papers of relevance to transboundary acidification, eutrophication, ground level ozone and particulate matter, involving EMEP/MSC-W and EMEP/CCC staff, have become available in 2020:

\enlargethispage{\baselineskip}
\begin{list}{}{\setlength{\leftmargin}{15pt}\setlength{\itemindent}{-\leftmargin}}\small
%%%%%%%%%%%%%%%%%%%%%%%%%%%%%%%%%%%%%%%%%%%%%%%%%%%%%%%%%%%%%%%%%%%%%%%%
\item[]
Amann, M.; Kiesewetter, G.; Schöpp, Wolfgang; Klimont, Zbigniew; Winiwarter, Wilfried; Cofala, Janusz; Rafaj, Peter; Hoglund-Isaksson, Lena; Gomez-Sabriana, Adriana; Heyes, Chris; Purohit, Pallav; Borken-Kleefeld, Jens; Wagner, Fabian; Sander, Robert; Fagerli, Hilde; Nyiri, Agnes; Cozzi, Laura; Pavarini, Claudia.
Reducing global air pollution: The scope for further policy interventions: Achieving clean air worldwide.
Philosophical Transactions of the Royal Society A: Mathematical, Physical and Engineering Sciences2020; 378.(2183)
DOI: 10.1098/rsta.2019.0331

\item[]
Blechschmidt, Anne-Marlene; Arteta, Joaquim; Coman, Adriana; Curier, Lyana; Eskes, Henk; Foret, Gilles; Gielen, Clio; Hendrick, François; Marécal, Virginie; Meleux, Frédérik; Parmentier, Jonath-an; Peters, Enno; Pinardi, Gaia; Piters, Ankie J.M.; Plu, Matthieu; Richter, Andreas; Segers, Arjo; Sofiev, Mikhail; Valdebenito, Alvaro; Van Roozendael, Michel; Vira, Julius; Vlemmix, Tim; Burrows, John P..
Comparison of tropospheric NO2 columns from MAX-DOAS retrievals and regional air quality model simulations.
Atmospheric Chemistry and Physics ; 2020; 20 p. 2795-2823
DOI: 10.5194/acp-20-2795-2020 

\item[]
Collaud Coen, M., Andrews, E., Alastuey, A., Arsov, T. P., Backman, J., Brem, B. T., Bukowiecki, N., Couret, C., Eleftheriadis, K., Flentje, H., Fiebig, M., Gysel-Beer, M., Hand, J. L., Hoffer, A., Hooda, R., Hueglin, C., Joubert, W., Keywood, M., Kim, J. E., Kim, S.-W., Labuschagne, C., Lin, N.-H., Lin, Y., Lund Myhre, C., Luoma, K., Lyamani, H., Marinoni, A., Mayol-Bracero, O. L., Mihalopoulos, N., Pandolfi, M., Prats, N., Prenni, A. J., Putaud, J.-P., Ries, L., Reisen, F., Sellegri, K., Sharma, S., Sheridan, P., Sherman, J. P., Sun, J., Titos, G., Torres, E., Tuch, T., Weller, R., Wiedensohler, A., Zieger, P., and Laj, P.
Multidecadal trend analysis of in situ aerosol radiative properties around the world.
Atmos. Chem. Phys., 20, 8867–8908, 2020.
DOI: 10.5194/acp-20-8867-2020 

\item[]
Denby, Bruce; Gauss, Michael; Wind, Peter; Mu, Qing; Wærsted, Eivind Grøtting; Fagerli, Hilde; Valdebenito Bustamante, Alvaro Moises; Klein, Heiko.
Description of the uEMEP\_v5 downscaling approach for the EMEP MSC-W chemistry transport model.
Geoscientific Model Development ; 2020; 13.(12); p. 6303-6323
DOI: 10.5194/gmd-13-6303-2020 

\item[]
Etzold, Sophia; Ferretti, Marco; Reinds, Gert-Jan; Solberg, Svein; Gessler, Arthur; Waldner, Peter; Schaub, Marcus; Simpson, David; Benham, Sue; Hansen, Karin; Ingerslev, Morten; Jonard, Mathieu; Karlsson, Per Erik; Lindroos, Antti-Jussi; Marchetto, Aldo; Manninger, Miklos; Meesenburg, Henning; Merilä, Päivi; Nöjd, Pekka; Rautio, Pasi; Sanders, Tanja GM; Seidling, Walter; Skudnik, Mitja; Thimonier, Anne; Verstraeten, Arne; Vesterdal, Lars; Vejpustkova, Monika; de Vries, Wim.
Nitrogen deposition is the most important environmental driver of growth of pure, even-aged and managed European forests.
Forest Ecology and Management ; 2020; 458; p. 1-13
DOI: 10.1016/j.foreco.2019.117762

\item[]
Flechard, Chris R.; Ibrom, Andreas; Skiba, Ute; de Vries, Wim; Van Oijen, Marcel; Cameron, David R.; Dise, Nancy B.; Korhonen, Janne; Buchmann, Nina; Legout, Arnaud; Simpson, David; Sanz, Maria J.; Aubinet, Marc; Loustau, Denis; Montagnani, Leonardo; Neirynck, Johan; Janssens, Ivan A.; Pihlatie, Mari; Kiese, Ralf; Siemens, Jan; Francez, Andre-Jean; Augustin, Jurgen; Varlagin, Andrej; Olejnik, Janusz; Juszczak, Radoslaw; Aurela, Mika; Berveiller, Daniel; Chojnicki, Bogdan H.; Dämmgen, Urich; Delpierre, Nicolas; Djuricic, Vesna; Drewer, Julia; Dufrene, Eric; Eugster, Werner; Fauvel, Yannick; Fowler, David; Frumau, Arnoud; Granier, Andre; Gross, Patrick; Hamon, Yannick; Helfter, Carole; Hensen, Arjan; Horvath, Laszlo; Kitzler, Barbara; Kruijt, Bart; Kutsch, Werner; Lobo-do-Vale, Raquel; Lohila, Annalea; Longdoz, Bernard; Marek, Michal V.; Matteucci, Giorgio; Mitosinkova, Marta; Moreaux, Virginie; Neftel, Albrecht; Ourcival, Jean-Marc; Pilegaard, Kim; Pita, Gabriel; Sanz, Francisco; Schjoerring, Jan K.; Sebastià, Maria-Teresa; Tang, Y. Sim; Uggerud, Hilde Thelle; Urbaniak, Marek; van Dijk, Netty; Vesala, Timo; Vidic, Sonja; Vincke, Caroline; Weidinger, Tamas; Sechmeister-Boltenstern, Sophie; Butterbach-Bahl, Klaus; Nemitz, Eiko; Sutton, Mark A..
Carbon–nitrogen interactions in European forests and semi-natural vegetation – Part 1: Fluxes and budgets of carbon, nitrogen and greenhouse gases from ecosystem monitoring and modelling.
Biogeosciences ; 2020; 17; p. 1583-1620
DOI: 10.5194/bg-17-1583-2020

\item[]
Jähn, Michael; Kuhlmann, Gerrit; Mu, Qing; Haussaire, Jean-Matthieu; Ochsner, David; Osterried, Katherine; Clément, Valentin; Brunner, Dominik.
An online emission module for atmospheric chemistry transport models: Implementation in COSMO-GHG v5.6a and COSMO-ART v5.1-3.1.
Geoscientific Model Development ; 2020; 13.(5); p. 2379-2392
DOI: 10.5194/gmd-13-2379-2020

\item[]
Jonson, Jan Eiof; Gauss, Michael; Schulz, Michael; Jalkanen, Jukka-Pekka; Fagerli, Hilde.
Effects of global ship emissions on European air pollution levels.
Atmospheric Chemistry and Physics 2020; 20, p. 11399–11422,
DOI: 10.5194/acp-20-11399-2020

\item[]
Korsakissok, I.; Périllat, R.; Andronopoulos, S.; Bedwell, P.; Berge, Erik; Charnock, T.; Geertsema, G.; Gering, F.; Hamburger, Thomas; Klein, Heiko; Leadbetter, S.; Lind, Ole Christian; Pázmándi, T.; Rudas, Cs.; Salbu, Brit; Sogachev, A.; Syed, Naeem Ul Hasan; Tomas, J.M.; Ulimoen, Magnus; de Vries, H.; Wellings, J..
Uncertainty propagation in atmospheric dispersion models for radiological emergencies in the pre- and early release phase: Summary of case studies.
Radioprotection - Revue de la Societé Francaise de Radioprotection ; 2020; 55; p. S57-S68
DOI: 10.1051/radiopro/2020013 

\item[]
Leadbetter, Susan J.; Andronopoulos, Spyros; Bedwell, Peter; Chevalier-Jabet, Karine; Gertie, Geertsema; Gering, Florian; Hamburger, Thomas; Jones, Andrew R.; Klein, Heiko;Korsakissok, Irene; Mathieu, Anne; Pázmándi, Tamas; Périllat, Raphael; Csilla, Rudas; Sogachev, Andrey; Szántó, Peter; Tomas, Jasper M.; Twenhöfel, Chris; de Vries, Hans; Wellings, Joseph.
Ranking uncertainties in atmospheric dispersion modelling following the accidental release of radioactive material.
Radioprotection - Revue de la Societé Francaise de Radioprotection 2020; 55.(HS1); p. 51-55
DOI: 10.1051/radiopro/2020012 

\item[]
Liland, Astrid; Lind, Ole Christian; Bartnicki, Jerzy; Brown, Justin; Dyve, Jan Erik; Iosjpe, Mikhail; Klein, Heiko; Lin, Yan; Simonsen, Magne; Strand, Per; Thørring, Håvard; Ytre-Eide, Martin Album; Salbu, Brit.
Using a chain of models to predict health and environmental impacts in Norway from a hypothetical nuclear accident at the Sellafield site.
Journal of Environmental Radioactivity ; 2020; 214-215; 106159
DOI: 10.1016/j.jenvrad.2020.106159

\item[]
Lee, H., Lee, K., Lunder, C. R., Krejci, R., Aas, W., Park, J., Park, K.-T., Lee, B. Y., Yoon, Y. J., and Park, K.
Atmospheric new particle formation characteristics in the Arctic as measured at Mount Zeppelin, Svalbard, from 2016 to 2018, 
Atmos. Chem. Phys., 20, 13425–13441,2020.
DOI:10.5194/acp-20-13425-2020

\item[]
Mortier, A., Gliß, J., Schulz, M., Aas, W., Andrews, E., Bian, H., Chin, M., Ginoux, P., Hand, J., Holben, B., Zhang, H., Kipling, Z., Kirkevåg, A., Laj, P., Lurton, T., Myhre, G., Neubauer, D., Olivié, D., von Salzen, K., Skeie, R. B., Takemura, T., and Tilmes, S.
Evaluation of climate model aerosol trends with ground-based observations over the last 2 decades – an AeroCom and CMIP6 analysis, 
Atmos. Chem. Phys., 20, 13355–13378, 2020.
DOI:10.5194/acp-20-13355-2020

\item[]
Mwase, Nandi S.; Ekstrøm, Alicia; Jonson, Jan Eiof; Svensson, Erik; Jalkanen, Jukka-Pekka; Wichmann, Janine; Molnar, Peter; Stockfelt, Leo.
Health Impact of Air Pollution from Shipping in the Baltic Sea: Effects of Different Spatial Resolutions in Sweden.
International Journal of Environmental Research and Public Health (IJERPH) 2020; 17.(21)
DOI: 10.3390/ijerph17217963 

\item[]
Evangeliou, N., Grythe, H., Klimont, Z.  Heyes,C. , Eckhardt, S., Lopez-Aparicio, S., Stohl,A. 
Atmospheric transport is a major pathway of microplastics to remote regions. 
Nat Commun 11, 3381 (2020). 
DOI: 10.1038/s41467-020-17201-9

\item[]
Pommier, Matthieu; Fagerli, Hilde; Schulz, Michael; Valdebenito, Alvaro; Kranenburg, Richard; Schaap, Martijn.
Prediction of source contributions to urban background PM10 concentrations in European cities: A case study for an episode in December 2016 using EMEP/MSC-W rv4.15 and LOTOS-EUROS v2.0 - Part 1: The country contributions.
Geoscientific Model Development ; 2020; 13.(4) ; p. 1787-1807
DOI: 10.5194/gmd-13-1787-2020 

\item[]
Simpson, David; Bergström, Robert; Briolat, Alan; Imhof, Hannah; Johansson, John; Priestley, Michael;
Valdebenito Bustamante, Alvaro Moises.
GenChem v1.0 – a chemical pre-processing and testing system for atmospheric modelling.
Geoscientific Model Development 2020 ; 13. p. 6447-6465
DOI: 10.5194/gmd-13-6447-2020 

\item[]
Sørensen, Jens Havskov; Bartnicki, Jerzy; Blixt Buhr, Anna Maria; Feddersen, Henrik; Hoe, Steen Cordt; Israelson, Carsten; Klein, Heiko; Lauritzen, Bent; Lindgren, Jonas; Schönfeldt, Fredrik; Sigg, Robert.
Uncertainties in atmospheric dispersion modelling during nuclear accidents.
Journal of Environmental Radioactivity 2020; 222. –
DOI: 10.1016/j.jenvrad.2020.106356 

\item[]
van der Swaluw, Eric; de Vries, Wilco; Kruit, Roy Wichink; Aben, Jan; Vieno, Massimo; Fagerli, Hilde; Wind, Peter; van Pul, Addo.
Trend Analysis of Air Pollution and Nitrogen Deposition Over the Netherlands Using the EMEP4NL and OPS Model.
Springer Proceedings in Complexity 2020 ; p.. 47-51
DOI: 10.1007/978-3-030-22055-6\_8 

\item[]
Wind, Peter; Denby, Bruce; Gauss, Michael.
Local fractions-a method for the calculation of local source contributions to air pollution, illustrated by examples using the EMEP MSC-W model (rv4\_33).
Geoscientific Model Development ; 2020; 13.(3); p. 1623-1634
DOI: 10.5194/gmd-13-1623-2020	Projects: NFR 267734 (AirQuip) 

\item[]
Zhang, Yuqiang; West, Jason; Emmons, Louisa K.; Flemming, Johannes; Jonson, Jan Eiof; Tronstad Lund, Marianne; Sekija, Takashi; Sudo, Kengo; Gaudel, Audrey; Chang, Kai-Lan; Nédélec, Philippe; Thouret, Valérie.
Contributions of World Regions to the Global Tropospheric Ozone Burden Change from 1980 to 2010.
Geophysical Research Letters 2020; 48 (1)
DOI: 10.1029/2020 GL089184

\end{list}


\subsection*{Associated EMEP reports and notes in 2021}

\leftline{\bf Joint reports}
\vspace{0.5cm}

\enlargethispage{\baselineskip}
\begin{list}{}{\setlength{\leftmargin}{15pt}\setlength{\itemindent}{-\leftmargin}}\small
% +
\item[]
Transboundary particulate matter, photo-oxidants, acidification and eutrophication components. Joint MSC-W \& CCC \& CEIP Report. EMEP Status Report 1/2021

\item[] EMEP MSC-W model performance for acidifying and
  eutrophying components, photo-oxidants and particulate matter in
  2019. Supplementary material to EMEP Status Report 1/2021

\item[]
 Assessment of transboundary pollution by toxic substances: Heavy metals and POPs.  Joint MSC-E  \& CCC  \& CEIP  \& INERIS Report. EMEP Status Report 2/2021

\end{list}


%% \leftline{\bf CIAM Technical and Data reports}

%% \enlargethispage{\baselineskip}
%% \begin{list}{}{\setlength{\leftmargin}{15pt}\setlength{\itemindent}{-\leftmargin}}\small

%% \item[] Wagner F., Winiwarter W., Klimont, Z., Amann, M., Sutton, M.
%% Ammonia reductions and costs implied by the three ambition levels
%% proposed in the Draft Annex IX to the Gothenburg protocol.
%% CIAM 5/2011–2 May, 2012


%% \item[] Amann, M., Bertok, I., Borken-Kleefeld, J., Cofala, J., Heyes, C.,
%%  H\"oglund-Isaksson, L., Klimont, Z., Rafaj, P., Sch\"opp, W., and Wagner, F.
%% Environmental Improvements of the Revision of the Gothenburg Protocol.
%% CIAM 1/2012.

%% \end{list}


\leftline{\bf CCC Technical and Data reports}

\enlargethispage{\baselineskip}
\begin{list}{}{\setlength{\leftmargin}{15pt}\setlength{\itemindent}{-\leftmargin}}\small
\item[]
Anne-Gunn Hjellbrekke. 
Data Report 2019. Particulate matter, carbonaceous and inorganic compounds.
EMEP/CCC-Report 1/2021

\item[]
Anne-Gunn Hjellbrekke and Sverre Solberg. 
Ozone measurements 2019. 
EMEP/CCC-Report 2/2021

\item[]
Wenche Aas and Pernilla Bohlin Nizzetto. 
Heavy metals and POP measurements 2019.
EMEP/CCC-Report 3/2021

\item[]
Sverre Solberg, Anja Claude and Stefan Reimann. 
VOC measurements 2019. 
EMEP/CCC-Report 4/2021

\end{list}



\leftline{\bf CEIP Technical and Data reports}
\enlargethispage{\baselineskip}
\begin{list}{}{\setlength{\leftmargin}{15pt}\setlength{\itemindent}{-\leftmargin}}\small

\item[]  
Sabine Schindlbacher, Bradly Matthews and Bernhard Ullrich. Uncertainties and recalculations of emission inventories submitted under CLRTAP, Technical Report CEIP 1/2021

\item[]
Bradley Matthews and Robert Wankmueller. Part I:  Main pollutants , Particulate Matter and BC (NOx, NMVOCs, SOx, NH3, CO, PM2.5, PM10, PMcoarse, BC), Technical Report CEIP 2/2021

\item[]
 Katarina Mareckova, Marion Pinterits, Bernhard Ullrich,  Robert Wankmueller, Thomas Bartmann and Sabine Schindlbacher. Inventory Review 2021,  Technical Report CEIP 3/2021
  
\item[]
Katarina Mareckova, Robert Wankmueller, Marion Pinterits, Bernhard Ullrich and Sabine Schindlbacher. Methodology report, Technical Report CEIP 4/2021

\end{list}


 \leftline{\bf MSC-W Technical and Data reports}

 \enlargethispage{\baselineskip}
 \begin{list}{}{\setlength{\leftmargin}{15pt}\setlength{\itemindent}{-\leftmargin}}\small

 \item[]
 Heiko Klein, Michael Gauss, \'Agnes  Ny\'{\i}ri, Svetlana Tsyro, Hilde Fagerli and Peter Wind. 
Transboundary air pollution by sulfur, nitrogen, ozone and particulate matter in 2019, Country Reports. EMEP/MSC-W Data Note 1/2021


 \end{list}


% ny bok:
%% \leftline{\bf Other associated reports, notes and books in 2012/2013}

%% \begin{list}{}{\setlength{\leftmargin}{15pt}\setlength{\itemindent}{-\leftmargin}}\small

%% \item[]

%% \end{list}


\newpage
\bibliographystyle{copernicus}         % change bibliography-name after each
\renewcommand\bibname{References}      % bibliographystyle command!
\addcontentsline{toc}{section}{References}
%\bibliography{Refs,EMEP_Reports}
\bibliography{Refs2021}
